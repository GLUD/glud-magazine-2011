% Esta obra está bajo una licencia Reconocimiento 2.5 Espańa de Creative
% Commons. Para ver una copia de esta licencia, visite 
% http://creativecommons.org/licenses/by/2.5/es/
% o envie una carta a Creative Commons, 171 Second Street, Suite 300, 
% San Francisco, California 94105, USA.

% Seccion Introducción
%

\rput(2.5,-2.3){\resizebox{!}{5.7cm}{{\epsfbox{images/mi_articulo/stallman.eps}}}}%Imagen de el comienzo de el articulo, coordenadas desde 
                                                                                   %la parte superior izquierda del margen de la pagina

% -------------------------------------------------
% Cabecera
\begin{flushright}
\msection{introcolor}{black}{0.25}{HABLANDO CON EL PADRE} %titulo de la sección

\mtitle{10cm}{Copyright Vs Comunidad en la Era de las Redes Informáticas} %titulo del articulo 

\msubtitle{8cm}{Conferencia de Richard Stallman} %subtitulo

{\sf Transcrpción, Foto, Introduccion y subrayas por Jinny Salcedo} %autor

{\psset{linecolor=black,linestyle=dotted}\psline(-12,0)}
\end{flushright}

\vspace{2mm}
% -------------------------------------------------

\begin{multicols}{2}


% Introducción
\intro{introcolor}{E}{sta es la transcripción de la conferencia que dio el Doctor Richard Stallman en la Universidad Nacional de Colombia en el marco de una giro que hizo por Colombia, ya que tuve la oportunidad de participar, y ya que aporta grandes cosas al articulo de opinión que estaba escribiendo decidí hacer este trabajo. También decidí hacerlo por que parece importante difundir las ideas que tiene este gran personaje, a cerca no solo de los software, si no también de otros aspectos de la vida, que casi nunca veces no se encuentras en sus publicaciones.
}

\vspace{2mm}

% Cuerpo del artículo

Confieso que no tenia muchas expectativas al entrar a la conferencia ya que Richard me parecía una persona (por las cosas que había leído de el) que hablaba de libertad solo en la parte técnica, en la parte de código, y estuve así la mayor parte de la conferencia, como se darán cuenta al final le hice una pregunta que ese Richard Stallman que yo tenia en mi imaginario, no abría podido contestar con la contundencia que lo hizo. Pero salí del auditorio con una idea totalmente diferente de el, eso lo pueden comprobar mis compańeros del GLUD que llegaron a decir que siendo yo uno de los mas reacios a las teorías de Stallman me había logrado " convertir ", y creo que es cierto.

Quiero hacer la aclaración que mi trabajo fue solo el de transcribir la conferencia lo mas fielmente que pude, a petición del propio Stallman ya que pidió compartirla sin obras derivadas ya que era una obra de su opinión (aunque la misma transcripción es una obra derivada), y por tal motivo tal vez para los mas juiciosos de la lengua castellana tenga muchos errores de todo tipo, de todos modos también es de poner en juego que siendo estadounidense tiene un manejo envidiable de nuestro idioma, sin mas preámbulo aquí va.\\

\sectiontext{white}{black}{LA CONFERENCIA} %como se hace una sección

Buenas Noches, si sacas fotos de mi por favor no las pongas en Fecebook, facebook es un motor de vigilancia, y si pones las foto de alguien en facebook, facebook se aprovecha de la foto para vigilarlo mas, pide que alguien ponga el nombre de la persona, y así la base de datos de facebook tiene mas datos sobre el, podemos disputar si poner la foto de tu amigo es buen trato amigable de tu amigo, pero solo pido que no pongas fotos de mi en facebook, otra petición si grabas la conferencia y quieres {\em {\color{introcolor}{distribuir copias por favor únicamente en los formatos favorables al Software Libre}}}, es decir los formatos ogg u otros libres, no en mp-nada por que son formatos patentados, no en flash por que flash suele exigir el uso de programas privativos por parte del usuario para mirar o escuchar y tampoco en vídeo player, Qick time o Windows media player, y por favor con la licencia CC no derivados , por que es una obra de mi opinión.
Esta conferencia responde a una pregunta que aveces me preguntan al final de una conferencia sobre el Software Libre, me preguntas si las ideas de Software Libre se aplican a otras cosas,  para que esta pregunta tenga sentido tengo que explicar las ideas del Software Libre brevemente.
Software Libre libre quiere decir Software que respeta la libertad de la comunidad de los usuarios y la idea es que si un programa no es libre su distribución es injusta y no debería existir, el software privativo que priva de la libertad a sus usuarios es una injusticia y tendremos que escaparnos de no ser víctimas de esta injusticia y la meta ultima es eliminar esta injusticia.
{\em {\color{introcolor}{Con el software solo hay dos posibilidades o los usuarios tienen el control del programa o el programa tiene el control de los usuarios,}}} para tener el control efectivamente del programa los usuarios necesitan unas libertades, por eso es apropiado llamarlo Software Libre, hay cuatro libertades esenciales, la la libertad cero es la de ejecutar el programa como quieras, la libertad uno es al de estudiar el código fuente del programa y cambiarlo para que haga tu informática como quieras; con esas dos libertades cada usuario tiene el control individual del programa, pero el control individual no basta, por que muchos usuarios no saben programar, no son capaces de ejercer la libertad uno pero tampoco par aun programador como yo basta la libertad uno  por que hay tanto Software Libre libre en el mundo, que ningún usuario es capaz de estudiar y comprender   todo el código fuente de los programas que usa, ni de escribir personalmente todos los cambios que desea entonces no podemos hacer lo que realmente queremos con el solo control individual, hace falta también el control colectivo que necesita dos libertades mas, la libertad dos es la de ayudar a los demás de redistribuir copias exactas del programa como lo has recibido y la libertad tres es la de contribuir a la comunidad es la libertad de distribuir copias de tus versiones cambiadas, en ambos casos cuando quieras por que ninguna

%%%%%%%%%%%%%%%%%%%%%%%%%%%%%%%%%%%%%%%%%%%%%%%%%%%%%%%%%%%%
\ebOpage{introcolor}{0.35}{HABLANDO CON EL PADRE}
%%%%%%%%%%%%%%%%%%%%%%%%%%%%%%%%%%%%%%%%%%%%%%%%%%%%%%%%%%%%
de estas cuatro libertades es obligatoria, tienes la libertad de hacer estas cosa si quieres pero no tienes la obligación de hacerlas.


%%%%%%%%%
\begin{entradilla} %codigo para una entradilla
{\em {\color{introcolor}{Libertades esenciales: }}}cero es la de ejecutar, uno la de estudiar el código, dos es la de ayudar a los demás, la de contribuir a la comunidad.
\end{entradilla}

%%%%%%%%%
%%%%%%%%%
Entonces con todas las cuatro libertades cualquier grupo de usuarios que quieran colaborar, pueden colaborar manteniendo su versión para su uso y quizás distribución, entonces el control colectivo es mucho mas amplio que el solo control al nivel de la totalidad de un país, hace falta que {\em {\color{introcolor}{cualquier grupo pueda tener el control de la versión que usa, }}} pero si los usuarios no tienen estas cuatro libertades entonces no tiene el control efectivo adecuado del programa, entonces es el programa el que tiene el control de los usuarios, pero siempre hay alguna entidad que tiene el control del programa y a través del programa somete a sus usuarios, el programa no libre le llamamos  privativo por que priva de su libertad a sus usuarios, es un yugo, {\em {\color{introcolor}{un instrumento de poder, genera este sistema de poder injusto }}} por el cual el dueńo o desarrollador ejerce poder sobre los usuarios y genera un sistema de {\em {\color{introcolor}{colonización digital, }}}como cualquier sistema colonial practica dividir para dominar y mantiene a sus usuarios divididos e impotentes, divididos por que se les prohíbe redistribuir copias e impotentes por que no tienen el código fuente por lo tanto no pueden ni cambiar el programa  ni averiguar lo que realmente les hace y los programas privativos frecuentemente tiene funcionalidades malévolas, de hecho el caso usual en el mundo es que un usuario de software privativo esta usando malware es decir programas privativos con funcionalidades malévolas por que el desarrollador dueńo de un programa privativo reconoce el poder que tiene y siente siempre la tentación de aprovecharse del poder introduciendo funcionalidades malévolas  y esta costumbre se ha vuelto muy común, casi todos los usuario de Software privativo del mundo usan programas privativos con funcionalidades malévolas, hay tres tipos, por ejemplo las funcionalidades de vigilar a la usuario, las funcionalidades de restringir al usuario se llaman los grilletes digitales o gestión digital de restricciones traducción de DRM, y hay también las puertas traseras que reciben comandos desde ?alguien? para hacer cosas al usuario sin pedir su permiso. 

%%%%%%%%%
\begin{entradilla} %codigo para una entradilla
{\em {\color{introcolor}{Funcionalidades malévolas: }}}vigilar al usuario, restringir al usuario,  puertas traseras.
\end{entradilla}

%%%%%%%%%

Un paquete privativo que quizás contiene los tres tipos de malévolo se llama  ( que quizás conozcas este nombre ) Microsoft Windows, y se trata de funcionalidades especificas conocidas, demostradas, no se trata de especular , no hay duda de eso y también el sistema del mac tiene grilletes digitales, el sistema de los Iphone's tiene todo los tres tipos de malévolos, los Iphone llevan los grilletes digitales mas apretados que nunca por que apple a tomado el control hasta de la instalación de aplicaciones, cuando los usuarios dice jailbreak reconocen  que estos productos son cárceles para sus usuarios también flash player, gratuito pero no libre tienen funcionalidades malévolas una de vigilancia y otra de grilletes digitales, el play station 3 de sony tiene grilletes digitales  y cuando alguien descubrió como hacerle jailbrake Sony envío a la policía hacia el,  y el kindle de amazon tiene todos todos los tres tipos de funcionalidad malévola, mas tarde hablo a cerca del kindle, y casi todos los teléfonos móviles, y aunque con muchos el usuario no pueda instalar software una empresa si puede remotamente instalar cambios de software por una puerta trasera y han empleado esta puerta trasera para convertirlos en dispositivos de escucha y sin conexión transmiten información seńales para ubicar al usuario sin pedir su permiso. Entonces los considero el sueńo de Stalin, los rechazo por que es mi deber de Ciudadano  poder mi dedo en el ojo del gran hermano , es el deber de todos, entonces he demostrado que las funcionalidades malevolentes son el caso normal, pero por que, es consecuencia del poder injusto que los desarrolladores tienen sobre los usuarios la cura es que no tengan este poder, con el Software Libre nadie tiene el poder sobre nadie por que todo usuario es libre, si usas un programa libre aunque no sepas programar hay otros usuarios algunos si saben programar y no quieren ser víctimas de funcionalidades malévolas , entonces cuando de vez en cuando leen el código para hacer algún cambio serán capaces de notar cualquier cosa malévola que halla y corregir el problema y publicar una versión sin lo malévolo y seria un gran escandalo y mantenemos la información de quien contribuyo cada pieza de código y así el culpable sera descubierto y muy criticado por la comunidad. Entonces es un defensa contra lo malévolo, es la única defensa conocida, que el Software sea Libre, no es perfecto pero es mucho mejor que ser indefensos como los usuarios del privativo.

\sectiontext{white}{black}{EL MOVIMIENTO Y LA FSF}

Lance el movimiento de Software Libre en 1983 y en 1984 comencé el desarrollo de un sistema operativo destinado a ser totalmente de Software Libre, es decir {\em {\color{introcolor}{sin  siquiera una linea de código privativo, }}}ara respetar completamente la libertad de los usuarios, para que la idea de usas computadores en libertad sea una opción practica hacia falta un sistema operativo libre y fue mi deber comenzarlo por que nadie mas lo haría, en 1992 llego el ultimo componente del sistema un kernel que se llama Linux, Linux no es un sistema operativo,{\em {\color{introcolor}{ cuando alguien dice el sistema operativo Linux se}}}  

%%%%%%%%%%%%%%%%%%%%%%%%%%%%%%%%%%%%%%%%%%%%%%%%%%%%%%%%%%%%
\ebOpage{introcolor}{0.35}{HABLANDO CON EL PADRE}
%%%%%%%%%%%%%%%%%%%%%%%%%%%%%%%%%%%%%%%%%%%%%%%%%%%%%%%%%%%%

{\em {\color{introcolor}{ equivoca, }}} se trata en verdad del sistema operativo GNU con kernel Linux, entonces el nombre apropiado para esta combinación es GNU ( el lo pronuncia ŃU) con Linux, en ingles el sistema se llama ?GNU? (con correcta pronunciación en ingles) se escribe GNU, pero en castellano se puede pronunciar ŃU como si comenzara por eńe, lo que también el nombre del



mismo animal.


Elegí esta palabra no por ser el nombre de este animal si no por ser un acrónimo recursivo  y un juego de palabras, GNU es decir  G N U  quiere decir GNU no es Unix, fue la manera según la costumbre para reconocer las ideas técnicas Unix pero decir que este sistema es otro, pero teniendo también otro significado era un juego de palabras y esta palabra se usa mucho por que en ingles según el diccionario la ?G? es muda y se pronuncia ?NU?, es decir nuevo, cada vez que quieres decir ?New? es decir nuevo puedes deletrearlo ?GNU?, es un juego de palabras quizás no muy bueno, pero no pude resistirlo, pero en ingles  es muy importante no pronunciarlo con ?New? por que si dices ?the new sistem? te equivocas por que nuestro sistema ya no es nuevo. Y por favor no lo llames Linux, es un error muy común y muy dańino a nuestro trabajo, los usuarios de nuestro sistema no saben desde donde viene el sistema entonces no saben por que considerar lo que decimos, por ejemplo los motivos que desarrollaron este sistema. 

%%%%%%%%%
\begin{entradilla} %codigo para una entradilla
Por favor no lo llames {\em {\color{introcolor}{ Linux}}}, es un error muy común y muy dańino a nuestro trabajo.
\end{entradilla}
%%%%%%%%%
Pero bueno dos cosa mas a cerca de software libre, abras oído alguna vez el termino código abierto ese termino fue inventado en su versión inglesa ?open sourse? al ańo 1998, durante los ańos 1990 en la comunidad de software libre había dos campos políticos, había el movimiento Software Libre que decía {\em {\color{introcolor}{lo hacemos por la libertad, }}} el software privativo es una injusticia tenemos que escaparnos, tenemos que trabajar para que todos puedan escaparse; y había el otro campo de los que fomentaba el uso y contribuían al  desarrollo del Software Libre pero sin considerarlo un asunto ético, sin plantear valores éticos como libertad y comunidad y en su lugar decían  este sistema es decir solo tenían , valores por lo menos en este asunto, solo tenían{\em {\color{introcolor}{valores prácticos de comodidad, }}} no plantean un asunto ético, en 1998 ellos inventaron el termino código abierta para nunca mas decir libre, para que nuestras ideas se olvidaran y así han hecho esfuerzos bastante grandes y nuestras ideas casi se olvidaron, pero no por que hemos luchado bastante fuerte para que no se olviden, fue en ese ańo que comencé a viajar casi todo el ańo haciendo  conferencias, por que los medios grandes, sobre todo los medios anglófonos nunca mencionan el Software Libre cuando hablan de mi dicen código abierto o open source, entonces daban una impresión completamente falsa de mi postura, tengo que escribir corrigiendo y  criticando.

Entonces tu postura es de tu elección pero ahora puedes reconocer si alguien dice código abierto que idea esta apoyando y puedes  elegir que idea apoyar , entonces si estas de acuerdo con nuestras ideas éticas por favor muestra tu apoyo al publico diciendo siempre Software Libre, libre, libre, por que es asunto de libertad, cada vez que lo digas y otros lo oigan nos apoya y este apoyo es un apoyo que necesitamos.

\sectiontext{white}{black}{SOFWARE LIBRE EN LAS ESCULAS} %como se hace una sección

Ultimo punto, las escuelas tienen que  enseńar únicamente con Software Libre, todos los niveles de escuelas y todas las actividades educativas, no solo para ahorra dinero es un beneficio secundario, pero hay motivos éticos por ejemplo las escuelas tiene un misión social de educar a buenos ciudadanos en una sociedad capaz fuerte independiente, solidaria y libre, en la informática es decir graduar a usuarios habilitados al uso de Software Libre y listo para vivir en un tal sociedad, pero enseńar el uso de un programa privativo es implantar dependencia a una entidad, no se debe enseńar Software privativo, va en contra de la misión de la escuela, la misión de la educación. Por que ofrecen unos desarrolladores de privativo copias gratuitas de sus programas no libres a las escuelas, es como por que los traficantes de drogas ofrecen ampollas gratuitas, y las ofrecerían a las escuelas también por que desearían aprovecharse de las escuelas para imponer la pendencia a la sociedad ?la primera dosis es gratis ?  dice Microsoft y a veces Apple, entonces la escuela rechazaría las drogas que sean gratuitas o no y tiene que rechazar privativo que sea gratuito o no. Pero también hay el motivo de la educación misma en la informática, algunos son programadores natos y a la edad de 10 a 13 ańos antojan aprender todo de la informática, si usan un programa quieren saber como lo hacen, pero cuando el pregunta al profesor,{\em {\color{introcolor}{ si el programa es privativo, solo puede contestarles es un secreto no podemos saberlo, }}}es decir que la educación no se permite, un programa privativo es conocimiento denegado, es el contrario del espíritu de la educación y no debería ser tolerado en una escuela, pero si el programa el libre el profesor puede explicarles cuanto sepa, luego darles copias de código fuente del programa diciéndoles léelo y comprenderás todo y los leerán por que antoja comprender todo y puedes decirles, si encuentra algún punto que no comprendes solo muéstramelo y podremos comprenderlo juntos como se aprende a escribir bien el código, es decir de manera que otros lo comprenderán, hace falta leer mucho código y escribir mucho código, solo el Software Libre permite leer el código de programas grandes que realmente se usan y luego hace falta escribir mucho código, es decir escribir código en programas grandes  pero para comenzar hace falta escribir pequeńos cambios, cambios sencillos en programas grandes, para ofrecer la posibilidad de

%%%%%%%%%%%%%%%%%%%%%%%%%%%%%%%%%%%%%%%%%%%%%%%%%%%%%%%%%%%%
\ebOpage{introcolor}{0.35}{HABLANDO CON EL PADRE}
%%%%%%%%%%%%%%%%%%%%%%%%%%%%%%%%%%%%%%%%%%%%%%%%%%%%%%%%%%%%

  educación 
profunda en la informática en la programación hace falta usar Software Libre, pero hay también la educación moral que se aplica a todos los estudiantes, una escuela tiene que ir {\em {\color{introcolor}{ mas allá que ensańar hechos y capacidades tiene que ensańar el espíritu de buena voluntad, }}}es decir el habito de ayudar a los demás, entonces cada clase debe tener esta regla, los  estudiantes si traes un programa no puedes guardarlo para ti, tienes que compartir copias con el resto de la clase, incluso su código fuente por si acaso alguien quiera aprender, por que esta clase es un lugar apara compartir los conocimientos, entonces no se permite traer un programa privativo a la clase, para dar el buen ejemplo la escuela tiene que seguir su propia regla, tiene que traer únicamente SL a la clase y distribuir copias a todos en al clase, si tienes una relación con una escuela , como por ejemplo con esta universidad es tu deber militar  por la migración de la escuela al software libre, y hace falta plantear el asunto públicamente para informar a mas gente del asunto por que mayormente no lo conocen y para buscar aliados. Conozco que en esta universidad utilizan un programa que se llama Sia, es un programa privativo , y que la universidad impone su uso, hace falta resistencia por los estudiantes, y lo que hace falta es reemplazarlo por otro tipo de sistema, que no exija que los estudiantes ni los profesores ejecuten código privativo, y es posible, supón que los estudiares de esta universidad podrían escribirlo y seria un buen entrenamiento.

%%%%%%
\sectiontext{white}{black}{LIBERTAD A LAS OBRAS} %como se hace una sección

%%%%%%

Me preguntaban  si estas ideas se aplican a algo que no sea software y cuando me lo preguntan suelen especificar el hardware, es que el hardware debería ser libre, pero que quería decir que un objeto físico sea libre en este mismo sentido, evidentemente se trata del mismo sentido de libertad, entonces tendría que llevar las cuatro libertades, como la libertad 0 que seria la libertad de usar el objeto como quieras, pero normalmente si compra un objeto tienes esta libertad, los objetos físicos no suelen venir con licencias para restringir a su dueńo en el uso del objeto, si hay actos prohibidos por la ley que no puedes hacer ni con con ni sin este objeto pero el objeto mismo no restringe, entonces tienes la libertad 0, y la libertad 1 es la libertad de estudiar y cambiar el código fuente pero los objetos físicos normalmente no tienen código fuente, tendremos que adaptar esta libertad, podría ser la libertad de estudiar y cambiar el objeto, y esta libertad la tienes, si eres dueńo del objeto estas libre de estudiarlo y cambiarlo pero las posibilidades practicas de hacerlo son bastante limitadas, si el objeto es de madera es mas fácil de cambiar, si es de metal es mas difícil, si es de plástico es muy difícil, si es de cerámica es imposible, si es un chip de computadora es imposible, pero al gran dificultad se presenta con las libertades dos y tres, por que es asunto de hacer y distribuir copias y no hay copiadoras para los objetos físicos, entonces la idea de hardware libre hoy en día no tiene mucho sentido, quizás en el futuro, quizás cuando tengamos algo como el transportador de ? Atar tres ? que podríamos cambiar un poco para copiar objetos, seria útil, pero lo que hay son las impresoras 3D que son capaces de fabricar objetos desde diseńos entonces{\em {\color{introcolor}{ estos diseńos  son obras, no son objetos, son obras y son ejemplos de una gran categoría de cosas para las cuales esta pregunta si tiene sentido, }}}es decir las obras, cualquier tipo de obra podrías tener un copia de la obra y podrías copiarla o cambiarla, entonces si el tener o no tener estas libertades es una cuestión con sentido, por que puedes hacer estas actividades con una obra, entonces la madera de la pregunta es que libertades deberíamos tener con las obras sobre todo las obras publicadas, si la obra no es un programa usualmente la única causa de restricciones al usuario es el derecho de autores, con los programas no es así la restricción principal viene de los contratos y a veces intentan imponer contratos también al usuario de otro tipo de obra pero principalmente es el {\em {\color{introcolor}{ derecho de autor }}} , entonces podemos formular la misma pregunta desde el otro lado diciendo que debería decir al ley del derecho de autor, para considerar esta pregunta es útil considerar la historia de al copia, por que el derecho de autor se a desarrollado en razón estrecha con al tecnología de la copia, los cambios tecnológicos no pueden cambiar nuestros principios éticos o no deberían cambiarlos por que los principios son mas profundos que la tecnología, pero para aplicar nuestros principios a un caso especifico lo que hacemos es considerar las opciones y considerar los resultados que tienen y buscar los resultados según nuestro principios, pero un cambio de contexto como de tecnología puede cambiar los resultados probables del mismo acto, entonces puede hacer el acto mas correcto o mas malo de lo que era, por ejemplo si pudiéramos resucitar a los muertos el asesinato no seria tan malo.

%%%%%%
\sectiontext{white}{black}{LA HISTORIA DE LA COPIA} %como se hace una sección
%%%%%%

Entonces abordemos la historia de la copia, la copia comenzó en el mundo antigua cuando se hacia leyendo la copia de un libro y escribiendo otra copia, una tecnología bastante sencilla y muy lenta, muy poco eficiente pero con otras características interesantes y pertinentes, por ejemplo no tenia economía de escala hacer diez copias te abría costado diez veces el tiempo de hacer una, tampoco necesitaba herramientas especiales solo herramientas que necesitabas para leer y escribir y tampoco una capacidad especial solo la capacidad de leer y escribir, por lo tanto cualquier persona letrada podía copiar mas o menos bastante bien como quien quiera y resulto un sistema descentralizado de copiar en cualquier lugar si había alguien que tenia  una copia y deseaba hacer otra lo hacia y nadie le decía que no pudiera por que no había derecho de autor en el mundo antiguo, la idea nunca se planteo pero excepto en un caso, si el príncipe de la región, si a el no le gustaba ese libro podría castigarte muy fuerte por copiarlo, pero ese no fue el derecho de autor, si no algo muy relacionado con la censura y era así durante muchos ańos  pero luego hubo un avance 

%%%%%%%%%%%%%%%%%%%%%%%%%%%%%%%%%%%%%%%%%%%%%%%%%%%%%%%%%%%%
\ebOpage{introcolor}{0.35}{HABLANDO CON EL PADRE}
%%%%%%%%%%%%%%%%%%%%%%%%%%%%%%%%%%%%%%%%%%%%%%%%%%%%%%%%%%%%

en la tecnología de copiado, es decir la imprenta. La imprenta hacia mas eficiente la copia pero de manera no uniforme, en el mundo antigua teníamos este caso (muestra sus dos manos al mismo nivel) la copia masiva y la copia de una sola copia era antiguamente poco eficiente pero con la imprenta teníamos esta situación (muestra sus dos manos, esta vez una mucho mas arriba que la otra), la imprenta hacia mucho mas eficiente la producción masiva sin beneficiar a hacer una sola copia, la manera mas eficiente de hacer una sola copia seguía siendo a mano, entonces la imprenta tenia unas características muy diferentes a las de la copia a mano, tenia una economía de escala por que poner la tipografía tomaba mucho tiempo, pero luego podías sacar muchas copias idénticas mucho mas rápidamente que escribirlas, y la imprenta necesita herramientas muy caras, herramientas especiales para imprimir, la mayoría de las personas letradas no tenia prensa y tampoco sabia usarla por que es otra capacidad diferente de al capacidad de leer y escribir, entonces la imprenta llevo a un sistema centralizado de producción de copias, copias de cualquier libro producían en unos lugares y luego se transportaban a donde alguien quisiera comprar. 
El derecho de autor comenzó en la época de la imprenta, había varios sistemas, un poco como el derecho de autor, pero el derecho de autor actual comenzó en Inglaterra en 1557, como un sistema de censura, explícitamente para la censura, la relación entre el derecho de autor y la censura ha durado hasta hoy, la idea fue que para poder imponer legalmente imprimir un libro el editor tenia que pedir permiso al estado y ese permiso fue otorgado como un monopolio permanente para ese editor y este sistema controlaba hasta mas o menos 1680, cuando permitieron vencerse la censura y todo este sistema, pero los editores reclamaban su monopolio perdido lo que fue legislado en 1700 y poco fue diferente, fue un monopolio de 14 ańos para el autor, no para el editor si no para el autor, y si a los 14 ańos seguir vivo, podía pedir otros 14 ańos con el máximo de 28 ańos, un método para fomentar la escritura. 
Cuando escribieron la constitución de los Estados Unidos había la idea de quizás darles a los autores el derecho a un monopolio, pero fue rechazada, en lugar de esa idea, esa constitución dice que el congreso tiene el poder de fomentar el progreso, con la medida de otorgar a los autores un monopolio por un tiempo limitado; tiene tres puntos interesantes, primero no exige que halla un derecho de autor , solo permite un derecho de autor, {\em {\color{introcolor}{ si hay algún derecho de autor es para promover el progreso, }}}  no es por que un autor sea mas importante que el resto del mundo, no es para el, no se hace según la constitución para el, solo para el publico y tiene que tener un plazo limitado, desde entonces los editores intentan obligarnos estos punto fundamentales.
En la época de la imprenta el derecho de autor funcionaba como un reglamentación industrial, reglamentando a los editores con el poder en las manos de los autores, pero el sistema fue organizado para proporcionar {\em {\color{introcolor}{ beneficios al gran publico }}}Entonces si siguiera sin cambiarse la época de la imprenta, probablemente no desearía criticar el derecho de autor, pero la época de la imprenta esta cediendo paulatinamente a la época de las redes informáticas, otro avance en la tecnología de la copia, que también la hace mas eficiente y también de manera no uniforme, teníamos en al época de la imprenta esta situación (pone sus manso un arriba de la otra con gran diferencia de alturas) la producción masiva bastante eficiente y la producción de una sola copia muy poco eficiente, pero con la tecnología digital tenemos esto (pone sus manos a diferente altura, pero esta vez la diferencia de alturas es poca), tiene beneficios en ambos casos pero el gran beneficio es para hacer una sola copia, por ejemplo en los discos compactos, los discos de producción masiva son mas baratos y mas durables, pero hacer un disco es bastante barato para que cientos de millones de humanos lo hagan, todos en el mundo desarrollado pueden hacerlo, es una situación mas parecida al mundo antiguo, en el mundo antiguo antiguo teníamos esto (pone sus manso al mismo nivel ) en el mundo actual tenemos esto (pone sus manos un poco desniveladas) como menos diferencia entre la eficiencia de la producción masiva y la producción de una sola copia, como en la época de la imprenta. 
Este cambio, cambia el efecto del derecho de autor, si no hubieran cambiado la letra de esa ley su efecto seria totalmente diferente, por que hoy en día quieren aplicar el derecho de autor a los lectores, a todo el mundo es decir, entonces ya no funciona como una reglamentación industrial para los editores en manos de los autores con beneficios al gran publico, si no, como una restricción insoportable al gran publico mayormente en las manos de los editores en nombre de los autores, es decir una ley injusta, no debería existir como es, es decir que la libertad natural trocábamos todos por otro beneficio por que no sabíamos en la época de la imprenta ejercerla, ahora ya sabemos ejercer la libertad, la queremos ejercer.

%%%%%%
\sectiontext{white}{black}{EL ESTADO Y LAS EMPRESAS} %como se hace una sección
%%%%%%

Que haría un gobierno que quisiera representar los intereses del pueblo, disminuiría el poder de los derechos de autor, no necesariamente hasta cero pero menos, pero podemos medir la falta de democracia en la mayoría de los países de mundo por su tendencia de hacer el contrario, de extender el poder del derecho de autor cuando deberían disminuirlo, que hacen, una dimensión del derecho de autor es su plazo, hay una onda mundial de la extensión del plazo, por ejemplo Colombia intentar el plazo del derecho de autor por 20 ańos mas bajo ordenes de los Estados Unidos, {\em {\color{introcolor}{ por que Colombia a perdido su independencia, }}} pero lo hacían en los Estados Unidos en 98 con una ley que llamamos "mickey mouse copyrigth dead ", que hacia extendía el derecho de autor por 20 ańos mas para todas las obras del pasado y del futuro, pero como podrían fomentar mas producción de obras en el pasado extendiendo hoy su derecho de autor, necesitaríamos una maquina del

%%%%%%%%%%%%%%%%%%%%%%%%%%%%%%%%%%%%%%%%%%%%%%%%%%%%%%%%%%%%
\ebOpage{introcolor}{0.35}{HABLANDO CON EL PADRE}
%%%%%%%%%%%%%%%%%%%%%%%%%%%%%%%%%%%%%%%%%%%%%%%%%%%%%%%%%%%%


tiempo y si tienen la maquina del tiempo no la han usado, por que la historia no dice que en los ańos 20 cuando los artistas descubrieron que en 98 su derecho de autor seria extendido por 20 ańos mas se pusieron al trabajo con mucho vigor, por que no usar su maquina del tiempo para darnos obras clásicas amadas.


%%%%%%%%%
\begin{entradilla} %codigo para una entradilla
Podemos medir la{\em {\color{introcolor}{ falta de democracia }}}en la mayoría de los países del mundo por su tendencia a aumentar el poder del derecho de autor.
\end{entradilla}

%%%%%%%%%


Teóricamente es concebible extender el derecho de autor para obras futuras pueda fomentar la escritura en el presente, pero solo para los artistas locos, y la mayoría no son locos, es un mito.{\em {\color{introcolor}{ Según la economía, el valor actual actual descontado de 20 ańos mas de derecho de autor comenzando 50 ańos después de tu muerte es demasiado poco, }}}para cambiar tu decisión racional de hacer o no hacer un obra, entonces la verdadera razón de ser de esa ley es que unas empresas tiene monopolios que deberían vencerse, como por ejemplo el derecho de autor sobre el personaje de Mickey Mouse, por que el derecho de autor sobre la primera película en la cual apareció Mickey Mouse tendría que vencerse, entonces Disney quería comprar esta ley  y otras empresas también, en los estados Unidos las empresas compra leyes,{\em {\color{introcolor}{ en Colombia los Estados Unidos ordena leyes, }}} igualmente injusto como sistemas de gobierno, es decir que ambos países son colonias del imperio de las empresas.

Sospechamos que en 2018  volverá la misma situación, y Disney comprara otra ley para extender por 20 ańos mas sus derechos de autor, sospechamos que todas las empresas tienen el plan de comprar una ley cada 20 ańos para que ninguna nunca caiga al dominio publico mas.

Pero el derecho de autor tiene otra dimensión mas importante, su amplitud, que usos de una obra serán controlados por el derecho de autor, y que usos serán libres; en la época de la imprenta el derecho de autor no debía controlar ni prohibir todos los usos de una obra, los usos controlados eran las excepciones en un campo mas amplio de uso libre, pero con la tecnología digital, los editores tienen la idea de agarrar poder total de imponernos un sistema de pagar cada vista y lo hacen con los grilletes digitales, quieren imponernos el uso de los grilletes digitales, los grilletes digitales siempre vienen en programas privativos por que su propósito es restringir al usuario, que presupone que el usuario no obtenga el control del programa, si pusieran grilletes digitales en un programa libre, cualquier usuario podría cambiar el programa para que no tenga los grilletes, entonces no serviría su propósito injusto, por lo tanto siempre usan software privativo para grilletes digitales. También hay hardware especifico para grilletes digitales por ejemplo las PC's modernas vienen con grilletes digitales cambien en el hardware, transmiten la seńal de vídeo en formato encriptado, específicamente entre el procesador y el monitor para restringir a los usuarios es hardware malévolo, pero mayormente lo hacen el software y cuando hay hardware malévolo hace falta software malévolo  para activar esa funcionalidad, entonces siempre esta en el software. Estos programas atacan nuestra libertad a dos niveles a al vez.

Suele ser un conspiración entre empresa editoriales y empresas de tecnología, una sola empresa no podría lograrlo, tiene que colaborar, que conspirar para restringir nuestro acceso a la tecnología. El primer caso que el gran publico veía fue en los disco DVD por que el formato de los discos puede tener el vídeo encriptado, la conspiración de los DVD's dijo que si quieres tienes que inscribirte en la conspiración y prometer no revelar el secreto del formato, por algún tiempo funciono pero luego descubrieron el formato y publicaron un programa libre capaz de desencriptar el vídeo y fue posible comprar un DVD y mirarlo con Software Libre pero los Estado Unidos saco una ley censurando les distribución de este Software Libre y logro distribuirla a otros lugares como otros países por medio de sus tratados de libre exportación.

A pesar de estas leyes los programas para leer DVD están fáciles de encontrar, entonces desarrollaron otro sistema de grilletes digitales que se llama AACS que se usa en los discos Blue Ray, y hay un programa libre para desencriptar AACS, y a veces encuentra alguna clave que sirve para algunos discos, pero los disco Blue Ray tienen también otro sistema de grilletes digitales y lo cambian cada trimestre, entonces no creo que podremos vencer ese sistema, no tenemos los recursos, con bastante gente colaborando si podríamos hacerlo, pero no tenemos bastante gente, no debemos suponer que siempre ganaremos por nuestra capacidad técnica, el enemigo no es incompetente, siempre es injusto, pero no siempre es incompetente, es un error grave subestimar la capacidad del enemigo.

Hemos visto también grilletes digitales en la distribución de vídeo en al red, hay sitio web como Netlist y Hulu, que distribuyen vídeo por al red con grilletes digitales y no creo que halla software libre que sea capaz de interactuar con esos sitios, entonces hace falta no usarlos. El principio fundamental de valorar tu libertad es nunca aceptar ninguno producto concebido para privarte de tu libertad de no tener disponible las herramientas necesarias para romper los grilletes, entonces si tienes el programa libre para mira los DVD's no tienes por que no mira DVD's , pero sin ese programa no debes mira DVD's  y no debes mira Netlsit, no debes mirar Hulu, hemos visto también los grilletes digitales en la musica con los mp3, hace doce ańos los llamábamos discos corruptos por que la información estaba escrita de tal forma que no lo podían leer las computadoras. 

%%%%%%%%%%%%%%%%%%%%%%%%%%%%%%%%%%%%%%%%%%%%%%%%%%%%%%%%%%%%
\ebOpage{introcolor}{0.35}{HABLANDO CON EL PADRE}
%%%%%%%%%%%%%%%%%%%%%%%%%%%%%%%%%%%%%%%%%%%%%%%%%%%%%%%%%%%%

%%%%%%%%%
\begin{entradilla} %codigo para una entradilla
{\em {\color{introcolor}{ nunca }}}nunca aceptar ninguno producto concebido para privarte de tu libertad.
\end{entradilla}

%%%%%%%%%

Hace una década empezaron a intentar ponerle grilletes digitales a los libros, cuando nos intentaron convencer de que todos usaríamos los libros electrónicos y a un editor se le ocurrió que tendría mucho éxito entre los tecnificarlos si comenzaba con mi biografía, entonces encontró a un autor que me pidió la colaboración y dije que colaboraría si ese libro salia sin grilletes digitales y con libertad de compartir, el editor dijo que no, pero encontramos otro editor, dispuesto a aceptar estas condiciones por que deseaba publicar este libro en papel, y se hizo, pero también publicaron el texto bajo una licencia libre, específicamente la ?GNU free documentation license? que usamos para nuestro manuales, pero de todos modos los libros electrónicos fracasaron, había sido un gran placer afirmar por que había fracasado por que al gente valoraba su libertad, pero evidentemente no, fue por motivos de comodidad de motivos prácticos, decían volverán no hemos vencido, tendremos otra batalla.

Hoy en día se ve esa batalla con productos como el  ?swindle? de Amazon, swindle quiere decir engańo, productos diseńados para cortar las libertades tradicionales de los lectores como la de compra un libro anónimamente, ellos registran todas las compras de libro en una gran base de datos, la existencia de tal lista en cualquier parte perjudica los derechos humanos es intolerable que exista, tampoco nos permiten al libertad de regalar el libro después de leerlo, o de prestar el libro, libertades que Amazon elimina con grilletes digitales y también con contratos para restringir a los usuarios, Amazon no respeta la propiedad privada dice que todas las copias pertenecen a Amazon, que tu no puede ser el dueńo de los libros que compraste, ni siquiera respeta la liberta de guardar el libro cuanto quieras y por fin dejarlo a tus herederos , llegando a borrar los libros remotamente cuando según ellos se vence el contrato.

          
Luego de un reconocido escandalo por borrar libro Amazon prometió no volver a borrar ningún libro y que solo lo haría pro ordenes del estado,yo me imagino que eso podría pasara con algo que hace ańos paso, la autora de Harry Potter consigo una orden e un juez para que los que hubieran comprado el libro no lo leyeran y que lo devolvieran literalmente, por que se había presentado un error en una librería y lo había sacado a al venta antes de estreno mundial y no cumplió con el objetivo de la autora de obtener mucho dinero como ella quiera manteniendo el suspenso para que el libro saliera al mismo tiempo en todos lados, cuando reconocieron  el error Rowling en su ansias de obtener mucho dinero quiso usar la ley  para reservar el error y que el error de su equipo lo pagaran sus lectores, entonces consigo esa orden diciendo que los que habían comprado legalmente el libro no lo leyeran, por eso lance el boicot a Harry Potter, no digo que no leas los libros o que no veas las películas, digo que no los pagues, ordenar que no los leas eso dejo al autor, pero si hubiera podido remotamente borrar eso libros lo habría hecho, pero no pido por que eran copias en papel, pero si fueran libros par el swindle lo abría hecho, es un peligro real a la libertad.

Hemos visto estos grilletes digitales en todos lo medios, hace falta rechazar  los sistemas con grilletes digitales, si no tienes con que romper los grilletes, pero una respuesta individual no basta si el enemigo organiza conspiraciones ,se que existen, por que nos son secretas publican sus planes, tiene sitio web,{\em {\color{introcolor}{ una conspiración para impedir el acceso a la tecnología debería ser un delito grave }}}como una conspiración para fijar los precios, pero no lo es, las empresas reconocen que nuestros gobiernos están de su lado contra nosotros, por lo tanto publican sus conspiraciones. Para esto en algunos países han creado sistemas de leyes  informales para castigar a la gente sin proceso, están dispuestos a acabar con el principio básico de la justicia ?ningún castigo sin proceso justo?, su deseo es castigar a los internautas con desconexión por la mera acusación de haber compartido.{\em {\color{introcolor}{ Quieren imponernos un sistema de opresión para mantener su dinero. }}} Toda esta guerra es injusta por que compartir es bueno.

El estado tiene que disminuir, si ese Estado fuera democrático tendría que disminuir el poder del derecho de autor, pero como,primero hay la dimensión de plazo, sugiero que el derecho de autor dure 10 ańos desde la fecha de publicación de la obra, por que esa fecha de publicación, antes de la publicación no tenemos copias, no nos importa mucho que podamos hacer con unas copias que no tenemos, pero por que 10 anos,por que el ciclo de publicación suele durar tres anos, dentro de tres ańos la mayoría de los libros ya no están disponibles, 10  ańos son mas de tres veces este ciclo, entonces debe de bastar, pero no lo digo como el plazo exacto o correcto, es aproximadamente correcto, pero no todos están de acuerdo, una ves propuse estas ideas en una mesa redonda con autores de ficción y recibí una critica, un autor dijo ?10 anos seria horrible, nada mas que 5?, me sorprendió también por que hasta ese momento lo editores habían logrado engańarme con su publicidad por que cuando piden mas poder sobre nosotros siempre dicen que es por los artistas e invitan algunas artistas muy conocidos que dicen ?si queremos mas poder? y nos dejan suponer que todos los artistas piensan igual pero no es así, solo las estrellas desean mas poder, por que las estrellas realmente ganan por ese poder pero los demás los que realmente necesitan mas dinero no ganan por este sistema, ganan muy poco y mas poder no les llevaría a beneficios este autor recibió un premio por un libro pero no había ganado mucho y su libro aparecía no disponible, sus fans le escribían diciendo como puedo conseguir una copia; Deseaba enviarles copias por correo electrónico pero no podía por que la empresa editorial a pesar del contrato que decía que al no estar disponible el libro los derecho volverían a el,la  empresa no quería admitir el echo de que el libro no estaba disponible y usaba el derecho de autor de su libro para negarle el derecho de distribuir copias de su propio libro para que se apreciara. 

Los artistas comienzan con la idea de mostrar su obra para que se aprecie, y aparte de unas estrellas que reciben bastante dinero para corrompernos,{\em {\color{introcolor}{  la mayoría no recibe nunca tanto dinero y sigue deseando que su obra se aprecie, }}}  el había aprendido por una lección dura que mas de 5 ańos de derechos de autor tenia muy poco probabilidad de beneficiarlo, y podría fácilmente dańarlo, entonces dijo nada mas de 5.  No exijo que sean 10 si todo el mundo prefiere 5 acepto 5. pero quiero proponer algo que sea mas conservador, menos radical.


%%%%%%
\sectiontext{white}{black}{TIPOS DE OBRAS} %como se hace una sección

%%%%%%

Hay también el asunto de la amplitud del derecho de autor, que usos de una obra deberían ser controlados por el derecho de autor,para esto, no tengo una recomendación uniforme, distingo varios tipos de obras según su manera de contribuir a la sociedad, tres categorías, hay las obras del uso practico, es decir el uso para hacer trabajos en la vida, hay las obras que muestran el pensamiento de alguien, y hay las obras de arte y divertimiento, tres maneras de contribuir a la sociedad.

Primero las obras de{\em {\color{introcolor}{ uso practico, }}}estas obras tienen que ser libres hasta antes de su publicación, si alguien te da una obra de uso practico, deberías tener inmediatamente aunque seas el único que haya recibido una copia, mereces la libertad en el uso de la obra, estas obras tienen que ser libres es decir con las 4 libertades esenciales, por que es el mismo argumento, o los usuarios tienen el control sobre la obra o la obra ejerce el poder sobre ellos, entonces para tener el control de la obra necesitas las 4 libertades que incluyen desde la libertad de redistribución comercial, hasta la libertad de vender copias, por que para que lo usuarios tengan control estas libertades hacen falta.

Afortunadamente tenemos una respuesta al argumento que {\em {\color{introcolor}{ si no sedemos la libertad no escribirían estas obras, }}}podemos ver que hay bastante escritura de obras de uso practico que son libres en el mundo actual en el cual no tienen que ser libres, es opcional hacerlas libres pero sin embargo publican muchas obras libres.

Las obras que{\em {\color{introcolor}{ muestran el pensamiento de alguien, }}}or ejemplo las autobiografías, los ensayos de opinión. Publicar una versión cambiada de una tal obra, es presentar mal a otro y no contribuye a la sociedad, no veo por que permitir la distribución de versiones modificadas sin permiso, entonces no hay por que permitir el uso comercial, propongo que estas obras, todas sean libres, pero hay una libertad que todo el mundo necesita, es la libertad de compartir, cuando digo compartir quiere decir redistribuir de manera no comercial copias exactas de la obra es decir, no vender, no alquilar, no modificar. Es una libertad limitada pero esta libertad mínima es necesaria para que el derecho de autor vuelva a ser soportable en la sociedad, por que pondría fin a la guerra contra compartir.  

Hay medidas injustas que nos imponen son para que no compartamos, pero compartir es bueno, no se debe eliminar la practica de compartir, y {\em {\color{introcolor}{ atacar la practica de compartir es atacar la sociedad, }}}y también con la tecnología digital compartir es fácil, y si algo es bueno y fácil la gente lo hace y que podría ser capaz de hacer que la gente cese de compartir, únicamente medidas crueles y draconianas, entonces toman una serie de medidas crueles y draconianas, como el encarcelamiento de quienes comparten, hay que legalizar compartir. Propongo un sistema reducido de derecho de autor, que se aplicaría a todo uso comercial y a toda modificación pero legalizando compartir, incluso por la red.

%%%%%%%%%
\begin{entradilla} %codigo para una entradilla
Si algo es{\em {\color{introcolor}{ bueno y fácil }}}la gente lo hace.
\end{entradilla}

%%%%%%%%%

La tercera categoría, la del {\em {\color{introcolor}{ arte y entretenimiento, }}}n estas obras el asunto del compartir versiones cambiadas fue difícil por que hay argumentos en ambas lados pero por fin reconocí que un artista que quiere publicar una versión cambiada de una obra puede esperar, si no tiene que esperar , si no tiene que esperar a mas de 10 ańos, entonces propongo el sistema disminuido de derecho de autor por 10 ańos después de eso la obra caerá al dominio publico y quien quiera podrá publicar versiones cambiadas, teóricamente todo obra caerá algún día en el dominio publico, algún día se podrán publicar versiones cambiadas.  Pero hay un asunto mas, el remix, el remix quiere decir tomar partes de varias obras y combinarlas quizás con algo nuevo para hacer una obra nueva completamente diferente en su propósito y su punto y este no es igual ha hacer una versión modificada. No puedo poner una linea exacta entre los dos pero en los casos normales son muy diferentes, remix tiene que ser legal, por que {\em {\color{introcolor}{ el propósito del derecho de autor es fomentar la producción de obras }}} e interpretarlo de forma a ser obstáculo para la producción de obras es un absurdo, tan absurdo que solo podría suceder bajo un sistema que a distorsionado completamente el propósito del estado como sistema para apoyar a unas empresas. Tenemos que legalizar la practica de compartir incluso por la red. Las fabricas de musica es decir los editores de discos, dirían ?es horrible robas el dinero a los músicos? pero no es verdad por que esas empresas ya lo han robado todo a los músicos, por que imponen a todo excepto las estrellas establecidas a contratos explotativos que dicen teóricamente si compras una copia, una parte del dinero es para los músicos, pero nunca los reciben por que según el mismo contrato los gastos de producción y publicidad se consideran como avance a los músicos y esta fracción del precio tiene que devolverse, y nunca llegan a devolver completamente este avance, entonces nunca reciben dinero de la venta de sus discos.  Si una banda llega al fin del primer contrato y puede negociar otro contrato no explotativo, por ese contrato realmente recibirá dinero cuando la gente compre los discos bajo ese contrato pero los discos anteriores quedan bajo el primer contrato y nunca reciben dinero de esos discos. Los beneficios que 

%%%%%%%%%%%%%%%%%%%%%%%%%%%%%%%%%%%%%%%%%%%%%%%%%%%%%%%%%%%%
\ebOpage{introcolor}{0.35}{HABLANDO CON EL PADRE}
%%%%%%%%%%%%%%%%%%%%%%%%%%%%%%%%%%%%%%%%%%%%%%%%%%%%%%%%%%%%

reciben los músicos son ser mas conocidos, pueden tener mas conciertos, pueden cobrar mas y así gana mas, pueden vender mercancías en sus conciertos, aunque las fabricas  intenten agarrar eso también, antes de ir aun concierto, o antes de comprar mercancías en un concierto, seria correcto investigar que porción del dinero sera para los músicos mismos y que fracción para una fabrica de musica, Es un beneficio, entonces el resultado del contrato de discos puede beneficiar a los músicos, pero hay otras maneras de hacer conocerse, no hace falta este complejo musico-industrial, entonces legalizar la practica de compartir musica seria completamente bueno, y los músicos excepto las estrellas, no perderán nada y en cuanto a las empresas de discos, no estoy en contra de fabricar y vender discos, es mi manera de adquirir musica comercial, por que puedo pagar anónimamente y en efectivo. Las empresas grandes de discos que han comprado leyes injustas que han demandado a miles de adolescentes por miles de dolares solo merecen perder todo, quiero darles lo que merecen y lo que merecen es desaparecer.

Siempre que tengamos empresas de venta de discos seria bueno tener un derecho de autor que les exigiera realmente pagar a los músicos, por que eso es uso comercial, no propongo eliminar el derecho de autor para el uso comercial de las obras artísticas. También hay las películas, habrás oído sumas astronómicas para hacer una película, pero un productor me explico que son falsas primero por que mas de la mitad no fue para la producción sino la publicidad y en la parte que es menos de la mitad de la producción exageran las sumas con trucos de contabilidad entonces el coste real es mucho menos. 
Y también se puede decir que {\em {\color{introcolor}{ Hollywood produce sistemáticamente basura, }}} estoy completamente en contra de la censura, es injusto censurar una obra aunque sea odiosa por su violencia, he visto películas celebres que por su violencia me dieron asco, pero estoy en contra de censurarlas, por que la censura es peor que cualquier obra, pero no es asunto de censurar o no censurar, no propongo censurar, la cuestión es seguir nuestra libertad no para ayudar a que hollywood siga produciendo basura, y a eso digo no, no quiero ceder esa libertar, es verdad que le sistema actual apoya muy mal a los artistas y un sistema disminuido probablemente no los apoyaría mejor. 

\sectiontext{white}{black}{APOYO A LOS ARTISTAS} %sección

Podríamos apoyar mejor a los artistas, pues con legalizar compartir. Tengo dos métodos para proponer un método es usar dinero publico, puede ser una parte del dinero publico actual o puede ser recogido quizás sobre un impuesto especial quizás sobre conectividad y repartir este dinero entre los artista según el éxito, podemos medir este éxito por la frecuencia de compartir las obras en las redes peer-to-per, con la raíz cubica, es decir un artista A que tenga 1000 mil veces el éxito de otro artista B solo recibirá 10 veces el dinero que recibiría B, así para que el artista B reciba algo considerable no se tendría que hacer asquerosamente rico al artista A y sigue funcionando como mas éxito mas dinero, pero {\em {\color{introcolor}{ se transfiere la mayoría del dinero desde las pocas estrellas a los muchos artistas igualmente capaces de éxito mediano, a los que realmente necesitan mas apoyo, }}}usaría nuestro dinero de manera mas eficiente que el sistema actual. También propongo el sistema de pagos voluntario poniendo un botón en cada reproductor para que sea fácil darlo en los programas,si fuera fácil lo harías, la cantidad puede ser 200 pesos,de pagos pequeńos, para que muchos estén dispuestos a pagar. 


\sectiontext{white}{black}{MÁS INFORMACION} %sección

Quiero decirles donde encontrar mas información sobre el Software libre es gnu.org  y hay también Free Software Foundation cuyo sitio es fsf.org , pude hacerte miembro de FSF que es una manera de apoyarnos en nuestro trabajo, soy voluntario de tiempo completo, la FSF no me paga pero tiene un equipo de empelados, y como obtenemos el dinero para es con  los miembros, también hay Free Software Foundation Latín América  que es fsfla.org, también merece tu apoyo.

\sectiontext{white}{black}{PREGUNTA} %sección

Es de admirar que en ningún momento de su charla (de dos horas y diez minutos), se detuvo ni siquiera para tomar agua  Acabada la conferencia, subasto un peluche de un Ńu y dijo que antes de las fotos y firmar cosas iba a responder preguntas. La mayoría de las preguntas fueron de temas técnicos, que a mi personalmente no me parecía adecuado ya que el todas esa respuestas se pueden encontrar en  sus escritos publicados en la web por tal motivo no me tome el trabajo de transcribirlas (eran otros 40 minutos de charla si quieres obtener las grabaciones en la pagina oficial de GLUD  glud.org van a a estar subidas), ya que una pregunta no se le puede hacer todos los días Richard Stallman aproveche la oportunidad solucionar un interrogante que el generalmente no resuelve en su escritos publicados en la web, hay les va.




%%%%%%%%%%%%%%%%%%%%%%%%%%%%%%%%%%%%%%%%%%%%%%%%%%%
{\em {\color{introcolor}{ Usted menciona dentro de la charla algo que para muchos de nosotros como Colombianos ya es sabido y es que no tenemos independencia . żcual piensa usted que es el papel que juega o jugara el software Libre para lograr la independencia en un país como este? }}}

Lograr la independencia es {\em {\color{introcolor}{ una lucha mucho mas larga que la informática, }}}la migración a Software libre es una oportunidad para difundir ideas libertarias, de hacerse independiente del imperio, pero también es un campo en el cual es posible cobrar algo, esta lucha mas estrecha,{\em {\color{introcolor}{ puede fomentar la lucha mas amplia, }}} pero no puede conseguir solo la independencia de un país, hace falta un movimiento de democracia e independencia, que {\em {\color{introcolor}{ anulemos los tratados de libre exportación, }}}por que cada tratado de libre exportación es antidemocrático, la democracia no quiere decir tener elecciones, es un sistema, un método para la democracia.
{\em {\color{introcolor}{La democracia quiere decir que los mucho pobres se unen para hacer conjuntos mas fuertes que los ricos, }}}las elecciones son para que los mucho puedan decidir las políticas  del estado, pero solo tener elecciones no garantiza la democracia, los ricos tienen la influencia para decidir quien gane, no es democracia, {\em {\color{introcolor}{ el sistema de democracia necesita mantener débiles a los ricos o mantener que nadie sea tan rico, }}}ahora no hace falta que todos tengan la cantidad igual de bienes, seria imposible, seria una dictadura, pero hace falta asegurar que los ricos y las empresas no tengan mucho poder político, que hacen los tratados de libre explotación? Su meta es transferir o disminuir el poder del estado y aumentar el poder de las empresas, básicamente si un tratado facilita la mudanza de empresas de un país a otro, si el estado intenta reglamentar las empresas como se debe, las empresas pueden decir que no pueden reglamentarlas por que son de otro país, si el estado intenta proteger o el medio ambiente o la salud publica o el dinero de educación o el nivel general de la vida, o cualquier cosa mas importante que el negocio, el tratado de libre exportación es una arma para los negocios para oponerse a que el estado cumpla con sus funciones, pero mucho tratados de libre exportación van mas allá, por que ofrecen la posibilidad de demandar contra políticas necesarias del estado, como en los estados unidos han demandado contra una ley que prohibía la venta de cigarrillo con unos sabores ańadidos por que son medidas para hacer adictos a los nińos, mas a los adolescentes, el punto es que es una medida para la salud publica pero la organización mundial de comercio decidió considerarlo como un obstáculo al comercio, y el {\em {\color{introcolor}{ omercio es mas importante que la vida según esa organización, }}}s una organización asesina, que se acabe con esa organización.




\bibliographystyle{abbrv}
\begin{bibliografia}
\bibitem{kopka}
Conferencia Copyright Vs Comunidad en la era de las redes informáticas \emph{Richard Stallman},Bogota Colombia Agosto de 2012
\end{bibliografia} 

\begin{biografia}{images/mi_articulo/jinny.eps}{Jinny Daniel Salcedo Pulgarín} % ańadir fotografía tamańo [2.5 cm x 3.3 cm ]
Estudiante de Ingeniería Electrónica en la universidad Distrital Francisco José de Caldas, cursa actualmente sexto semestre, miembor activo del grupo GNU/Linux. activista del movimiento de tecnologias comunitaras, convencido que la tecnologia es una de las herramientas con las que se puede reducir la gran brecha solcial Colombiana.
\end{biografia}

\end{multicols} %termina el entorno multicols
%\eOpage %comienza una pagina nueva

%%rput(7.5,-2.0)
%{\resizebox{17cm}{!}{{\epsfbox{images/mi_articulo/final.eps}}}}

\clearpage
\pagebreak
