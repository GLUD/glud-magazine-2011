% Esta obra está bajo una licencia Reconocimiento 2.5 Espańa de Creative
% Commons. Para ver una copia de esta licencia, visite 
% http://creativecommons.org/licenses/by/2.5/es/
% o envie una carta a Creative Commons, 171 Second Street, Suite 300, 
% San Francisco, California 94105, USA.

% Seccion Introducción
%

\rput(2.5,-2.3){\resizebox{!}{5.7cm}{{\epsfbox{images/mi_articulo/inicio.eps}}}}%Imagen de el comienzo de el articulo, coordenadas desde 
                                                                                   %la parte superior izquierda del margen de la pagina

% -------------------------------------------------
% Cabecera
\begin{flushright}
\msection{introcolor}{black}{0.25}{INTRODUCCIÓN} %titulo de la sección

\mtitle{10cm}{Los videojuegos en el software libre } %titulo del articulo 

\msubtitle{8cm}{Como se abrieron camino los videojuegos en el software libre} %subtitulo

{\sf por Andrés Cobos} %autor

{\psset{linecolor=black,linestyle=dotted}\psline(-12,0)}
\end{flushright}

\vspace{2mm}
% -------------------------------------------------

\begin{multicols}{2}


% Introducción
\intro{introcolor}{E}{l mundo de los videojuegos es un mundo muy emocionante y entretenido,
cada ańo se hacen espectaculares entregas de nuevas ideas e innovaciones en este ambito.
Por desgracia desde que el mundo de los videojuegos se inició en los computadores, los usuarios linux no siempre
teníamos  la facilidad de acceder a este contenido o de encontrar videojuegos que fuesen libres
y los cuales pudiésemos modificar y aprender sobre como están hechos.
}

\vspace{2mm}

% Cuerpo del artículo

Desde la invención de los ordenadores, la tecnología ha hecho posible que estos cumplan
muchas funciones interesantes que día a día han ayudado a cumplir los obejtivos que la sociedad
se va poniendo en el transcurso de su desarrollo, pero el trabajo no lo es todo.\\

\sectiontext{white}{black}{Inicios} %como se hace una sección

Tras el fin de la segunda guerra mundial (1946), empezaron a existir las computadoras potentes
y genios de la programación sobre estas máquinas empezarón a crear programas que entretenían a 
las personas de alguna forma creativa, es así  como se crearon simuladores de ajedrez y diferentes
programas cuya función era proponer una actividad de aprendizaje sobre estos sistemas de computo.
La idea de este tipo de actividades había sido tan bien acogida que abrió el paso para que las compańías
fabricaran centros de computo para solo este tipo de actividades, en esta era nació la primera generación de
videoconsolas y con ellas el primer y uno de los juegos mas famosos, "El pong" publicado por la compańía Atari,
basado en el tennis de mesa o ping-pong el cual tuvo un éxito rotundo. Durante los ańos siguientes estas 
industrias crecieron 
desmesuradamente, pues la gente pagaba lo que fuese por entretenerse con estas nuevas maravillas tecnológicas,
a esta época comprendida entre 1978 y 1983 se le conoce como la edad de oro de los videojuegos.


\sectiontext{white}{black}{Primeros videojuegos} %sección

Pero que tiene que ver esto con el software libre?
Bien, como ya se mencionó anteriormente, las videoconsolas eran centros de computo que fueron evolucionando, 
hoy día una videoconsola es un computador con altas capacidades de procesamiento que utilizan un sistema operativo especificamente configurado.
Después de un tiempo, se hizo posible jugar a los videojuegos en nuestros computadores, pero existían dos problemas
para los usuarios del software libre, el primero de ellos, quienes tenían una máquina funcionando con su distribución 
de linux favorita, no podían encontrar juegos por que las compańías no programaban estos mismos con compatibilidad 
con el sistema de ventanas utilizado en linux (x11) .
El segundo de los problemas era que los pocos videojuegos que funcionaban en linux no eran muy conocidos


\sectiontext{white}{black}{The linux game tome} %sección
En respuesta a los problemas  mencionados anteriormente "The linux game tome" creada
por Tessa Lau en 1995, es  una web dedicada a publicar una lista de videojuegos existentes
para linux, su lista inicial consistió en recopilar los videojuegos compatibles con x11, lograron
reunir 100 videojuegos y gracias a los avances ahora su lista está cerca de los 2000.


"The linux game tome" hacía la recopilación de videojuegos que estaban disponibles para linux, 
mas sin embargo no todos eran software libre o no cumplian con las 4 libertades básicas del software 
libre, un tiempo después inició la aparición de una serie de entregas de videojuegos que cumplían con 
las 4 libertades es decir, eran software libre acá una pequeńa lista y descripción de algunos.

\sectiontext{white}{black}{The battle for wesnoth}
Juego multiplataforma, opera bajo las licencias GNU GPL, es un juego de estrategia por 
turnos, es una de las entrega	s mas importantes en lo juegos libres, cuenta con una gran comunidad de colaboradores.

\sectiontext{white}{black}{Warzone 2100}
Juego multiplataforma, opera bajo las licencias GNU GPL, es un juego de estrategia en tiempo real, 
este juego fue creado originalmente como privativo para Microsoft y PlayStation, desarrollado por pumpkin studios
en 1999 y posteriormente liberado bajo la licencia GNU en 2004 lo que lo convirtió en un juego libre.

\sectiontext{white}{black}{Super tuxkart}
Juego multiplataforma, opera bajo la licencia GPL, es un juego en 3 dimensiones de carreras
el mejor estilo de Mario Kart

\sectiontext{white}{black}{Nueva llegada}
Hasta estos éxitos todo se hizo con motores gráficos libres y no podíamos encontrar un juego estilo quake 3 
o similar, pero todo cambió cuando una empresa "Id Software" , creadora de "Id tech 3" el motor de Quake 3,juego
que solo estaba disponible en plataformas privativas, decide en 2005 liberar bajo la licencia GNU su motor "Id tech 3",
esta sin duda alguna ha sido uno de los mayores aportes a la industria de los videojuegos libres pues debido a su calidad
se crearon un sin número de adaptaciones a este motor, dando paso a una gran cantidad de shooters para linux, he aquí
algunos de ellos.

\sectiontext{white}{black}{Urban Terror}
Shooter que comenzó como una modificación de Quake 3 y se convirtió en un juego por si solo

\sectiontext{white}{black}{True combat Elite}
Al igual que Urban Terror es una modificación de un juego llamado Wolfestein,
otro shooter con algo mas de historia.

\sectiontext{white}{black}{Warsow}
Es un videojuego que cuenta con un conjunto de elementos futuristas, los gráficos son en forma de historieta, 
es un videojuego que resalta por su estética.

\sectiontext{white}{black}{apreciación}
Como podemos ver, se aprecia el aporte que hizo la compańía  Id Software al software libre ya que desde su 
liberación hasta la fecha lleva 8 ańos dando entretenimiento a todos los juegadores usuarios del sistema linux,
aunque ya es un poco antiguo, la empresa id software planea liberar su quinta versión es decir id tech5 bajo
la licencia de Open Source.

\sectiontext{white}{black}{Comercialización}
A pesar de este recorrido de las compańías entrando al mundo linux, falta una faceta por 
mencionar, la faceta de comercialización. Uno de las principales causas por las que no se juega
en entornos linux es por que no se conocen los videojuegos ni tampoco las soluciones dadas por
servicios como "The linux game tome" , con respecto a esto cabe destacar la solución que le ha dado Ubuntu a esto.
Ubuntu creó su propia interfaz de aplicaciones donde los videojuegos tienen su lugar, mostrando todo 
tipo de juegos existentes ya cumplan con las libertades del software libre o no.


Uno de los mas importantes hechos con respecto a este tema últimamente ha sido la entrada de Steam,
una plataforma de distribución digital de videojuegos fundada por valve corporation, esta es multiplataforma 
y es considerada una importante distribuidora por lo que su llegada a las aplicaciones de ubuntu causó revuelos.


\sectiontext{white}{black}{Hardware}
Por otra parte, como se menciono anteriormente, uno de los aspectos que dificultaron el desarrollo y ejecucion de videojuegos en Linux fue su falta de compatibilidad de hardware y la falta de drivers nativos de codigo libre, que aunque en cierta medida se pueden suplir con drivers privativos implican una imcompatibilidad de sowtware entre el fabricante y el usuario, presentando un problema de calidad.

\sectiontext{white}{black}{Emuladores}
Sin embargo uno de los mayores avances del sowtware libre en el campo de los videojuegos es el desarrollo de emuladores de consolas, programas que se valen de la ingenieria inversa para simular la arquitectura y funciones de una consola de videojuegos, que se han ido desarrollando con el paso de los ańos y no solo permiten el uso de videojuegos de consolas exintas y actuales, sino que tambien permiten la modificacion de estos y el desarrollo de videojuegos independientes para plataformas comercialmente obsoletas. 

\sectiontext{white}{black}{SCUMMVM}
Uno de los mayores exponetes de este tipo de programas es el SCUMMVM, un programa de codigo abierto que emula el motor SCUMM, usado para las aventuras graficas de LucasArts y otras empresas, puesto que no eran compatibles con el sofware moderno, este emulador incluso es usado por empresas como Atari, que adaptan aventuras graficas educativas a tiendas de consolas como Wiiware o Xbox live Arcade, aunque esto acarreo consecuensias legales.

\sectiontext{white}{black}{Comunidad homebrew}
Un aspecto a destacar es el de la comunidad homebrew, que modifica una videoconsola para que adquiera caracteristicas y utilidades que no tenia en el momentode su fabricacion, ya sea reproduccion de musica, reproduccion de videos, uso de copias de seguridad o instalacion de programas para emular otras consolas de videojuegos, aunque siempre ha sido perseguida en cierta medida por las grandes compańias fabricantes de videojuegos, nunca se ha comprobado que sea ilegal.

\sectiontext{white}{black}{Prototipos}
Pero todo esto ha llevado a la comunidad de software libre a crear sus propias videoconsolas, modificables y de codigo abierto, comenzando por pequeńas consolas portatiles como la Dingoo o la GPX32, cuya caracteristica mas destacable  se encuentra en los emuladores ya mencionados combinados con funciones propias de un mp4 e incluso navegacion en red, hasta proyectos ambiciosos como la EVO Smart, que pretende ser un centro de entretenimiento parecido a la PS3, pero tambien una plataforma para que los desarrolladores vean sus juegos en consolas e incluso suban la calidad de sus trabajos, pero sin duda el proyecto que mas expectacion genera es OUYA, una consola de codigo libre basada en android, que es la mas cercana a verse materializada gracias al apoyo que ha recibido.
Estas iniciativas de consolas de sobremesa han surgido a raiz de la desaparicion de las open source portatiles, que han perdido bastante su utilidad frente a los telefonos moviles android, que tienen practicamente las mismas prestaciones, empresas como Nvidia y Valve tambien planean lanzar consolas basadas en linux, pero sin tener una filosofia clara de como sera su uso.

\sectiontext{white}{black}{Una pequeńa conclusión}
A grandes rasgos, el proceso para que los usuarios de linux puedan tener entretenimiento en el area
de los videojuegos ha sido algo tortuoso, pero esto parece empezar a cambiar, solo queda esperar que nuevos
titulos llegan con el nuevo motor id tech5 y  ver el avance de los nuevos proyectos para consolas con otras filosofias.

\bibliographystyle{abbrv}
\begin{bibliografia}
\bibitem{Historia}
\url{http://es.wikipedia.org/wiki/Historia_de_los_videojuegos}.
\bibitem{kopka1}
\url{http://es.wikipedia.org/wiki/The_Linux_Game_Tome}.
\bibitem{kopka2}
\url{http://es.wikipedia.org/wiki/Id_Tech_3}.
\bibitem{kopka3}
\url{http://www.kdeblog.com/tag/linux}.
\end{bibliografia}


\begin{biografia}{images/mi_articulo/autor.eps}{Andres Hamir Cobos Prada} % ańadir fotografía tamańo [2.5 cm x 3.3 cm ]
Estudiante de Ingeniería de Sistemas en la universidad Distrital Francisco José de Caldas, cursa actualmente octavo semestre.  
\end{biografia}

\end{multicols} %termina el entorno multicols
%\eOpage %comienza una pagina nueva

%\rput(7.5,-2.0){\resizebox{10cm}{!}{{\epsfbox{images/mi_articulo/salsilla.eps}}}}

\clearpage
\pagebreak
