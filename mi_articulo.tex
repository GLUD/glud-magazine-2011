% Este fichero es parte del Número 5 de la Revista Occam's Razor
% Revista Occam's Razor Número 5
%
% (c) 2010 The Occam's Razor Team
%
% Esta obra está bajo una licencia Reconocimiento 2.5 Espańa de Creative
% Commons. Para ver una copia de esta licencia, visite 
% http://creativecommons.org/licenses/by/2.5/es/
% o envie una carta a Creative Commons, 171 Second Street, Suite 300, 
% San Francisco, California 94105, USA.

% Seccion Introducción
%

\rput(2.5,-2.3){\resizebox{!}{5.7cm}{{\epsfbox{images/mi_articulo/Kokopelli.eps}}}}%Imagen de el comienzo de el articulo, coordenadas desde 
                                                                             %la parte superior izquierda del margen de la pagina

% -------------------------------------------------
% Cabecera
\begin{flushright}
\msection{introcolor}{black}{0.25}{INTRODUCCIÓN} %titulo de la seccion

\mtitle{10cm}{El Sueńo Que se Gesta en una Mente}

\msubtitle{8cm}{Anécdotas desde el Comienzo}

{\sf por Chimini}

{\psset{linecolor=black,linestyle=dotted}\psline(-12,0)}
\end{flushright}

\vspace{2mm}
% -------------------------------------------------

\begin{multicols}{2}


% Introducción
\intro{introcolor}{L}{o complicado de un proyecto es la planificación 
que se hace necesaria para poder llevar este a un desarrollo constante, 
con la menor cantidad de contratiempos, la realización efectiva de todas 
y cada una de las actividades propuestas en el mismo y claro esta llevarlo 
a feliz termino o en su defecto a un mantenimiento constante y sostenible 
con la menor cantidad de revisiones y auditorías posibles.
}

\vspace{2mm}

% Cuerpo del artículo

Pero todo lo que he dicho en el párrafo anterior no es mas que teoría 
acerca de un proyecto, lo realmente importante como se menciona en la 
primera oración del mismo es, realizar la debida planificación sobre 
el proyecto, como me dirían un colega, "lo importante es saber hacer 
la gestión", y como tambien en algún momento de mi carrera de pregrado un profesor 
me dejo claro en una clase, "lo importante para la planificación de un 
proyecto es ver el futuro y lo que no se pueda predecir se tiene que 
imaginar para poder estar preparado para afrontar cualquier contratiempo, 
solucionarlo en la menor brevedad y optimizando el máximo de recursos.\\
Pues bien este par de cosas la he tenido muy en cuenta cuando se comenzó 
a planear este proyecto, {\em {\color{introcolor}{una revista}}}, cuando mi boca se abrió para 
pronunciar estas palabras en una de las reuniones del GLUD 
(grupo GNU/Linux de la universidad Distrital) para proporcionar una 
repuesta a la pregunta żmuchachos que otra alternativa podemos poner en 
practica para promover los proyectos y las ideas del software libre?, 
pensar en la idea de una revista fue fácil pero hacer la planificación y 
la gestión para que este proyecto dejara de ser un sueńo y se convirtiera 
en una realidad fue una tarea un poco mas compleja de lo que pensé en un 
principio.\\

\begin{entradilla}
{\em {\color{introcolor}{GLUD Magazine}} es un espacio para los que creen
que este cuento todavia puede cambiar y que la solucion esta en el software
libre y opensource.}
\end{entradilla}

\sectiontext{white}{black}{LA ELECCIÓN}

A continuación enumero algunas de las inquietudes y contratiempos que se presentaron en la puesta en 
marcha de la revista del GLUD. Primero digamos que cuando empeso no se sabia ni en que se haria la 
edicion de este proyecto, es decir no se sabia si utilizar un editor del tipo WYSIWYG o uno del tipo 
WYSIWYM, para aqueyos que no estan familiarizados con esta distincio entre editored de texto pues 
bueno aquí les va una explicacion muy basica. El primer temino es el acronimo de What You See Is 
What You Get, los editores de este tipo nos permiten ver la estructura final de un producto mientras lo 
estamos editando, mientras que los que son del tipo WYSIWYM (What You See Is What You Mean) 
son menos intuitivos en la parte de el producto final realmente lo que se hace cuando se edita un 
documento de este tipo es introducir en el editor la sintaxis que tendria que llevar el documento, por 
ejmplo: si se desea un titulo se coloca un comando que designe titulo y el contenido de el titulo y asi 
con el resto del documento como colocar un autor y el cuerpo del documento pero no hay que 
preocuparse de como queda la estructura visual del mismo porque de esta funcion se encarga el 
programa editor directamente. Entonces lo que tiene que hacer el usuario es preocuparce  del contenido 
que desea que aparesca en el documento. 

Los beneficios de un tipo de editor con respecto a otro quedan a criterio del usuario, pero en el 
planteamiento de un proyecto como este se desidio hacerlo con un editor que proporcionara una salida 
grafica visualmente muy  atractiva y que permitiera generar atributos especiales dentro del documento 
de una forma facil asi no fuera uy intuitiva. Asi que debido a este hecho se desidio que se haria con 
latex, puesto que este es un editor del tipo  WYSIWYM el documento tendria que tener  un codigo y 
ademas sus parted podrian ser desagregadas debido a que se podria hacer un documento que compilara 
las partes por separado y luego otro muy pequeńo que uinera estas partes, asi seria mas facil pasar el 
formato a quienes uisieran aportar a la revista y que ellos lo retornaran con su contenido editado para 
que puedira ser adicionado de una forma facil al documento final.

%%%%%%%%%%%%%%%%%%%%%%%%%%%%%%%%%%%%%%%%%%%%%%%%%%%%%%%%%%%%
\ebOpage{introcolor}{0.35}{INTRODUCCIÓN}
%%%%%%%%%%%%%%%%%%%%%%%%%%%%%%%%%%%%%%%%%%%%%%%%%%%%%%%%%%%%

\sectiontext{white}{black}{EL CÓDIGO}

En mi vida habia manejado latex para hacer edicion de documentos como papers, cartas, lo mas basico, 
pero nunca me habia dado a la tarea de hacer algo como una revista. Me encontre con algo de lo cual 
no vi su magnitud sino hasta que lo tube directamente frente a mi nariz, hacer una revista era un poco 
mas complicado de lo que habia pensado en el pricipio, asi que al ver que mi proposito no avanzaba lo 
suficiente decidi que era hora de buscar ejemplos. Alguien ya tiene que haber pensado en hacer una 
revista con Latex, por que no mirar su trabajo y basarme en ellos. Fue un aspecto complicado de llebar 
a cabo debido a que no encontre revistas de este estilo sino hasta que por casualidad di con una revista 
que era la realizacion del sueńo pero que ya otra persona habia hecho en un distante lugar. 

De esta forma llegue a tropesarme con la revista OCCAM'S RAZOR, la cual tenia exactamente el 
aspecto que deseaba para la revista del GLUD, pero esta adicionalmente tenia otro atractivo que me 
genero una gran felicidad, estaba licenciada bajo creative commons y ellos estipulaban que podia usar 
el codigo y lo unico que tenia que hacer era darles el reconocimiento de que ellos eran los realmente 
responsables de el mismo. Asi que dadoa todos estos hecho lo unico que faltaba por hacer era enviarles 
un correo de parte del GLUD pidiendo su permiso para la utilizacion del coodigo y la respuesta de ellos 
fue que no habia problema asi que proseguimos con el proyecto.

% A continuación un ejemplo de como se puede hacer una entradilla de código en 
% lenguaje C, de parte de los amigos de occam's razor.

%\lstset{language=C,frame=tb,framesep=5pt,basicstyle=\footnotesize}   
%\begin{lstlisting}
%#include <osgDB/ReadFile>
%#include <osgViewer/Viewer>
%
%using namespace osgDB;
%int main(int ac, char **a)
%{
%  osgViewer::Viewer viewer;
%  viewer.setSceneData(readNodeFile(a[1]));
%
%  return viewer.run();
%}
%\end{lstlisting}

En mi vida habia manejado latex para hacer edicion de documentos como papers, cartas, lo mas basico, 
pero nunca me habia dado a la tarea de hacer algo como una revista. Me encontre con algo de lo cual 
no vi su magnitud sino hasta que lo tube directamente frente a mi nariz, hacer una revista era un poco 
mas complicado de lo que habia pensado en el pricipio, asi que al ver que mi proposito no avanzaba lo 
suficiente decidi que era hora de buscar ejemplos. Alguien ya tiene que haber pensado en hacer una 
revista con Latex, por que no mirar su trabajo y basarme en ellos. Fue un aspecto complicado de llebar 
a cabo debido a que no encontre revistas de este estilo sino hasta que por casualidad di con una revista 
que era la realizacion del sueńo pero que ya otra persona habia hecho en un distante lugar. 


\begin{entradilla}
{\em {\color{introcolor}{Latex}} facilito las cosas en el desarrollo del proyecto
}
\end{entradilla}


De esta forma llegue a tropesarme con la revista OCCAM'S RAZOR, la cual tenia exactamente el 
aspecto que deseaba para la revista del GLUD, pero esta adicionalmente tenia otro atractivo que me 
genero una gran felicidad, estaba licenciada bajo creative commons y ellos estipulaban que podia usar 
el codigo y lo unico que tenia que hacer era darles el reconocimiento de que ellos eran los realmente 
responsables de el mismo. Asi que dadoa todos estos hecho lo unico que faltaba por hacer era enviarles 
un correo de parte del GLUD pidiendo su permiso para la utilizacion del coodigo y la respuesta de ellos 
fue que no habia problema asi que proseguimos con el proyecto.

% A continuación un ejemplo de como se puede hacer una entradilla de código en 
% lenguaje Make, de parte de los amigos de occam's razor.


%\lstset{language=Make,frame=tb,framesep=5pt,basicstyle=\footnotesize}   
%\begin{lstlisting}
%MY_CFLAGS=-I${DEV_DIR}/include
%MY_OSG_LIBS=-L${DEV_DIR}/lib -losgViewer
%
%mini: mini-viewer.cpp
%	g++ -o $@ $< ${MY_CFLAGS} ${MY_OSG_LIBS}
%\end{lstlisting}

\sectiontext{white}{black}{LA REALIZACIÓN}

Puede decirce que el resto de las cosas que me pasaron antes de este punto fueron lo mas compicado de 
todo el tema pero realmente poner en practica el codigo fue complicado, se volvio insdispensable 
estudiar el codigo para poder entender todas las maravillas que este prometia y poder utilizarlas de la 
mejor manera posible, asi que luego de estudiarlo a fondo el codigo se hizo posible producir un primer 
formato para edicion de articulos en la revista y es este el cual se envia a los interezados en realizar una 
publicacion en la misma. Agradeciendo denuevo a todas aquellas personas que apoyaron la realizacion 
de este proyecto y a nuestros amigos de occam's razor me despido y espero poder escribir algo de 
mayor interes tecnologico y menos filosofico en otra oportunidad, por ahora deseo que sigan leyendo la 
revista y que ojala la disfruten al maximo.
A continuación dejo una de las imagenes originales de los amigos de occam's para que tengan un ejemplo
de como insertar las suyas en sus articulos
\end{multicols} %termina el entorno multicols
%\eOpage %comienza una pagina nueva
\rput(7.5,-2.0){\resizebox{10cm}{!}{{\epsfbox{images/mi_articulo/salsilla.eps}}}}

%\clearpage
%\pagebreak
