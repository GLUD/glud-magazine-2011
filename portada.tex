% Este fihcero es una fiel copia de el Número 5 de la revista tecnologica Occam's Razor
% 
% Revista GLUD Número 0
%
% Esta obra está bajo una licencia Reconocimiento 2.5 Espańa de Creative
% Commons. Para ver una copia de esta licencia, visite 
% http://creativecommons.org/licenses/by/2.5/es/
% o envie una carta a Creative Commons, 171 Second Street, Suite 300, 
% San Francisco, California 94105, USA.


\documentclass[10pt,a4paper,twoside]{article}

% Paquetes... probablemente alguno no sea necesario
% 

\usepackage[utf8]{inputenc}                                                    
\usepackage[greek,spanish]{babel}  % Símbolo del euro
\usepackage{graphicx}
\usepackage{a4,fancyhdr, multicol}
\usepackage{float}
\usepackage{pdftricks}
\usepackage{pstricks}
\usepackage{color}
\usepackage{pst-plot}
\usepackage{pst-eps}
\usepackage{wrapfig}
\usepackage{eso-pic}
\usepackage{listings}
\usepackage{textpos}
\usepackage{epsf}
\usepackage{setspace}
\usepackage{hyperref}
\usepackage{colortbl}

\hypersetup{
    bookmarks=true,         % show bookmarks bar?
    colorlinks=true,        % false: boxed links; true: colored links
    linkcolor=red,          % color of internal links
    citecolor=green,        % color of links to bibliography
    filecolor=magenta,      % color of file links
    urlcolor=blue           % color of external links
}


\usepackage[T1]{fontenc} 


% Configuración de tamańo de página
\setlength{\parindent}{0in}
\setlength{\parskip}{0.1cm}
\setlength{\oddsidemargin}{0.05mm}
\setlength{\evensidemargin}{0.05mm}

\addtolength{\textwidth}{4cm}
\addtolength{\topmargin}{-3.5cm}
\addtolength{\textheight}{4.5cm}

\pagestyle{fancy}

% Configuración de Cabeceras Fancy Header
\fancyhead{}
\fancyfoot{} % clear all footer fields
\fancyfoot[LE]{\textbf{\textsf{GLUD Magazine | \thepage}}}
\fancyfoot[RO]{\textbf{\textsf{\thepage | GLUD Magazine}}}
\renewcommand{\footrulewidth}{0.4pt}

% Elimina l.neas de cabecera
% 
\renewcommand{\headrulewidth}{0pt}

% Colores
\definecolor{introcolor}{rgb}{0.04,0.09,0.16}
\definecolor{listcolor}{rgb}{0.2,0.4,0.2}
\definecolor{titlecolor}{rgb}{0.4,0.5,0.1}
\definecolor{excolor}{rgb}{0.8,0.8,0.8}
\definecolor{notecolor}{rgb}{0.4,0.4,1.0}
\definecolor{encabezado}{rgb}{0.384,0.497,0.5}

% ********************************************************
% Definición de Comandos y Entornos
% ********************************************************

% Comandos de uso general
% ---------------------------------------------------------
% Secciones títulos y subtítulos de cada página
\newcommand{\msection}[4]{
{\begin{flushright}{
{\psset{linecolor=black,linestyle=dotted}\psline(-17,0)}
\colorbox{#1}{
\begin{minipage}{#3\linewidth}
\center
  \textcolor{#2}{
    \textsf{\textbf{#4}}}
\end{minipage}
}}\end{flushright}}

\vspace{4mm}
}

\newcommand{\mtitle}[2]{
  {\resizebox{#1}{0.7cm}{\textbf{\textsf{#2}}}}
  \vspace{1mm}
}


\newcommand{\msubtitle}[2]{
  {\resizebox{#1}{0.5cm}{{\gray{\textbf{\textsf{#2}}}}}}
  \vspace{1mm}
}

% Principio de Página. Pone el cuadro superior con la sección
\newcommand{\bOpage}[3]{
  \msection{#1}{black}{#2}{#3}
  \begin{multicols}{2}
}

% Fin de página. Termina el entorno multicols
\newcommand{\eOpage}{
\pagebreak
\end{multicols}
}

% Fin e Inicio de Página. Sino utilizamos figuras fuera de las
% columnas del cuerpo principal, esta es la forma adecuada de marcar
% cada página
\newcommand{\ebOpage}[3]{
\eOpage
\bOpage{#1}{#2}{#3}
}

% Crea el cuadro de introducción al principio de cada artículo
\newcommand{\intro}[3]{
\colorbox{#1}{
  \begin{minipage}{.9\linewidth}
    \vspace{2mm}
    {{\resizebox{!}{1.0cm}{#2}}{#3}}
  \vspace{1mm}
  \end{minipage}
}
\vspace{4mm}
}


% Comando para introducir figura en entorno multicol
\newcommand{\myfig}[3]{
\begin{center}
  \includegraphics[width=#3\hsize,angle=#1]{#2}
  \nobreak
\end{center}}

% Caption para figuras en entorno multicol
\setcounter{figure}{1}
\newcommand{\mycaption}[1]{
  \begin{quote}
    {\small
    {{\sc Figura} \arabic{figure}: #1}
    }
  \end{quote}
  \stepcounter{figure}
}


% Comandos para utilizar tablas y figuras en entorno multicols
\makeatletter
\newenvironment{tablehere}
  {\def\@captype{table}}
  {}

\newenvironment{figurehere}
  {\def\@captype{figure}}
  {}
\makeatother

% Caption para figuras en entorno multicol sin contador
\newcommand{\nncaption}[1]{
  %\begin{quote}
    {\footnotesize{\textbf{
    {#1}
    }}}
  %\end{quote}
}

\newcommand{\sectiontext}[3]{\vspace{4mm}{{\textcolor{titlecolor}{\large{\textbf{\textsf{#3}}}}}}
\vspace{1mm}
}


\newcommand{\EOP}{\psframe[fillstyle=solid, fillcolor=titlecolor, linecolor=titlecolor](0,0)(4pt,4pt)}


% Entornos (begin... end)
% ----------------------------------------------------
% Entorno para introducir ejemplos
\newenvironment{mexample}{
  \vspace{2mm}
  \bgroup
  \tiny
}{
  \egroup
  \vspace{4mm}
}

% Entorno para introducir entradillas en el texto
\newenvironment{entradilla}{
  \vspace{5mm}
  \hrule 
  \vspace{2mm}
  \bgroup
  \LARGE
\begin{spacing}{0.6}
}{
\end{spacing}
  \egroup
  \vspace{5mm}
  \hrule
  \vspace{5mm}
}
% Entorno para introducir Biografia del Autor.
\newenvironment{biografia}[2]{
\vspace{10mm}
\hrule
\begin{minipage}[t]{3cm}
\myfig{0}{#1}{1} 
\end{minipage}
%\includegraphics[width=2.54 \hsize]{#1}
  \nobreak
\begin{minipage}[r]{5cm}
 \bgroup
  \small
\textbf{#2}
}{
  \egroup
  \vspace{3mm}
\end{minipage}
  \hrule
  \vspace{5mm}
}
% Entorno para introducir la bibliografia del articulo.
\newenvironment{bibliografia}{
\begin{thebibliography}{1}
}{
\end{thebibliography}
}



% **************************************************************************
% Comienza el documento
\begin{document}

% Portada, no utiliza Fancy Header e introduce imagen de portada con PStricks
\pagestyle{empty}

\rput(8,-14){{\resizebox{!}{30cm}{\epsfbox{images/portada_n1.eps}}}}

\clearpage
\pagebreak

% Imagen con el sumario en la siguiente página
\rput(8,-14.0){\resizebox{!}{30cm}{{\epsfbox{images/sumario1.eps}}}}


\pagebreak

% Activa Fancy Headers stilo e incluye los distintos artículos
\pagestyle{fancy}

\definecolor{introcolor}{rgb}{0.384,0.497,0.685}

% Este fichero es parte del Número 5 de la Revista Occam's Razor
% Revista Occam's Razor Número 5
%
% (c)  2010 The Occam's Razor Team
%
% Esta obra está bajo una licencia Reconocimiento 2.5 Espańa de Creative
% Commons. Para ver una copia de esta licencia, visite 
% http://creativecommons.org/licenses/by/2.5/es/
% o envie una carta a Creative Commons, 171 Second Street, Suite 300, 
% San Francisco, California 94105, USA.

%\rput(10.0,-19.0){\resizebox{10cm}{!}{{\epsfbox{images/general/feliz_2011.eps}}}}
\rput(8,-1.6){\resizebox{!}{5cm}{{\epsfbox{images/general/editorial.eps}}}}
\rput(0.0,-13){\resizebox{7cm}{35.0cm}{{\epsfbox{images/general/bar1.eps}}}}
\rput(0.4,-4.5){\resizebox{!}{4.8cm}{{\epsfbox{images/portada_editorial.eps}}}}
\rput(0.7,-26.5){\resizebox{!}{0.9cm}{{\epsfbox{images/general/licencia.eps}}}}


\begin{textblock}{9.2}(3,0)
\begin{flushright}
{\resizebox{!}{1cm}{\textsc{Editorial}}}

\vspace{2mm}

{\LARGE Welcome to the jungle\\ GLUD magazine!}

by GLUD
\end{flushright}
\end{textblock}

\vspace{4mm}


\definecolor{barcolor}{rgb}{0.044,0.09,0.16}

\begin{textblock}{30}(-1.5, -1)
\begin{minipage}{0.12\linewidth}
\sf\color{barcolor}
\begin{center}

\vspace{1cm}

\colorbox{black}{
{\resizebox{3cm}{0.7cm}{\textcolor{white}{\bf\sf\large GLUD}}}
}

{\resizebox{2.5cm}{0.4cm}{\bf\sf\large Magazine}}

\vspace{3mm}

{\bf Número 1. Septiembre 2011}

\vspace{5cm}

\hrule

\vspace{3mm}

{\bf Dirección: }

\vspace{1mm}

Wilmar Fernando Pineda 

\vspace{2mm}

{\bf Editor:}

\vspace{1mm}

Wilmar Fernando Pineda 

Y Otros

\vspace{4mm}

{\bf Comite Editorial:}

\vspace{1mm}

Es necesario llenar esta parte con algunos nombres de las personas que deseen colaborar con esto.


\vspace{4.0mm}

{\bf Maquetación y Grafismo}

\vspace{1mm}

Este espacio esta reservado para quien este interesado en hacer el desarrollor de la portada, la contraportada
 y además nos desee colaborar con el diseńo grafico que concierne a la revista. 

\vspace{1mm}

\hrule

\vspace{2mm}

{\bf Publicidad}

\vspace{1mm}

Tambien podemos hacer un poco de publicidad

{\tt occams-razor@uvigo.es}

\vspace{2mm}

\hrule

\vspace{4mm}

{\bf Impresión}

Por ahora tu mismo\ldots Si te apetece

\vspace{2mm}

\hrule

\vspace{6mm}

\copyright  2011 GLUD

Esta obra está bajo una licencia Reconocimiento 2.5 Espańa de Creative
Commons. Para ver una copia de esta licencia, visite 

{\scriptsize http://creativecommons.org/licenses/by/2.5/es/} 

o envie una carta a Creative Commons, 171 Second Street, Suite 300, San Francisco, California 94105, USA.

\medskip

%{\color{black}{\textbf{Consulta la página 32 para las excepciones a esta licencia}}}

\end{center}
\end{minipage}

\end{textblock}

\begin{textblock}{20}(3,2.0)

\begin{minipage}{.45\linewidth}
\colorbox{introcolor}{
\begin{minipage}{1\linewidth}

{{\resizebox{!}{1.0cm}{E}}{n el primer numero de la revista del GLUD (Grupo GNU/Linux de la universidad Distrital) 
quisiéramos presentar nuestros mas sinceros agradecimientos a los desarrolladores de la 
idea del código fuente de la misma, es decir a nuestros amigos los gestores y encargados 
de mantener en funcionamiento la revista Occam's Razor, los cuales con su aporte hacen 
posible que no solo este sino que otros proyectos se desarrollen como una alternativa a 
la edición de documentos y que mejor que documentos como una revista de divulgación. Para 
concluir con el reconocimiento a nuestras amigos de Occam's Razor queremos desearles 
muchos éxitos en sus próximos números, que no olviden revisar nuestro proyecto y gracias 
muchachos por su colaboración en el área de las fuentes de Latex. Y a los lectores que no 
olviden visitar la pagina oficial del proyecto que es: http://webs.uvigo.es/occams-razor.





\bigskip

}
}

\end{minipage}
}

\vspace{6mm}

Ahora pasando a lo concerniente al proyecto de GLUD Magazine que es un proyecto el cual no 
retoma la idea original de el padre de ser una revista de tecnología, sino que en esta 
enfatiza netamente en los conceptos de software  libre y de código abierto. Es decir los 
temas que son tratados en esta generalmente se hacen en sistemas GNU/Linux acerca de: desarrollo 
de software, proyectos culturales, difusión, instalación, configuración, apropiación, traducciones, 
capacitaciones y filosofía. Todos estos tema relacionados claro con los movimientos  de el software 
libre y el open source o código abierto en nuestro idioma.\\ 

La idea de la revista nace con la finalidad de dar voz a todos aquellos que no la tienen y que merecen
un espacio en donde puedan plasmar su conocimiento, para que el mundo no se pierda de lo mucho
que estas personas puede o podemos dar, por que si estas leyendo esto significa que tu y todos a
quienes puedas distribuir este mensaje están invitados a publicar en la revista del GLUD, solo recuerda
que no hay un conocimiento pequeńo y que siempre habrá en el mundo alguien que valore lo que
haces, así que vamos a colaborar con el proyecto ya que esta pequeńa que ahora tienes en las manos es
un  proyecto joven y necesita de muchos aportes para poder darle al mundo algo de el conocimiento
del que se ha perdido porque no había un espacio en el cual plasmarlo.\\

Así que si tienes un proyecto o cualquier tema relacionado con respecto al software y la cultura libre y
open source te invitamos a que te contactes con nosotros en glud@udistrital.edu.co y que visites
nuestro sitio en internet que es, http://glud.org. Y si no sabes mucho del tema te invitamos a continuar
leyendo ya que hay unos artículos a continuación que te pueden aclarar muchas cosas acerca del
mismo.\\ 

Así que no siendo mas nosotros el GLUD y los amigos de Occam's Razor te invitamos a que sigas leyendo 
esta publicación y esperamos que la disfrutes.\\

\bigskip

%Deseamos que la distrutéis y ....

%\medskip
%{\large Feliz 2011!}

\begin{flushright}
{\Large\sc{GLUD \\}}
\end{flushright}


\vspace{5cm}

\bigskip


\end{minipage}

\bigskip

\colorbox{introcolor}{
\begin{minipage}{.45\linewidth}

\bigskip

{\footnotesize\sf{\color{white}
Las opiniones expresadas en los artículos, así como los contenidos de
los mismos, son responsabilidad de los autores de éstos.

Puede obtener la versión electrónica de esta publicación, así como el
{\em código fuente} de la misma y los distintos ficheros de datos
asociados a cada artículo en el sitio web:

{\tt http://webs.uvigo.es/occams-razor}

}}

\bigskip

\end{minipage}
}


\end{textblock}

\pagebreak


% A continuación se colocan los articulos de la revista.

% Este fichero es parte del Número 5 de la Revista Occam's Razor
% Revista Occam's Razor Número 5
%
% (c) 2010 The Occam's Razor Team
%
% Esta obra está bajo una licencia Reconocimiento 2.5 Espańa de Creative
% Commons. Para ver una copia de esta licencia, visite 
% http://creativecommons.org/licenses/by/2.5/es/
% o envie una carta a Creative Commons, 171 Second Street, Suite 300, 
% San Francisco, California 94105, USA.

% Seccion Introducción
%

\rput(2.5,-2.3){\resizebox{!}{5.7cm}{{\epsfbox{images/mi_articulo/Kokopelli.eps}}}}%Imagen de el comienzo de el articulo, coordenadas desde 
                                                                             %la parte superior izquierda del margen de la pagina

% -------------------------------------------------
% Cabecera
\begin{flushright}
\msection{introcolor}{black}{0.25}{INTRODUCCIÓN} %titulo de la seccion

\mtitle{10cm}{El Sueńo Que se Gesta en una Mente}

\msubtitle{8cm}{Anécdotas desde el Comienzo}

{\sf por Chimini}

{\psset{linecolor=black,linestyle=dotted}\psline(-12,0)}
\end{flushright}

\vspace{2mm}
% -------------------------------------------------

\begin{multicols}{2}


% Introducción
\intro{introcolor}{L}{o complicado de un proyecto es la planificación 
que se hace necesaria para poder llevar este a un desarrollo constante, 
con la menor cantidad de contratiempos, la realización efectiva de todas 
y cada una de las actividades propuestas en el mismo y claro esta llevarlo 
a feliz termino o en su defecto a un mantenimiento constante y sostenible 
con la menor cantidad de revisiones y auditorías posibles.
}

\vspace{2mm}

% Cuerpo del artículo

Pero todo lo que he dicho en el párrafo anterior no es mas que teoría 
acerca de un proyecto, lo realmente importante como se menciona en la 
primera oración del mismo es, realizar la debida planificación sobre 
el proyecto, como me dirían un colega, "lo importante es saber hacer 
la gestión", y como tambien en algún momento de mi carrera de pregrado un profesor 
me dejo claro en una clase, "lo importante para la planificación de un 
proyecto es ver el futuro y lo que no se pueda predecir se tiene que 
imaginar para poder estar preparado para afrontar cualquier contratiempo, 
solucionarlo en la menor brevedad y optimizando el máximo de recursos.\\
Pues bien este par de cosas la he tenido muy en cuenta cuando se comenzó 
a planear este proyecto, {\em {\color{introcolor}{una revista}}}, cuando mi boca se abrió para 
pronunciar estas palabras en una de las reuniones del GLUD 
(grupo GNU/Linux de la universidad Distrital) para proporcionar una 
repuesta a la pregunta żmuchachos que otra alternativa podemos poner en 
practica para promover los proyectos y las ideas del software libre?, 
pensar en la idea de una revista fue fácil pero hacer la planificación y 
la gestión para que este proyecto dejara de ser un sueńo y se convirtiera 
en una realidad fue una tarea un poco mas compleja de lo que pensé en un 
principio.\\

\begin{entradilla}
{\em {\color{introcolor}{GLUD Magazine}} es un espacio para los que creen
que este cuento todavia puede cambiar y que la solucion esta en el software
libre y opensource.}
\end{entradilla}

\sectiontext{white}{black}{LA ELECCIÓN}

A continuación enumero algunas de las inquietudes y contratiempos que se presentaron en la puesta en 
marcha de la revista del GLUD. Primero digamos que cuando empeso no se sabia ni en que se haria la 
edicion de este proyecto, es decir no se sabia si utilizar un editor del tipo WYSIWYG o uno del tipo 
WYSIWYM, para aqueyos que no estan familiarizados con esta distincio entre editored de texto pues 
bueno aquí les va una explicacion muy basica. El primer temino es el acronimo de What You See Is 
What You Get, los editores de este tipo nos permiten ver la estructura final de un producto mientras lo 
estamos editando, mientras que los que son del tipo WYSIWYM (What You See Is What You Mean) 
son menos intuitivos en la parte de el producto final realmente lo que se hace cuando se edita un 
documento de este tipo es introducir en el editor la sintaxis que tendria que llevar el documento, por 
ejmplo: si se desea un titulo se coloca un comando que designe titulo y el contenido de el titulo y asi 
con el resto del documento como colocar un autor y el cuerpo del documento pero no hay que 
preocuparse de como queda la estructura visual del mismo porque de esta funcion se encarga el 
programa editor directamente. Entonces lo que tiene que hacer el usuario es preocuparce  del contenido 
que desea que aparesca en el documento. 

Los beneficios de un tipo de editor con respecto a otro quedan a criterio del usuario, pero en el 
planteamiento de un proyecto como este se desidio hacerlo con un editor que proporcionara una salida 
grafica visualmente muy  atractiva y que permitiera generar atributos especiales dentro del documento 
de una forma facil asi no fuera uy intuitiva. Asi que debido a este hecho se desidio que se haria con 
latex, puesto que este es un editor del tipo  WYSIWYM el documento tendria que tener  un codigo y 
ademas sus parted podrian ser desagregadas debido a que se podria hacer un documento que compilara 
las partes por separado y luego otro muy pequeńo que uinera estas partes, asi seria mas facil pasar el 
formato a quienes uisieran aportar a la revista y que ellos lo retornaran con su contenido editado para 
que puedira ser adicionado de una forma facil al documento final.

%%%%%%%%%%%%%%%%%%%%%%%%%%%%%%%%%%%%%%%%%%%%%%%%%%%%%%%%%%%%
\ebOpage{introcolor}{0.35}{INTRODUCCIÓN}
%%%%%%%%%%%%%%%%%%%%%%%%%%%%%%%%%%%%%%%%%%%%%%%%%%%%%%%%%%%%

\sectiontext{white}{black}{EL CÓDIGO}

En mi vida habia manejado latex para hacer edicion de documentos como papers, cartas, lo mas basico, 
pero nunca me habia dado a la tarea de hacer algo como una revista. Me encontre con algo de lo cual 
no vi su magnitud sino hasta que lo tube directamente frente a mi nariz, hacer una revista era un poco 
mas complicado de lo que habia pensado en el pricipio, asi que al ver que mi proposito no avanzaba lo 
suficiente decidi que era hora de buscar ejemplos. Alguien ya tiene que haber pensado en hacer una 
revista con Latex, por que no mirar su trabajo y basarme en ellos. Fue un aspecto complicado de llebar 
a cabo debido a que no encontre revistas de este estilo sino hasta que por casualidad di con una revista 
que era la realizacion del sueńo pero que ya otra persona habia hecho en un distante lugar. 

De esta forma llegue a tropesarme con la revista OCCAM'S RAZOR, la cual tenia exactamente el 
aspecto que deseaba para la revista del GLUD, pero esta adicionalmente tenia otro atractivo que me 
genero una gran felicidad, estaba licenciada bajo creative commons y ellos estipulaban que podia usar 
el codigo y lo unico que tenia que hacer era darles el reconocimiento de que ellos eran los realmente 
responsables de el mismo. Asi que dadoa todos estos hecho lo unico que faltaba por hacer era enviarles 
un correo de parte del GLUD pidiendo su permiso para la utilizacion del coodigo y la respuesta de ellos 
fue que no habia problema asi que proseguimos con el proyecto.

% A continuación un ejemplo de como se puede hacer una entradilla de código en 
% lenguaje C, de parte de los amigos de occam's razor.

%\lstset{language=C,frame=tb,framesep=5pt,basicstyle=\footnotesize}   
%\begin{lstlisting}
%#include <osgDB/ReadFile>
%#include <osgViewer/Viewer>
%
%using namespace osgDB;
%int main(int ac, char **a)
%{
%  osgViewer::Viewer viewer;
%  viewer.setSceneData(readNodeFile(a[1]));
%
%  return viewer.run();
%}
%\end{lstlisting}

En mi vida habia manejado latex para hacer edicion de documentos como papers, cartas, lo mas basico, 
pero nunca me habia dado a la tarea de hacer algo como una revista. Me encontre con algo de lo cual 
no vi su magnitud sino hasta que lo tube directamente frente a mi nariz, hacer una revista era un poco 
mas complicado de lo que habia pensado en el pricipio, asi que al ver que mi proposito no avanzaba lo 
suficiente decidi que era hora de buscar ejemplos. Alguien ya tiene que haber pensado en hacer una 
revista con Latex, por que no mirar su trabajo y basarme en ellos. Fue un aspecto complicado de llebar 
a cabo debido a que no encontre revistas de este estilo sino hasta que por casualidad di con una revista 
que era la realizacion del sueńo pero que ya otra persona habia hecho en un distante lugar. 


\begin{entradilla}
{\em {\color{introcolor}{Latex}} facilito las cosas en el desarrollo del proyecto
}
\end{entradilla}


De esta forma llegue a tropesarme con la revista OCCAM'S RAZOR, la cual tenia exactamente el 
aspecto que deseaba para la revista del GLUD, pero esta adicionalmente tenia otro atractivo que me 
genero una gran felicidad, estaba licenciada bajo creative commons y ellos estipulaban que podia usar 
el codigo y lo unico que tenia que hacer era darles el reconocimiento de que ellos eran los realmente 
responsables de el mismo. Asi que dadoa todos estos hecho lo unico que faltaba por hacer era enviarles 
un correo de parte del GLUD pidiendo su permiso para la utilizacion del coodigo y la respuesta de ellos 
fue que no habia problema asi que proseguimos con el proyecto.

% A continuación un ejemplo de como se puede hacer una entradilla de código en 
% lenguaje Make, de parte de los amigos de occam's razor.


%\lstset{language=Make,frame=tb,framesep=5pt,basicstyle=\footnotesize}   
%\begin{lstlisting}
%MY_CFLAGS=-I${DEV_DIR}/include
%MY_OSG_LIBS=-L${DEV_DIR}/lib -losgViewer
%
%mini: mini-viewer.cpp
%	g++ -o $@ $< ${MY_CFLAGS} ${MY_OSG_LIBS}
%\end{lstlisting}

\sectiontext{white}{black}{LA REALIZACIÓN}

Puede decirce que el resto de las cosas que me pasaron antes de este punto fueron lo mas compicado de 
todo el tema pero realmente poner en practica el codigo fue complicado, se volvio insdispensable 
estudiar el codigo para poder entender todas las maravillas que este prometia y poder utilizarlas de la 
mejor manera posible, asi que luego de estudiarlo a fondo el codigo se hizo posible producir un primer 
formato para edicion de articulos en la revista y es este el cual se envia a los interezados en realizar una 
publicacion en la misma. Agradeciendo denuevo a todas aquellas personas que apoyaron la realizacion 
de este proyecto y a nuestros amigos de occam's razor me despido y espero poder escribir algo de 
mayor interes tecnologico y menos filosofico en otra oportunidad, por ahora deseo que sigan leyendo la 
revista y que ojala la disfruten al maximo.
A continuación dejo una de las imagenes originales de los amigos de occam's para que tengan un ejemplo
de como insertar las suyas en sus articulos
\end{multicols} %termina el entorno multicols
%\eOpage %comienza una pagina nueva
\rput(7.5,-2.0){\resizebox{10cm}{!}{{\epsfbox{images/mi_articulo/salsilla.eps}}}}

%\clearpage
%\pagebreak

% Esta obra está bajo una licencia Reconocimiento 2.5 Espańa de Creative
% Commons. Para ver una copia de esta licencia, visite 
% http://creativecommons.org/licenses/by/2.5/es/
% o envie una carta a Creative Commons, 171 Second Street, Suite 300, 
% San Francisco, California 94105, USA.

% Seccion Introducción
%

\rput(2.5,-2.3){\resizebox{!}{5.7cm}{{\epsfbox{images/mi_articulo/inicio.eps}}}}%Imagen de el comienzo de el articulo, coordenadas desde 
                                                                                   %la parte superior izquierda del margen de la pagina

% -------------------------------------------------
% Cabecera
\begin{flushright}
\msection{introcolor}{black}{0.25}{INTRODUCCIÓN} %titulo de la sección

\mtitle{10cm}{Los videojuegos en el software libre } %titulo del articulo 

\msubtitle{8cm}{Como se abrieron camino los videojuegos en el software libre} %subtitulo

{\sf por Andrés Cobos} %autor

{\psset{linecolor=black,linestyle=dotted}\psline(-12,0)}
\end{flushright}

\vspace{2mm}
% -------------------------------------------------

\begin{multicols}{2}


% Introducción
\intro{introcolor}{E}{l mundo de los videojuegos es un mundo muy emocionante y entretenido,
cada ańo se hacen espectaculares entregas de nuevas ideas e innovaciones en este ambito.
Por desgracia desde que el mundo de los videojuegos se inició en los computadores, los usuarios linux no siempre
teníamos  la facilidad de acceder a este contenido o de encontrar videojuegos que fuesen libres
y los cuales pudiésemos modificar y aprender sobre como están hechos.
}

\vspace{2mm}

% Cuerpo del artículo

Desde la invención de los ordenadores, la tecnología ha hecho posible que estos cumplan
muchas funciones interesantes que día a día han ayudado a cumplir los obejtivos que la sociedad
se va poniendo en el transcurso de su desarrollo, pero el trabajo no lo es todo.\\

\sectiontext{white}{black}{Inicios} %como se hace una sección

Tras el fin de la segunda guerra mundial (1946), empezaron a existir las computadoras potentes
y genios de la programación sobre estas máquinas empezarón a crear programas que entretenían a 
las personas de alguna forma creativa, es así  como se crearon simuladores de ajedrez y diferentes
programas cuya función era proponer una actividad de aprendizaje sobre estos sistemas de computo.
La idea de este tipo de actividades había sido tan bien acogida que abrió el paso para que las compańías
fabricaran centros de computo para solo este tipo de actividades, en esta era nació la primera generación de
videoconsolas y con ellas el primer y uno de los juegos mas famosos, "El pong" publicado por la compańía Atari,
basado en el tennis de mesa o ping-pong el cual tuvo un éxito rotundo. Durante los ańos siguientes estas 
industrias crecieron 
desmesuradamente, pues la gente pagaba lo que fuese por entretenerse con estas nuevas maravillas tecnológicas,
a esta época comprendida entre 1978 y 1983 se le conoce como la edad de oro de los videojuegos.


\sectiontext{white}{black}{Primeros videojuegos} %sección

Pero que tiene que ver esto con el software libre?
Bien, como ya se mencionó anteriormente, las videoconsolas eran centros de computo que fueron evolucionando, 
hoy día una videoconsola es un computador con altas capacidades de procesamiento que utilizan un sistema operativo especificamente configurado.
Después de un tiempo, se hizo posible jugar a los videojuegos en nuestros computadores, pero existían dos problemas
para los usuarios del software libre, el primero de ellos, quienes tenían una máquina funcionando con su distribución 
de linux favorita, no podían encontrar juegos por que las compańías no programaban estos mismos con compatibilidad 
con el sistema de ventanas utilizado en linux (x11) .
El segundo de los problemas era que los pocos videojuegos que funcionaban en linux no eran muy conocidos


\sectiontext{white}{black}{The linux game tome} %sección
En respuesta a los problemas  mencionados anteriormente "The linux game tome" creada
por Tessa Lau en 1995, es  una web dedicada a publicar una lista de videojuegos existentes
para linux, su lista inicial consistió en recopilar los videojuegos compatibles con x11, lograron
reunir 100 videojuegos y gracias a los avances ahora su lista está cerca de los 2000.


"The linux game tome" hacía la recopilación de videojuegos que estaban disponibles para linux, 
mas sin embargo no todos eran software libre o no cumplian con las 4 libertades básicas del software 
libre, un tiempo después inició la aparición de una serie de entregas de videojuegos que cumplían con 
las 4 libertades es decir, eran software libre acá una pequeńa lista y descripción de algunos.

\sectiontext{white}{black}{The battle for wesnoth}
Juego multiplataforma, opera bajo las licencias GNU GPL, es un juego de estrategia por 
turnos, es una de las entrega	s mas importantes en lo juegos libres, cuenta con una gran comunidad de colaboradores.

\sectiontext{white}{black}{Warzone 2100}
Juego multiplataforma, opera bajo las licencias GNU GPL, es un juego de estrategia en tiempo real, 
este juego fue creado originalmente como privativo para Microsoft y PlayStation, desarrollado por pumpkin studios
en 1999 y posteriormente liberado bajo la licencia GNU en 2004 lo que lo convirtió en un juego libre.

\sectiontext{white}{black}{Super tuxkart}
Juego multiplataforma, opera bajo la licencia GPL, es un juego en 3 dimensiones de carreras
el mejor estilo de Mario Kart

\sectiontext{white}{black}{Nueva llegada}
Hasta estos éxitos todo se hizo con motores gráficos libres y no podíamos encontrar un juego estilo quake 3 
o similar, pero todo cambió cuando una empresa "Id Software" , creadora de "Id tech 3" el motor de Quake 3,juego
que solo estaba disponible en plataformas privativas, decide en 2005 liberar bajo la licencia GNU su motor "Id tech 3",
esta sin duda alguna ha sido uno de los mayores aportes a la industria de los videojuegos libres pues debido a su calidad
se crearon un sin número de adaptaciones a este motor, dando paso a una gran cantidad de shooters para linux, he aquí
algunos de ellos.

\sectiontext{white}{black}{Urban Terror}
Shooter que comenzó como una modificación de Quake 3 y se convirtió en un juego por si solo

\sectiontext{white}{black}{True combat Elite}
Al igual que Urban Terror es una modificación de un juego llamado Wolfestein,
otro shooter con algo mas de historia.

\sectiontext{white}{black}{Warsow}
Es un videojuego que cuenta con un conjunto de elementos futuristas, los gráficos son en forma de historieta, 
es un videojuego que resalta por su estética.

\sectiontext{white}{black}{apreciación}
Como podemos ver, se aprecia el aporte que hizo la compańía  Id Software al software libre ya que desde su 
liberación hasta la fecha lleva 8 ańos dando entretenimiento a todos los juegadores usuarios del sistema linux,
aunque ya es un poco antiguo, la empresa id software planea liberar su quinta versión es decir id tech5 bajo
la licencia de Open Source.

\sectiontext{white}{black}{Comercialización}
A pesar de este recorrido de las compańías entrando al mundo linux, falta una faceta por 
mencionar, la faceta de comercialización. Uno de las principales causas por las que no se juega
en entornos linux es por que no se conocen los videojuegos ni tampoco las soluciones dadas por
servicios como "The linux game tome" , con respecto a esto cabe destacar la solución que le ha dado Ubuntu a esto.
Ubuntu creó su propia interfaz de aplicaciones donde los videojuegos tienen su lugar, mostrando todo 
tipo de juegos existentes ya cumplan con las libertades del software libre o no.


Uno de los mas importantes hechos con respecto a este tema últimamente ha sido la entrada de Steam,
una plataforma de distribución digital de videojuegos fundada por valve corporation, esta es multiplataforma 
y es considerada una importante distribuidora por lo que su llegada a las aplicaciones de ubuntu causó revuelos.


\sectiontext{white}{black}{Hardware}
Por otra parte, como se menciono anteriormente, uno de los aspectos que dificultaron el desarrollo y ejecucion de videojuegos en Linux fue su falta de compatibilidad de hardware y la falta de drivers nativos de codigo libre, que aunque en cierta medida se pueden suplir con drivers privativos implican una imcompatibilidad de sowtware entre el fabricante y el usuario, presentando un problema de calidad.

\sectiontext{white}{black}{Emuladores}
Sin embargo uno de los mayores avances del sowtware libre en el campo de los videojuegos es el desarrollo de emuladores de consolas, programas que se valen de la ingenieria inversa para simular la arquitectura y funciones de una consola de videojuegos, que se han ido desarrollando con el paso de los ańos y no solo permiten el uso de videojuegos de consolas exintas y actuales, sino que tambien permiten la modificacion de estos y el desarrollo de videojuegos independientes para plataformas comercialmente obsoletas. 

\sectiontext{white}{black}{SCUMMVM}
Uno de los mayores exponetes de este tipo de programas es el SCUMMVM, un programa de codigo abierto que emula el motor SCUMM, usado para las aventuras graficas de LucasArts y otras empresas, puesto que no eran compatibles con el sofware moderno, este emulador incluso es usado por empresas como Atari, que adaptan aventuras graficas educativas a tiendas de consolas como Wiiware o Xbox live Arcade, aunque esto acarreo consecuensias legales.

\sectiontext{white}{black}{Comunidad homebrew}
Un aspecto a destacar es el de la comunidad homebrew, que modifica una videoconsola para que adquiera caracteristicas y utilidades que no tenia en el momentode su fabricacion, ya sea reproduccion de musica, reproduccion de videos, uso de copias de seguridad o instalacion de programas para emular otras consolas de videojuegos, aunque siempre ha sido perseguida en cierta medida por las grandes compańias fabricantes de videojuegos, nunca se ha comprobado que sea ilegal.

\sectiontext{white}{black}{Prototipos}
Pero todo esto ha llevado a la comunidad de software libre a crear sus propias videoconsolas, modificables y de codigo abierto, comenzando por pequeńas consolas portatiles como la Dingoo o la GPX32, cuya caracteristica mas destacable  se encuentra en los emuladores ya mencionados combinados con funciones propias de un mp4 e incluso navegacion en red, hasta proyectos ambiciosos como la EVO Smart, que pretende ser un centro de entretenimiento parecido a la PS3, pero tambien una plataforma para que los desarrolladores vean sus juegos en consolas e incluso suban la calidad de sus trabajos, pero sin duda el proyecto que mas expectacion genera es OUYA, una consola de codigo libre basada en android, que es la mas cercana a verse materializada gracias al apoyo que ha recibido.
Estas iniciativas de consolas de sobremesa han surgido a raiz de la desaparicion de las open source portatiles, que han perdido bastante su utilidad frente a los telefonos moviles android, que tienen practicamente las mismas prestaciones, empresas como Nvidia y Valve tambien planean lanzar consolas basadas en linux, pero sin tener una filosofia clara de como sera su uso.

\sectiontext{white}{black}{Una pequeńa conclusión}
A grandes rasgos, el proceso para que los usuarios de linux puedan tener entretenimiento en el area
de los videojuegos ha sido algo tortuoso, pero esto parece empezar a cambiar, solo queda esperar que nuevos
titulos llegan con el nuevo motor id tech5 y  ver el avance de los nuevos proyectos para consolas con otras filosofias.

\bibliographystyle{abbrv}
\begin{bibliografia}
\bibitem{Historia}
\url{http://es.wikipedia.org/wiki/Historia_de_los_videojuegos}.
\bibitem{kopka1}
\url{http://es.wikipedia.org/wiki/The_Linux_Game_Tome}.
\bibitem{kopka2}
\url{http://es.wikipedia.org/wiki/Id_Tech_3}.
\bibitem{kopka3}
\url{http://www.kdeblog.com/tag/linux}.
\end{bibliografia}


\begin{biografia}{images/mi_articulo/autor.eps}{Andres Hamir Cobos Prada} % ańadir fotografía tamańo [2.5 cm x 3.3 cm ]
Estudiante de Ingeniería de Sistemas en la universidad Distrital Francisco José de Caldas, cursa actualmente octavo semestre.  
\end{biografia}

\end{multicols} %termina el entorno multicols
%\eOpage %comienza una pagina nueva

%\rput(7.5,-2.0){\resizebox{10cm}{!}{{\epsfbox{images/mi_articulo/salsilla.eps}}}}

\clearpage
\pagebreak

% Esta obra está bajo una licencia Reconocimiento 2.5 Espańa de Creative
% Commons. Para ver una copia de esta licencia, visite 
% http://creativecommons.org/licenses/by/2.5/es/
% o envie una carta a Creative Commons, 171 Second Street, Suite 300, 
% San Francisco, California 94105, USA.

% Seccion Introducción
%

\rput(2.5,-2.3){\resizebox{!}{5.7cm}{{\epsfbox{images/ensayo/Kokopelli.eps}}}}%Imagen de el comienzo de el articulo, coordenadas desde 
                                                                                   %la parte superior izquierda del margen de la pagina

% -------------------------------------------------
% Cabecera
\begin{flushright}
\msection{introcolor}{black}{0.25}{OPINIÓN} %titulo de la sección

\mtitle{12cm}{Implicaciones de utilizar software propietario para un integrante Egresado del GLUD} %titulo del articulo 

\msubtitle{8cm}{Análisis sobre una cruda realidad} %subtitulo

{\sf Por: Fernando Pineda} %autor

{\psset{linecolor=black,linestyle=dotted}\psline(-12,0)}
\end{flushright}

\vspace{2mm}
% -------------------------------------------------

\begin{multicols}{2}


% Introducción
\intro{introcolor}{E}{n el presente ensayo se pretende desarrollar un análisis de las implicaciones que puede tener el
usar software privativo por parte de uno de los miembros egresados del GLUD (Grupo GNU/Linux de la Universidad Distrital), 
en ese aspecto de desarrollan de forma mas detallada tres aspectos antes de llegar a lo que son la implicaciones reales y 
unas pequeńas conclusiones. 
}

\vspace{2mm}
% Cuerpo del artículo
Primero se establece en la sección ``La Elección'', los parámetros que pueden influir en un usuario, miembro egresado
del GLUD para que vea como una opcion el hecho de usar software privativo en vez de software libre, se explica como esta
persona es libre de tomar la elección y de porque su entono, las personas y las circunstancias lo obligan o lo coaccionan
para que pueda tomar una decisión en el hecho de usar software privativo o Libre.\\

En La segunda sección se presenta una visión desde el punto de vista social, teniendo en cuenta la interacción del egresado 
del GLUD con las demás personas pertenecientes a las comunidad y su comportamiento dentro de las misma. En la sección que 
le sigue se presenta el aspecto ético, el cual se ve condicionado por las reglas de la ética que debe tener en cuenta un 
miembro del GLUD y un egresado del mismo.\\

Finalmente en el documento se presenta el tema principal que son la implicaciones que tienen para un egresado del GLUD el 
uso de software privativo como también las consecuencias de las mismas y el hecho de que al volverse miembro del grupo 
se adquieren unas responsabilidades relacionadas con el cumplimiento de las normas, la filosofía del software libre, 
la convivencia en sociedad, el construir comunidad y seguir promoviendo la filosofía del mismo aun cuando este fuera del GLUD. 



\sectiontext{white}{black}{LA ELECCIÓN} %como se hace una sección

``żQué hace hombre a un hombre? se preguntó un amigo mío alguna vez. żSon sus orígenes? La forma en que llegó a la vida? 
No lo creo. Son las elecciones que hace. No es la forma en que inicia las cosas, sino la forma en que decide poner fin 
a ellas''.\cite{hellboy} Esta frase dicha por un personaje de ficción hace pensar, żqué realmente es una construcción?, 
como mucha personas pienso que la evolución de todo ser (incluyendo al ser humano) depende principalmente del entono que 
lo rodea, teniendo en cuenta ese aspecto podemos mirar realmente que hace que las personas que nos rodean y las 
situaciones que se presentan ante nosotros sean tan influyentes de cierto modo.\\

Ahora tomando el tema del software libre y combinándolo con la frase que inicia el párrafo anterior tendríamos lo siguiente:\\
``żQue hace usuario del software libre a un usuario de software libre?''. żSon sus maestros?, żla forma como entra la mundo del 
software libre?, żlas personas que lo rodean?, żel software que usa?. Como saberlo, es como buscar una aguja en un pajar, pero, 
si lo que realmente lo hace ser un usuario de software son las decisiones que toma, su forma de hacer las cosas, la tendencia 
a hacer respetar la filosofía del software libre y a difundirla. En mi concepción del mundo está, que un partidario del software 
libre es aquella persona que no tiene miedo a caer una y mil veces con tal de aprender algo, pero de sus errores sabe hacer una 
construcción que realmente lo inspira a dar mas de su parte para superarlos y siempre los ve como una enseńanza. \\

Si miramos lo que realmente hace que un usuario del software libre tome o no una decisión entonces tenemos que tener en cuenta 
que hizo que tomara esa decisión, si fueron las circunstancias, si fue el medio, si fue una influencia negativa. Pero la decisión
siempre está en manos de dicho sujeto, de como se comporte, de que tan bueno sea su pensamiento con respecto a la filosofía del 
Software libre y su fuerza para poder afrontar las condiciones adversas que pone las sociedad para quienes entran en este mundo. 
Debido a que las personas que entran al software libre siempre están expuestos a los ataques que contra ellos propician los 
usuarios de software privativo (que particularmente siempre creen tener las razón), mientras que el usuario de software libre 
no busca convencer a nadie sino que expone unos argumentos, unos beneficios y espera a que la persona que lo escucha tome o no 
su propia decisión con respecto al software libre (claro que no falta el fanático del software libre que se cree dios y creé tener
la verdad revelada).


%%%%%%%%%%%%%%%%%%%%%%%%%%%%%%%%%%%%%%%%%%%%%%%%%%%%%%%%%%%%
\ebOpage{introcolor}{0.35}{OPINIÓN}
%%%%%%%%%%%%%%%%%%%%%%%%%%%%%%%%%%%%%%%%%%%%%%%%%%%%%%%%%%%%

\sectiontext{white}{black}{EL ASPECTO SOCIAL} %sección

Este aspecto comprende la interrelación del miembro del GLUD con la comunidad, misma que ha de desarrollarse de manera armónica y 
siempre con la finalidad de mejorar la visión que tienen las personas del grupo y la calidad de la enseńanza y difusión del software 
libre; así como la de ejercer una influencia positiva en el entorno. Si en general no se cumple este propósito el integrante o 
egresado del GLUD no ha aprovechado su estancia en el mismo para poder aprender el concepto de comunidad y las relaciones que 
entre los sujetos se establecen al vivir en tal.\\
Teniendo en cuenta lo dicho anteriormente entonces se dice que para un egresado o integrante del GLUD es de vital importancia 
poner en practica las enseńanzas que adquirió en el mismo para con todas las personas, sean componentes de su entorno y las que 
no conoce, de esta forma hace que el nombre del GLUD quede en alto siempre. También debe recordar no entrar en conflicto a causa 
de su filosofía del software libre, sino que tiene que recordar lo que realmente es el software libre y poner en claro los argumentos 
de los cuales esta se vale para producir un avance en las sociedad.\\
Por otra parte sin dejar de lado lo dicho anteriormente tomemos el aspecto del software privativo y la relación de uso que un egresado o
integrante del GLUD puede establecer con el mismo, de ser así este estaría incurriendo en un hecho que se puede ver desde distintas 
interpretaciones, la cuales se exponen a continuación:\\
\begin{itemize}
 \item \textit{El punto de vista radical:} Si no usa el software que se le dice se pueden generar malas interpretaciones,  rencillas 
 y disgustos con personas que se relacionan con usted. 
 \item \textit{El punto de vista diplomático:} En este aspecto usted usa el software propietario, pero no esta de acuerdo con la 
 filosofía del mismo y eventualmente deja de usarlo. 
 \item \textit{El punto de vista resignado:} Aquí las persona no solo usa el software privativo sino que acepta su filosofía y nunca lo abandona.
\end{itemize}

Estos son los diferentes aspectos sobre los cuales es posible ver el efecto que produce a nivel social el uso y la aceptación del 
software privativo por parte de uno de los egresados del GLUD, todo depende de la forma como la persona que ve las cosas las interprete,
esto se presenta como la relación que tiene el grupo con las personas que pasaron por él y ahora no están, en ese aspecto es necesario
que los egresados tengan en cuenta que son los representantes del GLUD en el mundo y que por este hecho es que deben hacer los posible por 
dar una buena impresión y dejar en alto la cara del GLUD y siempre tener claro que es la filosofía del software libre y no solo saber que es 
sino también seguirla.


%\begin{entradilla}
%{\em {\color{introcolor}{Latex}} facilito las cosas en el desarrollo del proyecto
%}
%\end{entradilla}


\sectiontext{white}{black}{El ASPECTO ÉTICO}

Este es uno de los aspectos más relevantes en el arte del software, en el que la profesión alcanza una alta dignidad. El desarrollador y 
el usuario en su quehacer, deben estar guiado por dos principios el amor a su comunidad y del amor a su arte. Esta idea un poco orientada 
hacia la filosofía de Hipocrates, esta plantea que es un aspecto en el que la ética se basa como un conjunto de las reglas que se dan como 
alternativas para los usuario y desarrolladores, de esta forma se hace evidente que el mismo amor al arte que estos desempeńan es el 
que hace que tomen o no alguna decisiones cruciales para su desempeńo ético y sumado a eso esta la lealtad que deben a su comunidad.\\

La teoría clásica acerca de la ética de la tecnología nos plantea que esta no puede ser considerada ni buena ni mala, sino que el resultado 
de lo que haga el hombre con la tecnología es lo que determina si la tecnología se usa de una forma buena o mala, entonces en este sistema 
clásico la tecnología no depende sino de lo que el hombre haga con ella, mas el hombre no esta condicionado por la misma. Mientras que en 
la teoría moderna se plantea con una idea menos aristotélica y mas platónica, por ejemplo que la tecnología es un medio por el cual se puede 
ejercer control a las personas y paso a citar:\\
``la educación es el principal elemento represivo, el medio más eficaz para el control, el más apropiado homogeneizador social. En la 
educación se hará al ciudadano: se condicionara su sensibilidad, su voluntad y su pensamiento, de modo que nada pueda desear sino aquella 
situación que <<por naturaleza>> le pertenece''\cite{platon}. Si analizamos esta frase y la vemos desde la situación actual, es claro que 
en la educación nos condicionan ``para'' y es en ese sentido donde algunos egresados miembros y demás vinculados con el GLUD fallan, 
por que en algún momento se dan a la tarea de pensar que la mejor forma de competir es sabiendo mucho y entran en ese pensamiento y terminan 
diciendo que no importan los medios por los cuales sea lo importante es ganar, ganar y ganar.\\ 

Esta ultima frase del párrafo anterior a la cual me refiero ahora es entonces la que genera una incomprensión y un pensamiento errado con 
respecto al software libre y allí migran los egresados del GLUD al software privativo porque supuestamente es mas competitivo y tienen 
mejores capacidades, pero eso no es cierto, porque es solo la tecnología que puede ser mas avanzada pero esta nos condiciona a usarla mientras 
que el software libre no condiciona a nadie a usarlo y por lo contrario le plantea las libertades, en conclusión es uso de software privativo 
por parte de un egresado del GLUD es condicionarse a las reglas que el distribuidor, el vendedor o hasta el mismo software en sus condiciones 
de uso y distribución le impongan, por ende los mejor es tener clara la filosofía del software libre y llevarla en la conciencia siempre.

%%%%%%%%%%%%%%%%%%%%%%%%%%%%%%%%%%%%%%%%%%%%%%%%%%%%%%%%%%%%
\ebOpage{introcolor}{0.35}{OPINIÓN}
%%%%%%%%%%%%%%%%%%%%%%%%%%%%%%%%%%%%%%%%%%%%%%%%%%%%%%%%%%%%

\sectiontext{white}{black}{LAS IMPLICACIONES}

El titulo de este documento habla acerca de las implicaciones que tienen para un egresado del GLUD usar software propietario, pero durante todo 
el documento no se han expuesto estas, ha sido de manera premeditada, ya que era pertinente primero dar algunas nociones introductorias y establecer
unos parámetros que permitieran a los lectores poder tener una visión mas acorde al caso y por ese motivo en las secciones anteriores se habla 
de las elecciones a las que se ve expuesto el egresado y miembro del grupo, de como su filosofía puede ser atacada y vulnerada. También se cometa 
acerca de los principales aspectos (porque hay muchos mas) que deben ser tenidos en cuenta en las decisiones tomadas por alguien que pertenece o 
perteneció al GLUD y por ultimo se dan una pequeńas conclusiones.\\

Antes de comenzar con el tema central de esto quisiera expresar que este es mi pensamiento personal acerca del tema y que no corresponde a un 
pensamiento global del GLUD, por tanto no debe culparse de la atrocidades que aquí consigno a las personas pertenecientes a este grupo mas que a mi.\\
Es evidente que dentro de los objetivos del GLUD no esta la enseńanza ni promoción de software privativo, por este motivo si alguno de los integrantes 
hace uso del mismo es responsabilidad enteramente de él mismo y de nadie mas (a menos que sea influenciado por alguien mas). En ese caso es el y 
su ética los que tienen la opción de decidir que es lo mejor para si mismo, con esto quiero decir que la persona que use software privativo 
después de haber estado o estando en el GLUD no lo hace porque en este se le haya enseńado esto y lo hace bajo su responsabilidad.\\

Tomando entonces el párrafo anterior como un referente tenemos que si alguno de los egresados o miembros del GLUD toman la decisión de usar software 
privativo estaría en una clara contradicción con lo aprendido en este grupo y no tendría porque hacerse llamar miembro o egresado del mismo, por 
el solo acto de conciencia, además de eso también esta poniendo en tela de juicio que su paso por el grupo fuese fructífero, mas allá de la sola 
explotación de los recursos y el talento humano que en este grupo encontró. Por que si este es el caso, entonces se presenta que no estaba allí 
por la enseńanza y el aprendizaje de lo que es la comunidad y el software libre, mas que por la adquisición de algunos conocimientos técnicos 
y los recursos de los que pudiera sacar provecho en su momento, entonces si el conocimiento que adquirió en el grupo no le sirve para la vida esto 
implica que su ética esta corrupta y que la codicia humana que habita en él es mucho mas grande que el conocimiento y el deseo de crecimiento 
en forma de comunidad, el compartir conocimiento, el desarrollar ideas para que los otros miembros de la comunidad se beneficien y así la comunidad 
se fortalezca.\\

Teniendo todo lo dicho anteriormente en cuenta de no ser por un caso extremo, si un egresado del GLUD llega a usar software privativo la implicación 
mas grave es que tendría que llevar en su conciencia todo el peso de lo que es traicionar las enseńanzas que el grupo se esforzó por transmitirle y 
defraudar la confianza que un día el grupo puso en él al permitirle que fuera uno mas de los miembros activos.

\sectiontext{white}{black}{CONCLUSIONES}
\begin{itemize}
 \item Los integrantes del GLUD adquieren responsabilidades como miembros activos que deben seguir ejerciendo aun cuando ya no lo sean y estas son 
 llevar siempre el buen nombre del grupo y no defraudar su filosofía.
 \item Muchas veces es importante ser fiel a la filosofía del software libre a pesar de las adversidades hasta cuando estas ya sean infranqueables y 
 siempre aceptando el punto de vista de los demás y aprendiendo de ellos.
 \item El software privativo es malo, es antisocial, debe desaparecer.
\end{itemize}


\bibliographystyle{abbrv}
\begin{bibliografia}
\bibitem{hellboy}
Gillermo.~del Toro, \emph{Hellboy}, 1.\hskip 1em plus
  0.5em minus 0.4em\relax USA: Columbia-Pictures, 2004.
\bibitem{platon}
A.~Platon, \emph{La Republica}, Platon Diálogos, Tomo~1 - Prologo, pg~6. \hskip 1em plus
  0.5em minus 0.4em\relax , Bogotá: Ediciones Universales, 1995.
\end{bibliografia}


\begin{biografia}{images/ensayo/autor.eps}{Wilmar Fernando Pineda Rojas} % ańadir fotografía tamańo [2.5 cm x 3.3 cm ]
Estudiante de Ingeniería Catastral y Geodesia en la universidad Distrital Francisco José de Caldas, cursa actualmente noveno semestre factorial, 
es activista del movimiento del software libre y aun cree que ``este cuento hay que lucharlo por la gente que lo que pasa es que estamos en malas manos, 
cree en la democracia, en la libertad y que el software libre es la solución a todos los problemas de la humanidad, eso es un norte demasiado largo''. 
En la actualidad se desempeńa como Miembro activo del GLUD (Grupo GNU/Linux de la Universidad Distrital Francisco José de Caldas), 
es participe de el proyecto SIGLA, la edición de la revista GLUD Magazine y colaborador de RadioGLUD.  
\end{biografia}


\end{multicols} %termina el entorno multicols
%\eOpage %comienza una pagina nueva

%\rput(7.5,-2.0){\resizebox{10cm}{!}{{\epsfbox{images/mi_articulo/salsilla.eps}}}}

%\clearpage
%\pagebreak

% Esta obra está bajo una licencia Reconocimiento 2.5 Espańa de Creative
% Commons. Para ver una copia de esta licencia, visite 
% http://creativecommons.org/licenses/by/2.5/es/
% o envie una carta a Creative Commons, 171 Second Street, Suite 300, 
% San Francisco, California 94105, USA.

% Seccion Introducción
%

\rput(2.5,-2.3){\resizebox{!}{5.7cm}{{\epsfbox{images/OTySL/Kokopelli.eps}}}}%Imagen de el comienzo de el articulo, coordenadas desde 
                                                                                   %la parte superior izquierda del margen de la pagina

% -------------------------------------------------
% Cabecera
\begin{flushright}
\msection{introcolor}{black}{0.25}{OPINIÓN} %titulo de la sección

\mtitle{12cm}{LA INFORMACIÓN GEOGRÁFICA, EL SOFTWARE LIBRE Y EL ORDENAMIENTO TERRITORIAL} %titulo del articulo 

\msubtitle{8cm}{El software libre en el ordenamiento del territorio} %subtitulo

{\sf Por: Edda Camila Rodríguez Mojica} %autor

{\psset{linecolor=black,linestyle=dotted}\psline(-12,0)}
\end{flushright}

\vspace{2mm}
% -------------------------------------------------

\begin{multicols}{2}


% Introducción
\intro{introcolor}{E}{
l proceso de progreso en los ámbitos sociales y culturales se refleja en las intervenciones sobre
el territorio. En éste sentido se han generado dos intrumentos desde la parte social, el plan de desarrollo y desde la parte territorial y espacial,
 el plan de ordenamiento territorial.  Sin embargo la visión de la población es corta cuando desconocen el territorio  y el software libre, en éste sentido permite 
construir escenarios futuros, partiendo del conocimiento de la comunidad que se puede implementar desde la generación y aprobación de proyectos de inversión.
En resumen ésa es la idea que conforma el presente artículo, que debe ser enriquecido poco a poco con la experiencia que se obtenga.
}

\vspace{2mm}
% Cuerpo del artículo
\begin{entradilla} %codigo para una entradilla
{\em {\color{introcolor}{CONSTRUIR, SIEMPRE}} \begin{center}{``Que la gente pueda opinar no es suficiente, que pueda actuar es necesario, y que pueda actuar en aquello que le interesa, en su comunidad, en su barrio, en su municipio. Pero para poder actuar tiene que tener bases, instrumentos culturales y materiales''.
Estanislao Zuleta.}
\end{center}
}
\end{entradilla}

La gestión del territorio se logra a partir de las diferentes decisiones que pueden tomar los actores del mismo. Sin embargo desde que no se establezcan mecanismos para confrontar el conocimiento constante del espacio, las problemáticas sociales, ambientales y económicas, entonces la organización del territorio no alcanzará la expresión que se ha buscado; en consecuencia los escenarios deseados sólo se constituirán en un ideal ambiguo, que sirve únicamente para la propuesta de gobierno y no para  la construcción social y territorial. 
Los hechos y modos de actuar de las entidades estatales son claras con el manejo de la información geográfica, porque en realidad no es conveniente que se sepa del territorio sino lo mínimo. Por eso los pocos que se han puesto en la tarea de estudiarlo en Colombia con la minucia necesaria han encontrado problemas al describir lo que ellos ven.  Aun así hay información que debe estar disponible a escalas adecuadas, de manera que los diferentes interesados puedan acceder sin ningún problema, es el caso de los ríos, las zonas ecológicamente necesarias y no aptas de reconstrucción, parques naturales, zonas urbanas, veredas, zonas de riesgo, vías, zonas mineras, etc., ya sea con fines empresariales o por el proceso de generación de identidad territorial y de cumplimiento de los POT, que implica el reconocimiento de todos los municipios y de todas las problemáticas presentes.

La producción de la información geográfica requiere conocimientos de cómo adquirirla y plasmarla en un mapa, para que éste sea entendido por aquellos que tomarán las decisiones. Respetando el proceso de producción cartográfico żPor qué no disminuir la inversión en las licencias arcGIS, Erdas, PCI geomatics, y enfocar la inversión en la investigación y desarrollo en el software libre GIS? żPorqué no manejar la estructura de las bases de datos bajo software libre? Por otro lado żla información sería libre en algunos casos y bajo qué casos?.  El ordenamiento del territorio requiere de esa información, porque se constituye en la realidad del uso del suelo que permitirá definir los lineamientos a los cuáles se dirige la política de gobierno, sea a nivel nacional, departamental o municipal. Dentro del proceso de organización física del espacio el primer paso es la construcción de escenarios, y éstos deberían ser participativos, a lo cual sigue el estudio de proyectos serán que definirán la intervención territorial. 
 



\sectiontext{white}{black}{Construcción de escenarios  participativos desde el software Libre} %como se hace una sección


Los escenarios permiten anticiparse al futuro, identificar  dificultades, opciones y caminos posibles para llegar a fines determinados. En éste caso el análisis del futuro territorial es  una alternativa de generación de ventajas comparativas,  basándose en la  prospectiva más no en la proyección, argumentando que la primera tendrá en cuenta el pasado, pero no dependerá de éste para explicar los escenarios futuros. Por lo tanto el escenario es la herramienta que permitirá generar estrategias de desarrollo y ejecución, dentro del proceso  participación política y de intervención que se ha ido restableciendo nuevamente.  El reconocimiento del territorio es necesario para saber cómo se ve el país desde el punto de vista físico, con respecto a los habitantes. 
%%%%%%%%%%%%%%%%%%%%%%%%%%%%%%%%%%%%%%%%%%%%%%%%%%%%%%%%%%%%
\ebOpage{introcolor}{0.35}{OPINIÓN}
%%%%%%%%%%%%%%%%%%%%%%%%%%%%%%%%%%%%%%%%%%%%%%%%%%%%%%%%%%%%
Pero el país es descentralizado, por lo tanto las responsabilidades recaen en últimas en el mínimo espacio de desarrollo : los municipios.  
Entonces a nivel municipal generar la información territorial desde el software libre para que otros puedan distribuirla, modificarla, mejorarla y actualizarla es el proceso que debe articularse para obtener así mayor conciencia de lo que el territorio significa. La significación del territorio debe comprenderse desde el espacio dinámico y  la criticidad en algunos aspectos como el económico. En este sentido la parte educativa debe efectuarse de manera que se puedan construir mapas fácilmente e identificar zonas  de interés. Esta parte es fundamental. La construcción de escenarios y de argumentos culturales e incluso políticos es difícil de realizar. Esta construcción debe estar guiada y debe ser amplia en puntos de vista, no cerrarse a las opciones de diversidad de pensamiento y de soluciones, en general con éste mecanismo se podrán identificar puntos comunes que permitan diagnosticar los problemas territoriales.  
Es claro que si bien los escenarios pueden implicar múltiples características, éstos deben separarse según la desagregación del territorio en subsitemas para poder construir bases fuertes con los cuales ejecutar las decisiones: subsistema natural, subsitema social, subsistema económico y subsitema urbano-regional (rural y suburbano). Con respecto a esto la información debe disponerse para cualquier persona.
Lo ideal en la construcción de escenarios como procedimiento es el siguiente:
\begin{enumerate}
\item Reconocimiento del territorio y sus problemáticas por subsistemas: En éste proceso se trata de incluir un espacio de participación política desde las empresas y desde la vida cotidiana de las personas, es decir la generación de información geográfica puede estar a cargo de éstas personas a manera de taller. Se plantea nivel rural y urbano. Para esto puede crearse a partir del software libre mapas online para que la gente los recree, y para que en algunos casos la información esté protegida, pero disponible. 
\item Caracterización histórica de los procesos. Dentro de los subsistemas las tendencias deben ser visibles y sus análisis permitirían observan las posibles consecuencias de las mismas, así como lo que originaron en el pasado.
\item Identificación de fines comunes y de posibilidades aptas: después de analizar la información adquirida y los puntos de vista de la población, que es suficiente para saber qué escenario es el adecuado, se busca responder a la pregunta żqué se puede realizar y que no? żqué posibilidades hay y cuales son imposibles de conseguir? 
\end{enumerate}
Con respecto al proceso anterior, éste debe ser realizado por los alcaldes para saber qué se debe plantear como prioritario dentro de cualquier período gestión. Desde el enfoque de participación dado es necesario que se obtengan los recursos adecuados.
Las ventajas del procedimiento anterior y de la conceptualización obligatoria que se requiere dentro de lo planteado :
\begin{itemize}
 \item Desde la moral y ética, la población adquiere formas de pensar diferentes, acercándose al territorio como el actor de cambio, introduciendo la identidad del territorio a su vida cotidiana y aprendiendo poco a poco a respetar al mismo.
 \item Profundización y expansión en la filosofía de software libre: En general si el software libre es la base para la generación de información geográfica y todo el mundo  puede usarla, modificarla y redistribuirla, se observa que poco a poco la nueva información y los procesos investigativos de progreso pueden encaminarse.
 \item Por parte de los gobernantes tienen asegurado el conocimiento de lo que la gente quiere. Sin embargo ésta participación en la construcción del escenario no es suficiente para que las personas se inmiscuyan en los diferentes procesos políticos que se pueden desempeńar dentro del municipio, sino que deben estar complementados con talleres constantes en problemáticas actuales de su territorio. Por ejemplo en procesos medioambientales y mineros, es el caso de Tasco, ża cuanta gente le preguntaron si preferían el páramo o la mina?
\end{itemize}
Sin embargo lo anterior es idealismo y no realidad. La realidad lamentablemente depende de aquellos que poseen los recursos para la inversión en el territorio, y hasta que éstos intervengan de manera comunitaria, realmente se verán los resultados. 


\sectiontext{white}{black}{De los escenarios al ordenamiento territorial} %sección

 Dentro de las decisiones que toma el alcalde en el municipio, los proyectos son el principal mecanismo de ejecución de actividades. Los proyectos tienen que ser analizados y la corrupción debe evitarse en la mayor medida posible; por otro lado deben ser coherentes con el plan de desarrollo que es la parte social del plan de gobierno y con el plan de ordenamiento territorial , donde se plasmarán las zonas ambientalmente protegidas y que no son susceptibles a intervención, las zonas que por su importancia cultural, histórica y patrimonial  tampoco tienen posibilidad de manejo, siempre y cuando no sea la conservación, zonas de  servicios indispensables (salud, educación y vivienda), zonas de alto riesgo(vulnerabilidad y amenaza), zonas aptas para la ocupación, etc., que deben ser concordantes con los aspectos planteados para el desarrollo según el gobierno. 

%%%%%%%%%%%%%%%%%%%%%%%%%%%%%%%%%%%%%%%%%%%%%%%%%%%%%%%%%%%%
\ebOpage{introcolor}{0.35}{OPINIÓN}
%%%%%%%%%%%%%%%%%%%%%%%%%%%%%%%%%%%%%%%%%%%%%%%%%%%%%%%%%%%%

En   Colombia los proyectos hacen parte de un plan, y éste se encuentra sujeto a un programa, sin embargo las actividades dependen de los resultados  parciales del proyecto, por lo que el proyecto es el generador de las intervenciones del territorio, estos últimos  surgen  de las siguientes entidades: de la administración de la alcaldía, de la comunidad, de las organizaciones gremiales, de las ONG.
Cualquier proyecto, afecta a alguno de los subsistemas anteriormente mencionados, por lo que se requerirá la información geográfica para los correspondientes estudios, y ésta información no siempre está disponible porque el único ente que la tiene es el IGAC. En éste sentido la información de conseguirse libre y se debe manejar con software libre, porque se debe hacer con software legal, y esto disminuiría los gastos para las empresas como para los generadores de la IG.

%\begin{entradilla}
%{\em {\color{introcolor}{Latex}} facilito las cosas en el desarrollo del proyecto
%}
%\end{entradilla}


\sectiontext{white}{black}{Conclusiones}
\begin{itemize}
 \item Los procesos de generación de los POT y los PD requieren siempre una construcción de escenarios participativos, con fines de generación de identidad territorial, en los cuales la información geográfica se recrea, se actualiza y se presenta como información general de uso libre. Este tipo de recopilación, alimenta las bases de datos y además integra las personas al territorio, y permiten entrever la significación e importancia que éste adquiere para cualquier solución e intervención territorial.
 \item El proceso de identificación de fortalezas del software libre en el manejo de la información es amplia, sin embargo sino se intervienen con proyectos dentro de las administraciones, los procesos no se llevarán a cabo. En éste sentido se respeta lo que ha intentado hacer casa del bosque, pero no se apoya la exclusión de pequeńas comunidades de software libre, cualquiera que sea, y mucho menos se apoya la generación de problemáticas que puedan desempeńarse a través de los diferentes procesos que con el SL nazcan.
 \item Dentro de las expectativas que se deben plantear, la idea es implementar el software libre por pasos,  y uno de esos pasos es que se intervengan en las administraciones con proyectos de éste tipo. En este sentido si se genera la posibilidad por medio de una rosca, pues es una buena oportunidad, y no hay que dejarla pasar.
 \item Por otro lado la comunidad es una de la originadoras de las ideas de proyectos como seńalé anteriormente y se pueden conquistar espacios en los cuales se pueda intervenir a través de las estrategias computacionales, tratando de no perder la idea de poder tener un mundo donde el software y  en general el conocimiento intelectual no sea para unos pocos, sino para la generalidad de la población.
\end{itemize}



\bibliographystyle{abbrv}
\begin{bibliografia}
\bibitem{Secretaría de desarrollo social,Secretaría de medio ambiente y recursos naturales,instituto nacional de ecología, Universidad autónoma de México} Indicadores para la
caracterización y ordenamiento del territorio.
\emph{}, 1.\hskip 1em plus
 0.5em minus 0.4em

\bibitem{Anónimo}
\emph{Documento de contextualización polítca}
\bibitem{JUAN UGUARTE CORTES}
 \emph{Teoría general del Municipio: Fines, fundamento y estructura, 2003}
\hskip 1em plus
0.5em minus 0.4em 
\bibitem{ERNESTO FIRMENICH BIANCHI}
\emph{Metodología para la construcción de escenarios,2008} \hskip 1em plus
0.5em minus 0.4em
\bibitem{JULLY ELENA BOLAŃOS LOPEZ}
\emph{Manual de procedimientos: Banco de programas y proyectos de Inversión pública del municipio la Victoria, Boyacá,2010.}
\hskip 1em plus
0.5em minus 0.4em
\bibitem{Revisado por Gabriel Suarez}
\emph{Abc del POT Bogotá, 2009}
\hskip 1em plus
0.5em minus 0.4em

\end{bibliografia}


\begin{biografia}{images/OTySL/autor.eps}{Edda Camila Rodriguez Mojica} % ańadir fotografía tamańo [2.5 cm x 3.3 cm ]
Estudiante de Ingeniería Catastral y Geodesia en la universidad Distrital Francisco José de Caldas, que se encuentra desempeńando actividades de trabajo de grado
y está interesada en el software libre como potencialidad que se debe desarrollar y desempeńar en cualquier carrera para que se obtenga una construcción social y progresista.
Hace parte del GLUD, del cual se ha enamorado, y piensa que es una opción bonita de contribuir a mejorar el país.  
\end{biografia}


\end{multicols} %termina el entorno multicols
%\eOpage %comienza una pagina nueva

%\rput(7.5,-2.0){\resizebox{10cm}{!}{{\epsfbox{images/mi_articulo/salsilla.eps}}}}

%\clearpage
%\pagebreak

% Esta obra está bajo una licencia Reconocimiento 2.5 Espańa de Creative
% Commons. Para ver una copia de esta licencia, visite 
% http://creativecommons.org/licenses/by/2.5/es/
% o envie una carta a Creative Commons, 171 Second Street, Suite 300, 
% San Francisco, California 94105, USA.

% Seccion Introducción
%

\rput(2.5,-2.3){\resizebox{!}{5.7cm}{{\epsfbox{images/mi_articulo/Kokopelli.eps}}}}%Imagen de el comienzo de el articulo, coordenadas desde 
                                                                                   %la parte superior izquierda del margen de la pagina

% -------------------------------------------------
% Cabecera
\begin{flushright}
\msection{introcolor}{black}{0.25}{ENSAYO} %titulo de la sección

\mtitle{10cm}{Libertad, realidad o ilusión} %titulo del articulo 

\msubtitle{8cm}{Libertad de la información en Colombia} %subtitulo

{\sf por List Cabanzo Ivan Alejandro} %autor

{\psset{linecolor=black,linestyle=dotted}\psline(-12,0)}
\end{flushright}

\vspace{2mm}
% -------------------------------------------------

\begin{multicols}{2}


% Introducción
\intro{introcolor}{D}{esde pequeńos, queremos saber todo acerca del mundo 
que nos rodea, żCómo funciona esto?, żQué es aquello?, żPor qué sucede esto?...  
y con el tiempo vamos accediendo al limitado conocimiento que nos brindan, 
desde el colegio hasta el trabajo, siempre se aprende algo nuevo, muchas
personas se quedan con la idea  que obtuvieron  por sus superiores, 
los medios de comunicación, algún amigo o familiar, no es de muchos introducirse
en un tema y saciarse con todo lo que ha aprendido, causando que la censura
sea un acto indiferente, de hecho, desconocido.
}

\vspace{2mm}

% Cuerpo del artículo

Nuestro país, abundante de sabiduría y tradiciones generadas por antepasados, se ha dejado
llevar por falsas promesas, cegados por falsas informaciones que dan en los medios y
la indiferencia de quienes saben la verdad, se podría  declarar una Colombia parcialmente censurada.
Cuando un colombiano promedio observa alguna oposición en contra de alguna Ley mal hecha,
lo primero que piensa es en él mismo, - Ąpara que hacer ello si nada logramos!, 
- Ąlo único que ellos hacen es perturbar la tranquilidad!, - Ąotro trancón más! , 
basándose en lo poco informado que esta dicha persona, el statu quo (todo esta bien, no pongas atención) 
impide saber que esta pasando en realidad.\\

Afortunadamente, no todas las personas se limitan a este tipo de información, una verdad a medias no 
aporta nada, el estudio y ardúo esfuerzo de varias comunidades, ha permitido continuar con la poca
libertad que hay en internet, y digo poca porque muchas paginas importantes se encuentran bloqueadas por
las ISP, y, aunque los autoritarios en Colombia no son muy competentes tienen claro su objetivo; dentro de
todas esas comunidades, me enfocaré en una en especial, aquella que nació gracias a la colaboración y el
trabajo mutuo, que permite que internet exista, aquella comunidad que no vive entre las sombras como
muchos creen: los hackers, los verdaderos, no los vándalos ni bandidos, son aquellos que son muy buenos
en lo que hacen, ellos hicieron de internet su herramienta más valiosa.\\



\begin{entradilla} %codigo para una entradilla
{\em {\color{introcolor}{Los Auténticos Programadores}} llamados así en su inicio, provenían
habitualmente de disciplinas como la ingeniería o la física y con frecuencia se trataba de radioaficionados.\cite{kopka1}}
\end{entradilla}

\sectiontext{white}{black}{Las malas leyes} %como se hace una sección

La información tiene valor, todo tiene un precio ? si bien puede ser cierto o no, quienes controlan el tipo 
de información que se publica en nuestro país tienen claro este concepto, existen ciertos datos que pueden 
llegar a ser ?perjudiciales? pero no precisamente para nosotros, por ello se ha intervenido en la radio, la
televisión y la prensa; el Internet en cambio se ha mantenido relativamente  de pie, y  digo relativamente por
que a pesar de tener acceso a datos importantes, que dan paso a la libertad de expresión, poco a poco se van
privatizando con leyes como SOPA, PIPA, ACTA o la misma Ley Lleras, financiadas por corporaciones 
para satisfacer sus ?proyectos? para controlar Internet.\\

El Internet no puede tener un dueńo, desde sus inicios como ArpaNet se pensó como una herramienta para compartir 
el saber,(independientemente de su uso) pese a su noble creación, entes (que a partir de ahora los llamaré como autoritarios) 
quieren poner precio sobre el contenido que circula por la red de redes, ocultar todo lo que le perjudique,
saber todo absolutamente de todos; En ese sentido, la poca libertad que tenemos se quiere exterminar, 
żpuedo sentirme libre sabiendo que estoy siendo vigilado todo el tiempo o se me esté negando información útil? Muy seguramente 
la respuesta será negativa, a nadie le gusta ser sometido a espionaje (lo privado es mio y solo mio),  y tenemos derecho a 
saber la verdad. sin embargo, no se puede negar que existen ciertos contenidos  que deben publicarse con restricciones, 
(libertad no tes igual a libertinaje). \\

Los autoritarios prosperan en la censura y el secreto. Y desconfían de la cooperación voluntaria y del intercambio de información
?sólo les agrada la cooperación que tienen bajo su control. La libertad es buena, los nińos necesitan guía, y los criminales,
restricciones; por tanto se puede estar de acuerdo en aceptar algún tipo de autoridad.\\


%%%%%%%%%%%%%%%%%%%%%%%%%%%%%%%%%%%%%%%%%%%%%%%%%%%%%%%%%%%%
\ebOpage{introcolor}{0.35}{INTRODUCCIÓN}
%%%%%%%%%%%%%%%%%%%%%%%%%%%%%%%%%%%%%%%%%%%%%%%%%%%%%%%%%%%%

\sectiontext{white}{black}{Las salidas de la censura} %sección


% A continuación un ejemplo de como se puede hacer una entradilla de código en 
% lenguaje C, de parte de los amigos de occam's razor.

%\lstset{language=C,frame=tb,framesep=5pt,basicstyle=\footnotesize}   
%\begin{lstlisting}
%#include <osgDB/ReadFile>
%#include <osgViewer/Viewer>
%
%using namespace osgDB;
%int main(int ac, char **a)
%{
%  osgViewer::Viewer viewer;
%  viewer.setSceneData(readNodeFile(a[1]));
%
%  return viewer.run();
%}
%\end{lstlisting}

Dejar ese pensamiento individualista es un paso bastante grande, se esta atacando su libertad, mi libertad,
la libertad de todos los que te rodean, ser indiferente es apoyar (sin saberlo) al dominio del contenido de Internet;
hacktivistas siempre demuestran su esfuerzo en contra de ello, pero la mayoría no son ?crakers?, ni delincuentes,
son personas que están cocientes  de esta situación, no necesito una mascara para estar en contra de estas acciones,
infórmate bien, no te quedes con lo poco que te dan, despierta ese interés que se ha perdido de pequeńo,
siempre habrá algo más que aprender.







% A continuación un ejemplo de como se puede hacer una entradilla de código en 
% lenguaje Make, de parte de los amigos de occam's razor.


%\lstset{language=Make,frame=tb,framesep=5pt,basicstyle=\footnotesize}   
%\begin{lstlisting}
%MY_CFLAGS=-I${DEV_DIR}/include
%MY_OSG_LIBS=-L${DEV_DIR}/lib -losgViewer
%
%mini: mini-viewer.cpp
%	g++ -o $@ $< ${MY_CFLAGS} ${MY_OSG_LIBS}
%\end{lstlisting}

\sectiontext{white}{black}{LA REALIZACIÓN}

Ejemplo de una tabla\\

\begin{tabular}{|>{\columncolor{encabezado}} c |>{\columncolor{introcolor}} c |>{\columncolor{introcolor}} c |>{\columncolor{introcolor}} c |>{\columncolor{introcolor}} c |}
\hline
\multicolumn{5}{|>{\columncolor{encabezado}}c|}{multicolumna 1-2}\\
\hline
\rowcolor{encabezado}Col 1 & Col 2 & Col 3 & Col 4 & Col 5 \\
\hline
rgb & cmyk & gray & predefinido & definido\\ \hline
\end{tabular}

Para que la libertad florezca, el Internet debe mantenerse libre del control gubernamental y  corporativo;
Pero como hemos visto con la DMCA, las corporaciones que quieren controlar la red tienen que hacerlo mediante
la compra de malas leyes del gobierno (En Colombia la Ley Lleras), por tanto, el frente más importante en la 
batalla sigue siendo la partida de malas leyes y reglamentos.


\bibliographystyle{abbrv}
\begin{bibliografia}
\bibitem{kopka}
H.~Kopka and P.~W. Daly, \emph{A Guide to \LaTeX}, 3rd~ed.\hskip 1em plus
  0.5em minus 0.4em\relax Harlow, England: Addison-Wesley, 1999.
\bibitem{kopka1}
A.~Kopka and P.~W. Daly, \emph{A Guide to \LaTeX}, 3rd~ed.\hskip 1em plus
  0.5em minus 0.4em\relax Harlow, England: Addison-Wesley, 1999.
\bibitem{kopka2}
B.~Kopka and P.~W. Daly, \emph{A Guide to \LaTeX}, 3rd~ed.\hskip 1em plus
  0.5em minus 0.4em\relax Harlow, England: Addison-Wesley, 1999.
\end{bibliografia}


\begin{biografia}{images/mi_articulo/autor.eps}{Iván Alejandro List Cabanzo} % ańadir fotografía tamańo [2.5 cm x 3.3 cm ]
Futuro desarrollador de software libre, Hoy estudiante de la Universidad Distrital, la curiosidad es algo que nunca me falta sobre todo a la hora de hablar de informática, apasionado a la seguridad informática desde pequeńo, seguidor de la filosofía GNU/Linux. El único limite del ser humano es su imaginación
\end{biografia}

\end{multicols} %termina el entorno multicols
%\eOpage %comienza una pagina nueva

%\rput(7.5,-2.0){\resizebox{10cm}{!}{{\epsfbox{images/mi_articulo/salsilla.eps}}}}

\clearpage
\pagebreak

% Esta obra está bajo una licencia Reconocimiento 2.5 Espańa de Creative
% Commons. Para ver una copia de esta licencia, visite 
% http://creativecommons.org/licenses/by/2.5/es/
% o envie una carta a Creative Commons, 171 Second Street, Suite 300, 
% San Francisco, California 94105, USA.

% Seccion Introducción
%

\rput(2.5,-2.3){\resizebox{!}{5.7cm}{{\epsfbox{images/mi_articulo/stallman.eps}}}}%Imagen de el comienzo de el articulo, coordenadas desde 
                                                                                   %la parte superior izquierda del margen de la pagina

% -------------------------------------------------
% Cabecera
\begin{flushright}
\msection{introcolor}{black}{0.25}{HABLANDO CON EL PADRE} %titulo de la sección

\mtitle{10cm}{Copyright Vs Comunidad en la Era de las Redes Informáticas} %titulo del articulo 

\msubtitle{8cm}{Conferencia de Richard Stallman} %subtitulo

{\sf Transcrpción, Foto, Introduccion y subrayas por Jinny Salcedo} %autor

{\psset{linecolor=black,linestyle=dotted}\psline(-12,0)}
\end{flushright}

\vspace{2mm}
% -------------------------------------------------

\begin{multicols}{2}


% Introducción
\intro{introcolor}{E}{sta es la transcripción de la conferencia que dio el Doctor Richard Stallman en la Universidad Nacional de Colombia en el marco de una giro que hizo por Colombia, ya que tuve la oportunidad de participar, y ya que aporta grandes cosas al articulo de opinión que estaba escribiendo decidí hacer este trabajo. También decidí hacerlo por que parece importante difundir las ideas que tiene este gran personaje, a cerca no solo de los software, si no también de otros aspectos de la vida, que casi nunca veces no se encuentras en sus publicaciones.
}

\vspace{2mm}

% Cuerpo del artículo

Confieso que no tenia muchas expectativas al entrar a la conferencia ya que Richard me parecía una persona (por las cosas que había leído de el) que hablaba de libertad solo en la parte técnica, en la parte de código, y estuve así la mayor parte de la conferencia, como se darán cuenta al final le hice una pregunta que ese Richard Stallman que yo tenia en mi imaginario, no abría podido contestar con la contundencia que lo hizo. Pero salí del auditorio con una idea totalmente diferente de el, eso lo pueden comprobar mis compańeros del GLUD que llegaron a decir que siendo yo uno de los mas reacios a las teorías de Stallman me había logrado " convertir ", y creo que es cierto.

Quiero hacer la aclaración que mi trabajo fue solo el de transcribir la conferencia lo mas fielmente que pude, a petición del propio Stallman ya que pidió compartirla sin obras derivadas ya que era una obra de su opinión (aunque la misma transcripción es una obra derivada), y por tal motivo tal vez para los mas juiciosos de la lengua castellana tenga muchos errores de todo tipo, de todos modos también es de poner en juego que siendo estadounidense tiene un manejo envidiable de nuestro idioma, sin mas preámbulo aquí va.\\

\sectiontext{white}{black}{LA CONFERENCIA} %como se hace una sección

Buenas Noches, si sacas fotos de mi por favor no las pongas en Fecebook, facebook es un motor de vigilancia, y si pones las foto de alguien en facebook, facebook se aprovecha de la foto para vigilarlo mas, pide que alguien ponga el nombre de la persona, y así la base de datos de facebook tiene mas datos sobre el, podemos disputar si poner la foto de tu amigo es buen trato amigable de tu amigo, pero solo pido que no pongas fotos de mi en facebook, otra petición si grabas la conferencia y quieres {\em {\color{introcolor}{distribuir copias por favor únicamente en los formatos favorables al Software Libre}}}, es decir los formatos ogg u otros libres, no en mp-nada por que son formatos patentados, no en flash por que flash suele exigir el uso de programas privativos por parte del usuario para mirar o escuchar y tampoco en vídeo player, Qick time o Windows media player, y por favor con la licencia CC no derivados , por que es una obra de mi opinión.
Esta conferencia responde a una pregunta que aveces me preguntan al final de una conferencia sobre el Software Libre, me preguntas si las ideas de Software Libre se aplican a otras cosas,  para que esta pregunta tenga sentido tengo que explicar las ideas del Software Libre brevemente.
Software Libre libre quiere decir Software que respeta la libertad de la comunidad de los usuarios y la idea es que si un programa no es libre su distribución es injusta y no debería existir, el software privativo que priva de la libertad a sus usuarios es una injusticia y tendremos que escaparnos de no ser víctimas de esta injusticia y la meta ultima es eliminar esta injusticia.
{\em {\color{introcolor}{Con el software solo hay dos posibilidades o los usuarios tienen el control del programa o el programa tiene el control de los usuarios,}}} para tener el control efectivamente del programa los usuarios necesitan unas libertades, por eso es apropiado llamarlo Software Libre, hay cuatro libertades esenciales, la la libertad cero es la de ejecutar el programa como quieras, la libertad uno es al de estudiar el código fuente del programa y cambiarlo para que haga tu informática como quieras; con esas dos libertades cada usuario tiene el control individual del programa, pero el control individual no basta, por que muchos usuarios no saben programar, no son capaces de ejercer la libertad uno pero tampoco par aun programador como yo basta la libertad uno  por que hay tanto Software Libre libre en el mundo, que ningún usuario es capaz de estudiar y comprender   todo el código fuente de los programas que usa, ni de escribir personalmente todos los cambios que desea entonces no podemos hacer lo que realmente queremos con el solo control individual, hace falta también el control colectivo que necesita dos libertades mas, la libertad dos es la de ayudar a los demás de redistribuir copias exactas del programa como lo has recibido y la libertad tres es la de contribuir a la comunidad es la libertad de distribuir copias de tus versiones cambiadas, en ambos casos cuando quieras por que ninguna

%%%%%%%%%%%%%%%%%%%%%%%%%%%%%%%%%%%%%%%%%%%%%%%%%%%%%%%%%%%%
\ebOpage{introcolor}{0.35}{HABLANDO CON EL PADRE}
%%%%%%%%%%%%%%%%%%%%%%%%%%%%%%%%%%%%%%%%%%%%%%%%%%%%%%%%%%%%
de estas cuatro libertades es obligatoria, tienes la libertad de hacer estas cosa si quieres pero no tienes la obligación de hacerlas.


%%%%%%%%%
\begin{entradilla} %codigo para una entradilla
{\em {\color{introcolor}{Libertades esenciales: }}}cero es la de ejecutar, uno la de estudiar el código, dos es la de ayudar a los demás, la de contribuir a la comunidad.
\end{entradilla}

%%%%%%%%%
%%%%%%%%%
Entonces con todas las cuatro libertades cualquier grupo de usuarios que quieran colaborar, pueden colaborar manteniendo su versión para su uso y quizás distribución, entonces el control colectivo es mucho mas amplio que el solo control al nivel de la totalidad de un país, hace falta que {\em {\color{introcolor}{cualquier grupo pueda tener el control de la versión que usa, }}} pero si los usuarios no tienen estas cuatro libertades entonces no tiene el control efectivo adecuado del programa, entonces es el programa el que tiene el control de los usuarios, pero siempre hay alguna entidad que tiene el control del programa y a través del programa somete a sus usuarios, el programa no libre le llamamos  privativo por que priva de su libertad a sus usuarios, es un yugo, {\em {\color{introcolor}{un instrumento de poder, genera este sistema de poder injusto }}} por el cual el dueńo o desarrollador ejerce poder sobre los usuarios y genera un sistema de {\em {\color{introcolor}{colonización digital, }}}como cualquier sistema colonial practica dividir para dominar y mantiene a sus usuarios divididos e impotentes, divididos por que se les prohíbe redistribuir copias e impotentes por que no tienen el código fuente por lo tanto no pueden ni cambiar el programa  ni averiguar lo que realmente les hace y los programas privativos frecuentemente tiene funcionalidades malévolas, de hecho el caso usual en el mundo es que un usuario de software privativo esta usando malware es decir programas privativos con funcionalidades malévolas por que el desarrollador dueńo de un programa privativo reconoce el poder que tiene y siente siempre la tentación de aprovecharse del poder introduciendo funcionalidades malévolas  y esta costumbre se ha vuelto muy común, casi todos los usuario de Software privativo del mundo usan programas privativos con funcionalidades malévolas, hay tres tipos, por ejemplo las funcionalidades de vigilar a la usuario, las funcionalidades de restringir al usuario se llaman los grilletes digitales o gestión digital de restricciones traducción de DRM, y hay también las puertas traseras que reciben comandos desde ?alguien? para hacer cosas al usuario sin pedir su permiso. 

%%%%%%%%%
\begin{entradilla} %codigo para una entradilla
{\em {\color{introcolor}{Funcionalidades malévolas: }}}vigilar al usuario, restringir al usuario,  puertas traseras.
\end{entradilla}

%%%%%%%%%

Un paquete privativo que quizás contiene los tres tipos de malévolo se llama  ( que quizás conozcas este nombre ) Microsoft Windows, y se trata de funcionalidades especificas conocidas, demostradas, no se trata de especular , no hay duda de eso y también el sistema del mac tiene grilletes digitales, el sistema de los Iphone's tiene todo los tres tipos de malévolos, los Iphone llevan los grilletes digitales mas apretados que nunca por que apple a tomado el control hasta de la instalación de aplicaciones, cuando los usuarios dice jailbreak reconocen  que estos productos son cárceles para sus usuarios también flash player, gratuito pero no libre tienen funcionalidades malévolas una de vigilancia y otra de grilletes digitales, el play station 3 de sony tiene grilletes digitales  y cuando alguien descubrió como hacerle jailbrake Sony envío a la policía hacia el,  y el kindle de amazon tiene todos todos los tres tipos de funcionalidad malévola, mas tarde hablo a cerca del kindle, y casi todos los teléfonos móviles, y aunque con muchos el usuario no pueda instalar software una empresa si puede remotamente instalar cambios de software por una puerta trasera y han empleado esta puerta trasera para convertirlos en dispositivos de escucha y sin conexión transmiten información seńales para ubicar al usuario sin pedir su permiso. Entonces los considero el sueńo de Stalin, los rechazo por que es mi deber de Ciudadano  poder mi dedo en el ojo del gran hermano , es el deber de todos, entonces he demostrado que las funcionalidades malevolentes son el caso normal, pero por que, es consecuencia del poder injusto que los desarrolladores tienen sobre los usuarios la cura es que no tengan este poder, con el Software Libre nadie tiene el poder sobre nadie por que todo usuario es libre, si usas un programa libre aunque no sepas programar hay otros usuarios algunos si saben programar y no quieren ser víctimas de funcionalidades malévolas , entonces cuando de vez en cuando leen el código para hacer algún cambio serán capaces de notar cualquier cosa malévola que halla y corregir el problema y publicar una versión sin lo malévolo y seria un gran escandalo y mantenemos la información de quien contribuyo cada pieza de código y así el culpable sera descubierto y muy criticado por la comunidad. Entonces es un defensa contra lo malévolo, es la única defensa conocida, que el Software sea Libre, no es perfecto pero es mucho mejor que ser indefensos como los usuarios del privativo.

\sectiontext{white}{black}{EL MOVIMIENTO Y LA FSF}

Lance el movimiento de Software Libre en 1983 y en 1984 comencé el desarrollo de un sistema operativo destinado a ser totalmente de Software Libre, es decir {\em {\color{introcolor}{sin  siquiera una linea de código privativo, }}}ara respetar completamente la libertad de los usuarios, para que la idea de usas computadores en libertad sea una opción practica hacia falta un sistema operativo libre y fue mi deber comenzarlo por que nadie mas lo haría, en 1992 llego el ultimo componente del sistema un kernel que se llama Linux, Linux no es un sistema operativo,{\em {\color{introcolor}{ cuando alguien dice el sistema operativo Linux se}}}  

%%%%%%%%%%%%%%%%%%%%%%%%%%%%%%%%%%%%%%%%%%%%%%%%%%%%%%%%%%%%
\ebOpage{introcolor}{0.35}{HABLANDO CON EL PADRE}
%%%%%%%%%%%%%%%%%%%%%%%%%%%%%%%%%%%%%%%%%%%%%%%%%%%%%%%%%%%%

{\em {\color{introcolor}{ equivoca, }}} se trata en verdad del sistema operativo GNU con kernel Linux, entonces el nombre apropiado para esta combinación es GNU ( el lo pronuncia ŃU) con Linux, en ingles el sistema se llama ?GNU? (con correcta pronunciación en ingles) se escribe GNU, pero en castellano se puede pronunciar ŃU como si comenzara por eńe, lo que también el nombre del



mismo animal.


Elegí esta palabra no por ser el nombre de este animal si no por ser un acrónimo recursivo  y un juego de palabras, GNU es decir  G N U  quiere decir GNU no es Unix, fue la manera según la costumbre para reconocer las ideas técnicas Unix pero decir que este sistema es otro, pero teniendo también otro significado era un juego de palabras y esta palabra se usa mucho por que en ingles según el diccionario la ?G? es muda y se pronuncia ?NU?, es decir nuevo, cada vez que quieres decir ?New? es decir nuevo puedes deletrearlo ?GNU?, es un juego de palabras quizás no muy bueno, pero no pude resistirlo, pero en ingles  es muy importante no pronunciarlo con ?New? por que si dices ?the new sistem? te equivocas por que nuestro sistema ya no es nuevo. Y por favor no lo llames Linux, es un error muy común y muy dańino a nuestro trabajo, los usuarios de nuestro sistema no saben desde donde viene el sistema entonces no saben por que considerar lo que decimos, por ejemplo los motivos que desarrollaron este sistema. 

%%%%%%%%%
\begin{entradilla} %codigo para una entradilla
Por favor no lo llames {\em {\color{introcolor}{ Linux}}}, es un error muy común y muy dańino a nuestro trabajo.
\end{entradilla}
%%%%%%%%%
Pero bueno dos cosa mas a cerca de software libre, abras oído alguna vez el termino código abierto ese termino fue inventado en su versión inglesa ?open sourse? al ańo 1998, durante los ańos 1990 en la comunidad de software libre había dos campos políticos, había el movimiento Software Libre que decía {\em {\color{introcolor}{lo hacemos por la libertad, }}} el software privativo es una injusticia tenemos que escaparnos, tenemos que trabajar para que todos puedan escaparse; y había el otro campo de los que fomentaba el uso y contribuían al  desarrollo del Software Libre pero sin considerarlo un asunto ético, sin plantear valores éticos como libertad y comunidad y en su lugar decían  este sistema es decir solo tenían , valores por lo menos en este asunto, solo tenían{\em {\color{introcolor}{valores prácticos de comodidad, }}} no plantean un asunto ético, en 1998 ellos inventaron el termino código abierta para nunca mas decir libre, para que nuestras ideas se olvidaran y así han hecho esfuerzos bastante grandes y nuestras ideas casi se olvidaron, pero no por que hemos luchado bastante fuerte para que no se olviden, fue en ese ańo que comencé a viajar casi todo el ańo haciendo  conferencias, por que los medios grandes, sobre todo los medios anglófonos nunca mencionan el Software Libre cuando hablan de mi dicen código abierto o open source, entonces daban una impresión completamente falsa de mi postura, tengo que escribir corrigiendo y  criticando.

Entonces tu postura es de tu elección pero ahora puedes reconocer si alguien dice código abierto que idea esta apoyando y puedes  elegir que idea apoyar , entonces si estas de acuerdo con nuestras ideas éticas por favor muestra tu apoyo al publico diciendo siempre Software Libre, libre, libre, por que es asunto de libertad, cada vez que lo digas y otros lo oigan nos apoya y este apoyo es un apoyo que necesitamos.

\sectiontext{white}{black}{SOFWARE LIBRE EN LAS ESCULAS} %como se hace una sección

Ultimo punto, las escuelas tienen que  enseńar únicamente con Software Libre, todos los niveles de escuelas y todas las actividades educativas, no solo para ahorra dinero es un beneficio secundario, pero hay motivos éticos por ejemplo las escuelas tiene un misión social de educar a buenos ciudadanos en una sociedad capaz fuerte independiente, solidaria y libre, en la informática es decir graduar a usuarios habilitados al uso de Software Libre y listo para vivir en un tal sociedad, pero enseńar el uso de un programa privativo es implantar dependencia a una entidad, no se debe enseńar Software privativo, va en contra de la misión de la escuela, la misión de la educación. Por que ofrecen unos desarrolladores de privativo copias gratuitas de sus programas no libres a las escuelas, es como por que los traficantes de drogas ofrecen ampollas gratuitas, y las ofrecerían a las escuelas también por que desearían aprovecharse de las escuelas para imponer la pendencia a la sociedad ?la primera dosis es gratis ?  dice Microsoft y a veces Apple, entonces la escuela rechazaría las drogas que sean gratuitas o no y tiene que rechazar privativo que sea gratuito o no. Pero también hay el motivo de la educación misma en la informática, algunos son programadores natos y a la edad de 10 a 13 ańos antojan aprender todo de la informática, si usan un programa quieren saber como lo hacen, pero cuando el pregunta al profesor,{\em {\color{introcolor}{ si el programa es privativo, solo puede contestarles es un secreto no podemos saberlo, }}}es decir que la educación no se permite, un programa privativo es conocimiento denegado, es el contrario del espíritu de la educación y no debería ser tolerado en una escuela, pero si el programa el libre el profesor puede explicarles cuanto sepa, luego darles copias de código fuente del programa diciéndoles léelo y comprenderás todo y los leerán por que antoja comprender todo y puedes decirles, si encuentra algún punto que no comprendes solo muéstramelo y podremos comprenderlo juntos como se aprende a escribir bien el código, es decir de manera que otros lo comprenderán, hace falta leer mucho código y escribir mucho código, solo el Software Libre permite leer el código de programas grandes que realmente se usan y luego hace falta escribir mucho código, es decir escribir código en programas grandes  pero para comenzar hace falta escribir pequeńos cambios, cambios sencillos en programas grandes, para ofrecer la posibilidad de

%%%%%%%%%%%%%%%%%%%%%%%%%%%%%%%%%%%%%%%%%%%%%%%%%%%%%%%%%%%%
\ebOpage{introcolor}{0.35}{HABLANDO CON EL PADRE}
%%%%%%%%%%%%%%%%%%%%%%%%%%%%%%%%%%%%%%%%%%%%%%%%%%%%%%%%%%%%

  educación 
profunda en la informática en la programación hace falta usar Software Libre, pero hay también la educación moral que se aplica a todos los estudiantes, una escuela tiene que ir {\em {\color{introcolor}{ mas allá que ensańar hechos y capacidades tiene que ensańar el espíritu de buena voluntad, }}}es decir el habito de ayudar a los demás, entonces cada clase debe tener esta regla, los  estudiantes si traes un programa no puedes guardarlo para ti, tienes que compartir copias con el resto de la clase, incluso su código fuente por si acaso alguien quiera aprender, por que esta clase es un lugar apara compartir los conocimientos, entonces no se permite traer un programa privativo a la clase, para dar el buen ejemplo la escuela tiene que seguir su propia regla, tiene que traer únicamente SL a la clase y distribuir copias a todos en al clase, si tienes una relación con una escuela , como por ejemplo con esta universidad es tu deber militar  por la migración de la escuela al software libre, y hace falta plantear el asunto públicamente para informar a mas gente del asunto por que mayormente no lo conocen y para buscar aliados. Conozco que en esta universidad utilizan un programa que se llama Sia, es un programa privativo , y que la universidad impone su uso, hace falta resistencia por los estudiantes, y lo que hace falta es reemplazarlo por otro tipo de sistema, que no exija que los estudiantes ni los profesores ejecuten código privativo, y es posible, supón que los estudiares de esta universidad podrían escribirlo y seria un buen entrenamiento.

%%%%%%
\sectiontext{white}{black}{LIBERTAD A LAS OBRAS} %como se hace una sección

%%%%%%

Me preguntaban  si estas ideas se aplican a algo que no sea software y cuando me lo preguntan suelen especificar el hardware, es que el hardware debería ser libre, pero que quería decir que un objeto físico sea libre en este mismo sentido, evidentemente se trata del mismo sentido de libertad, entonces tendría que llevar las cuatro libertades, como la libertad 0 que seria la libertad de usar el objeto como quieras, pero normalmente si compra un objeto tienes esta libertad, los objetos físicos no suelen venir con licencias para restringir a su dueńo en el uso del objeto, si hay actos prohibidos por la ley que no puedes hacer ni con con ni sin este objeto pero el objeto mismo no restringe, entonces tienes la libertad 0, y la libertad 1 es la libertad de estudiar y cambiar el código fuente pero los objetos físicos normalmente no tienen código fuente, tendremos que adaptar esta libertad, podría ser la libertad de estudiar y cambiar el objeto, y esta libertad la tienes, si eres dueńo del objeto estas libre de estudiarlo y cambiarlo pero las posibilidades practicas de hacerlo son bastante limitadas, si el objeto es de madera es mas fácil de cambiar, si es de metal es mas difícil, si es de plástico es muy difícil, si es de cerámica es imposible, si es un chip de computadora es imposible, pero al gran dificultad se presenta con las libertades dos y tres, por que es asunto de hacer y distribuir copias y no hay copiadoras para los objetos físicos, entonces la idea de hardware libre hoy en día no tiene mucho sentido, quizás en el futuro, quizás cuando tengamos algo como el transportador de ? Atar tres ? que podríamos cambiar un poco para copiar objetos, seria útil, pero lo que hay son las impresoras 3D que son capaces de fabricar objetos desde diseńos entonces{\em {\color{introcolor}{ estos diseńos  son obras, no son objetos, son obras y son ejemplos de una gran categoría de cosas para las cuales esta pregunta si tiene sentido, }}}es decir las obras, cualquier tipo de obra podrías tener un copia de la obra y podrías copiarla o cambiarla, entonces si el tener o no tener estas libertades es una cuestión con sentido, por que puedes hacer estas actividades con una obra, entonces la madera de la pregunta es que libertades deberíamos tener con las obras sobre todo las obras publicadas, si la obra no es un programa usualmente la única causa de restricciones al usuario es el derecho de autores, con los programas no es así la restricción principal viene de los contratos y a veces intentan imponer contratos también al usuario de otro tipo de obra pero principalmente es el {\em {\color{introcolor}{ derecho de autor }}} , entonces podemos formular la misma pregunta desde el otro lado diciendo que debería decir al ley del derecho de autor, para considerar esta pregunta es útil considerar la historia de al copia, por que el derecho de autor se a desarrollado en razón estrecha con al tecnología de la copia, los cambios tecnológicos no pueden cambiar nuestros principios éticos o no deberían cambiarlos por que los principios son mas profundos que la tecnología, pero para aplicar nuestros principios a un caso especifico lo que hacemos es considerar las opciones y considerar los resultados que tienen y buscar los resultados según nuestro principios, pero un cambio de contexto como de tecnología puede cambiar los resultados probables del mismo acto, entonces puede hacer el acto mas correcto o mas malo de lo que era, por ejemplo si pudiéramos resucitar a los muertos el asesinato no seria tan malo.

%%%%%%
\sectiontext{white}{black}{LA HISTORIA DE LA COPIA} %como se hace una sección
%%%%%%

Entonces abordemos la historia de la copia, la copia comenzó en el mundo antigua cuando se hacia leyendo la copia de un libro y escribiendo otra copia, una tecnología bastante sencilla y muy lenta, muy poco eficiente pero con otras características interesantes y pertinentes, por ejemplo no tenia economía de escala hacer diez copias te abría costado diez veces el tiempo de hacer una, tampoco necesitaba herramientas especiales solo herramientas que necesitabas para leer y escribir y tampoco una capacidad especial solo la capacidad de leer y escribir, por lo tanto cualquier persona letrada podía copiar mas o menos bastante bien como quien quiera y resulto un sistema descentralizado de copiar en cualquier lugar si había alguien que tenia  una copia y deseaba hacer otra lo hacia y nadie le decía que no pudiera por que no había derecho de autor en el mundo antiguo, la idea nunca se planteo pero excepto en un caso, si el príncipe de la región, si a el no le gustaba ese libro podría castigarte muy fuerte por copiarlo, pero ese no fue el derecho de autor, si no algo muy relacionado con la censura y era así durante muchos ańos  pero luego hubo un avance 

%%%%%%%%%%%%%%%%%%%%%%%%%%%%%%%%%%%%%%%%%%%%%%%%%%%%%%%%%%%%
\ebOpage{introcolor}{0.35}{HABLANDO CON EL PADRE}
%%%%%%%%%%%%%%%%%%%%%%%%%%%%%%%%%%%%%%%%%%%%%%%%%%%%%%%%%%%%

en la tecnología de copiado, es decir la imprenta. La imprenta hacia mas eficiente la copia pero de manera no uniforme, en el mundo antigua teníamos este caso (muestra sus dos manos al mismo nivel) la copia masiva y la copia de una sola copia era antiguamente poco eficiente pero con la imprenta teníamos esta situación (muestra sus dos manos, esta vez una mucho mas arriba que la otra), la imprenta hacia mucho mas eficiente la producción masiva sin beneficiar a hacer una sola copia, la manera mas eficiente de hacer una sola copia seguía siendo a mano, entonces la imprenta tenia unas características muy diferentes a las de la copia a mano, tenia una economía de escala por que poner la tipografía tomaba mucho tiempo, pero luego podías sacar muchas copias idénticas mucho mas rápidamente que escribirlas, y la imprenta necesita herramientas muy caras, herramientas especiales para imprimir, la mayoría de las personas letradas no tenia prensa y tampoco sabia usarla por que es otra capacidad diferente de al capacidad de leer y escribir, entonces la imprenta llevo a un sistema centralizado de producción de copias, copias de cualquier libro producían en unos lugares y luego se transportaban a donde alguien quisiera comprar. 
El derecho de autor comenzó en la época de la imprenta, había varios sistemas, un poco como el derecho de autor, pero el derecho de autor actual comenzó en Inglaterra en 1557, como un sistema de censura, explícitamente para la censura, la relación entre el derecho de autor y la censura ha durado hasta hoy, la idea fue que para poder imponer legalmente imprimir un libro el editor tenia que pedir permiso al estado y ese permiso fue otorgado como un monopolio permanente para ese editor y este sistema controlaba hasta mas o menos 1680, cuando permitieron vencerse la censura y todo este sistema, pero los editores reclamaban su monopolio perdido lo que fue legislado en 1700 y poco fue diferente, fue un monopolio de 14 ańos para el autor, no para el editor si no para el autor, y si a los 14 ańos seguir vivo, podía pedir otros 14 ańos con el máximo de 28 ańos, un método para fomentar la escritura. 
Cuando escribieron la constitución de los Estados Unidos había la idea de quizás darles a los autores el derecho a un monopolio, pero fue rechazada, en lugar de esa idea, esa constitución dice que el congreso tiene el poder de fomentar el progreso, con la medida de otorgar a los autores un monopolio por un tiempo limitado; tiene tres puntos interesantes, primero no exige que halla un derecho de autor , solo permite un derecho de autor, {\em {\color{introcolor}{ si hay algún derecho de autor es para promover el progreso, }}}  no es por que un autor sea mas importante que el resto del mundo, no es para el, no se hace según la constitución para el, solo para el publico y tiene que tener un plazo limitado, desde entonces los editores intentan obligarnos estos punto fundamentales.
En la época de la imprenta el derecho de autor funcionaba como un reglamentación industrial, reglamentando a los editores con el poder en las manos de los autores, pero el sistema fue organizado para proporcionar {\em {\color{introcolor}{ beneficios al gran publico }}}Entonces si siguiera sin cambiarse la época de la imprenta, probablemente no desearía criticar el derecho de autor, pero la época de la imprenta esta cediendo paulatinamente a la época de las redes informáticas, otro avance en la tecnología de la copia, que también la hace mas eficiente y también de manera no uniforme, teníamos en al época de la imprenta esta situación (pone sus manso un arriba de la otra con gran diferencia de alturas) la producción masiva bastante eficiente y la producción de una sola copia muy poco eficiente, pero con la tecnología digital tenemos esto (pone sus manos a diferente altura, pero esta vez la diferencia de alturas es poca), tiene beneficios en ambos casos pero el gran beneficio es para hacer una sola copia, por ejemplo en los discos compactos, los discos de producción masiva son mas baratos y mas durables, pero hacer un disco es bastante barato para que cientos de millones de humanos lo hagan, todos en el mundo desarrollado pueden hacerlo, es una situación mas parecida al mundo antiguo, en el mundo antiguo antiguo teníamos esto (pone sus manso al mismo nivel ) en el mundo actual tenemos esto (pone sus manos un poco desniveladas) como menos diferencia entre la eficiencia de la producción masiva y la producción de una sola copia, como en la época de la imprenta. 
Este cambio, cambia el efecto del derecho de autor, si no hubieran cambiado la letra de esa ley su efecto seria totalmente diferente, por que hoy en día quieren aplicar el derecho de autor a los lectores, a todo el mundo es decir, entonces ya no funciona como una reglamentación industrial para los editores en manos de los autores con beneficios al gran publico, si no, como una restricción insoportable al gran publico mayormente en las manos de los editores en nombre de los autores, es decir una ley injusta, no debería existir como es, es decir que la libertad natural trocábamos todos por otro beneficio por que no sabíamos en la época de la imprenta ejercerla, ahora ya sabemos ejercer la libertad, la queremos ejercer.

%%%%%%
\sectiontext{white}{black}{EL ESTADO Y LAS EMPRESAS} %como se hace una sección
%%%%%%

Que haría un gobierno que quisiera representar los intereses del pueblo, disminuiría el poder de los derechos de autor, no necesariamente hasta cero pero menos, pero podemos medir la falta de democracia en la mayoría de los países de mundo por su tendencia de hacer el contrario, de extender el poder del derecho de autor cuando deberían disminuirlo, que hacen, una dimensión del derecho de autor es su plazo, hay una onda mundial de la extensión del plazo, por ejemplo Colombia intentar el plazo del derecho de autor por 20 ańos mas bajo ordenes de los Estados Unidos, {\em {\color{introcolor}{ por que Colombia a perdido su independencia, }}} pero lo hacían en los Estados Unidos en 98 con una ley que llamamos "mickey mouse copyrigth dead ", que hacia extendía el derecho de autor por 20 ańos mas para todas las obras del pasado y del futuro, pero como podrían fomentar mas producción de obras en el pasado extendiendo hoy su derecho de autor, necesitaríamos una maquina del

%%%%%%%%%%%%%%%%%%%%%%%%%%%%%%%%%%%%%%%%%%%%%%%%%%%%%%%%%%%%
\ebOpage{introcolor}{0.35}{HABLANDO CON EL PADRE}
%%%%%%%%%%%%%%%%%%%%%%%%%%%%%%%%%%%%%%%%%%%%%%%%%%%%%%%%%%%%


tiempo y si tienen la maquina del tiempo no la han usado, por que la historia no dice que en los ańos 20 cuando los artistas descubrieron que en 98 su derecho de autor seria extendido por 20 ańos mas se pusieron al trabajo con mucho vigor, por que no usar su maquina del tiempo para darnos obras clásicas amadas.


%%%%%%%%%
\begin{entradilla} %codigo para una entradilla
Podemos medir la{\em {\color{introcolor}{ falta de democracia }}}en la mayoría de los países del mundo por su tendencia a aumentar el poder del derecho de autor.
\end{entradilla}

%%%%%%%%%


Teóricamente es concebible extender el derecho de autor para obras futuras pueda fomentar la escritura en el presente, pero solo para los artistas locos, y la mayoría no son locos, es un mito.{\em {\color{introcolor}{ Según la economía, el valor actual actual descontado de 20 ańos mas de derecho de autor comenzando 50 ańos después de tu muerte es demasiado poco, }}}para cambiar tu decisión racional de hacer o no hacer un obra, entonces la verdadera razón de ser de esa ley es que unas empresas tiene monopolios que deberían vencerse, como por ejemplo el derecho de autor sobre el personaje de Mickey Mouse, por que el derecho de autor sobre la primera película en la cual apareció Mickey Mouse tendría que vencerse, entonces Disney quería comprar esta ley  y otras empresas también, en los estados Unidos las empresas compra leyes,{\em {\color{introcolor}{ en Colombia los Estados Unidos ordena leyes, }}} igualmente injusto como sistemas de gobierno, es decir que ambos países son colonias del imperio de las empresas.

Sospechamos que en 2018  volverá la misma situación, y Disney comprara otra ley para extender por 20 ańos mas sus derechos de autor, sospechamos que todas las empresas tienen el plan de comprar una ley cada 20 ańos para que ninguna nunca caiga al dominio publico mas.

Pero el derecho de autor tiene otra dimensión mas importante, su amplitud, que usos de una obra serán controlados por el derecho de autor, y que usos serán libres; en la época de la imprenta el derecho de autor no debía controlar ni prohibir todos los usos de una obra, los usos controlados eran las excepciones en un campo mas amplio de uso libre, pero con la tecnología digital, los editores tienen la idea de agarrar poder total de imponernos un sistema de pagar cada vista y lo hacen con los grilletes digitales, quieren imponernos el uso de los grilletes digitales, los grilletes digitales siempre vienen en programas privativos por que su propósito es restringir al usuario, que presupone que el usuario no obtenga el control del programa, si pusieran grilletes digitales en un programa libre, cualquier usuario podría cambiar el programa para que no tenga los grilletes, entonces no serviría su propósito injusto, por lo tanto siempre usan software privativo para grilletes digitales. También hay hardware especifico para grilletes digitales por ejemplo las PC's modernas vienen con grilletes digitales cambien en el hardware, transmiten la seńal de vídeo en formato encriptado, específicamente entre el procesador y el monitor para restringir a los usuarios es hardware malévolo, pero mayormente lo hacen el software y cuando hay hardware malévolo hace falta software malévolo  para activar esa funcionalidad, entonces siempre esta en el software. Estos programas atacan nuestra libertad a dos niveles a al vez.

Suele ser un conspiración entre empresa editoriales y empresas de tecnología, una sola empresa no podría lograrlo, tiene que colaborar, que conspirar para restringir nuestro acceso a la tecnología. El primer caso que el gran publico veía fue en los disco DVD por que el formato de los discos puede tener el vídeo encriptado, la conspiración de los DVD's dijo que si quieres tienes que inscribirte en la conspiración y prometer no revelar el secreto del formato, por algún tiempo funciono pero luego descubrieron el formato y publicaron un programa libre capaz de desencriptar el vídeo y fue posible comprar un DVD y mirarlo con Software Libre pero los Estado Unidos saco una ley censurando les distribución de este Software Libre y logro distribuirla a otros lugares como otros países por medio de sus tratados de libre exportación.

A pesar de estas leyes los programas para leer DVD están fáciles de encontrar, entonces desarrollaron otro sistema de grilletes digitales que se llama AACS que se usa en los discos Blue Ray, y hay un programa libre para desencriptar AACS, y a veces encuentra alguna clave que sirve para algunos discos, pero los disco Blue Ray tienen también otro sistema de grilletes digitales y lo cambian cada trimestre, entonces no creo que podremos vencer ese sistema, no tenemos los recursos, con bastante gente colaborando si podríamos hacerlo, pero no tenemos bastante gente, no debemos suponer que siempre ganaremos por nuestra capacidad técnica, el enemigo no es incompetente, siempre es injusto, pero no siempre es incompetente, es un error grave subestimar la capacidad del enemigo.

Hemos visto también grilletes digitales en la distribución de vídeo en al red, hay sitio web como Netlist y Hulu, que distribuyen vídeo por al red con grilletes digitales y no creo que halla software libre que sea capaz de interactuar con esos sitios, entonces hace falta no usarlos. El principio fundamental de valorar tu libertad es nunca aceptar ninguno producto concebido para privarte de tu libertad de no tener disponible las herramientas necesarias para romper los grilletes, entonces si tienes el programa libre para mira los DVD's no tienes por que no mira DVD's , pero sin ese programa no debes mira DVD's  y no debes mira Netlsit, no debes mirar Hulu, hemos visto también los grilletes digitales en la musica con los mp3, hace doce ańos los llamábamos discos corruptos por que la información estaba escrita de tal forma que no lo podían leer las computadoras. 

%%%%%%%%%%%%%%%%%%%%%%%%%%%%%%%%%%%%%%%%%%%%%%%%%%%%%%%%%%%%
\ebOpage{introcolor}{0.35}{HABLANDO CON EL PADRE}
%%%%%%%%%%%%%%%%%%%%%%%%%%%%%%%%%%%%%%%%%%%%%%%%%%%%%%%%%%%%

%%%%%%%%%
\begin{entradilla} %codigo para una entradilla
{\em {\color{introcolor}{ nunca }}}nunca aceptar ninguno producto concebido para privarte de tu libertad.
\end{entradilla}

%%%%%%%%%

Hace una década empezaron a intentar ponerle grilletes digitales a los libros, cuando nos intentaron convencer de que todos usaríamos los libros electrónicos y a un editor se le ocurrió que tendría mucho éxito entre los tecnificarlos si comenzaba con mi biografía, entonces encontró a un autor que me pidió la colaboración y dije que colaboraría si ese libro salia sin grilletes digitales y con libertad de compartir, el editor dijo que no, pero encontramos otro editor, dispuesto a aceptar estas condiciones por que deseaba publicar este libro en papel, y se hizo, pero también publicaron el texto bajo una licencia libre, específicamente la ?GNU free documentation license? que usamos para nuestro manuales, pero de todos modos los libros electrónicos fracasaron, había sido un gran placer afirmar por que había fracasado por que al gente valoraba su libertad, pero evidentemente no, fue por motivos de comodidad de motivos prácticos, decían volverán no hemos vencido, tendremos otra batalla.

Hoy en día se ve esa batalla con productos como el  ?swindle? de Amazon, swindle quiere decir engańo, productos diseńados para cortar las libertades tradicionales de los lectores como la de compra un libro anónimamente, ellos registran todas las compras de libro en una gran base de datos, la existencia de tal lista en cualquier parte perjudica los derechos humanos es intolerable que exista, tampoco nos permiten al libertad de regalar el libro después de leerlo, o de prestar el libro, libertades que Amazon elimina con grilletes digitales y también con contratos para restringir a los usuarios, Amazon no respeta la propiedad privada dice que todas las copias pertenecen a Amazon, que tu no puede ser el dueńo de los libros que compraste, ni siquiera respeta la liberta de guardar el libro cuanto quieras y por fin dejarlo a tus herederos , llegando a borrar los libros remotamente cuando según ellos se vence el contrato.

          
Luego de un reconocido escandalo por borrar libro Amazon prometió no volver a borrar ningún libro y que solo lo haría pro ordenes del estado,yo me imagino que eso podría pasara con algo que hace ańos paso, la autora de Harry Potter consigo una orden e un juez para que los que hubieran comprado el libro no lo leyeran y que lo devolvieran literalmente, por que se había presentado un error en una librería y lo había sacado a al venta antes de estreno mundial y no cumplió con el objetivo de la autora de obtener mucho dinero como ella quiera manteniendo el suspenso para que el libro saliera al mismo tiempo en todos lados, cuando reconocieron  el error Rowling en su ansias de obtener mucho dinero quiso usar la ley  para reservar el error y que el error de su equipo lo pagaran sus lectores, entonces consigo esa orden diciendo que los que habían comprado legalmente el libro no lo leyeran, por eso lance el boicot a Harry Potter, no digo que no leas los libros o que no veas las películas, digo que no los pagues, ordenar que no los leas eso dejo al autor, pero si hubiera podido remotamente borrar eso libros lo habría hecho, pero no pido por que eran copias en papel, pero si fueran libros par el swindle lo abría hecho, es un peligro real a la libertad.

Hemos visto estos grilletes digitales en todos lo medios, hace falta rechazar  los sistemas con grilletes digitales, si no tienes con que romper los grilletes, pero una respuesta individual no basta si el enemigo organiza conspiraciones ,se que existen, por que nos son secretas publican sus planes, tiene sitio web,{\em {\color{introcolor}{ una conspiración para impedir el acceso a la tecnología debería ser un delito grave }}}como una conspiración para fijar los precios, pero no lo es, las empresas reconocen que nuestros gobiernos están de su lado contra nosotros, por lo tanto publican sus conspiraciones. Para esto en algunos países han creado sistemas de leyes  informales para castigar a la gente sin proceso, están dispuestos a acabar con el principio básico de la justicia ?ningún castigo sin proceso justo?, su deseo es castigar a los internautas con desconexión por la mera acusación de haber compartido.{\em {\color{introcolor}{ Quieren imponernos un sistema de opresión para mantener su dinero. }}} Toda esta guerra es injusta por que compartir es bueno.

El estado tiene que disminuir, si ese Estado fuera democrático tendría que disminuir el poder del derecho de autor, pero como,primero hay la dimensión de plazo, sugiero que el derecho de autor dure 10 ańos desde la fecha de publicación de la obra, por que esa fecha de publicación, antes de la publicación no tenemos copias, no nos importa mucho que podamos hacer con unas copias que no tenemos, pero por que 10 anos,por que el ciclo de publicación suele durar tres anos, dentro de tres ańos la mayoría de los libros ya no están disponibles, 10  ańos son mas de tres veces este ciclo, entonces debe de bastar, pero no lo digo como el plazo exacto o correcto, es aproximadamente correcto, pero no todos están de acuerdo, una ves propuse estas ideas en una mesa redonda con autores de ficción y recibí una critica, un autor dijo ?10 anos seria horrible, nada mas que 5?, me sorprendió también por que hasta ese momento lo editores habían logrado engańarme con su publicidad por que cuando piden mas poder sobre nosotros siempre dicen que es por los artistas e invitan algunas artistas muy conocidos que dicen ?si queremos mas poder? y nos dejan suponer que todos los artistas piensan igual pero no es así, solo las estrellas desean mas poder, por que las estrellas realmente ganan por ese poder pero los demás los que realmente necesitan mas dinero no ganan por este sistema, ganan muy poco y mas poder no les llevaría a beneficios este autor recibió un premio por un libro pero no había ganado mucho y su libro aparecía no disponible, sus fans le escribían diciendo como puedo conseguir una copia; Deseaba enviarles copias por correo electrónico pero no podía por que la empresa editorial a pesar del contrato que decía que al no estar disponible el libro los derecho volverían a el,la  empresa no quería admitir el echo de que el libro no estaba disponible y usaba el derecho de autor de su libro para negarle el derecho de distribuir copias de su propio libro para que se apreciara. 

Los artistas comienzan con la idea de mostrar su obra para que se aprecie, y aparte de unas estrellas que reciben bastante dinero para corrompernos,{\em {\color{introcolor}{  la mayoría no recibe nunca tanto dinero y sigue deseando que su obra se aprecie, }}}  el había aprendido por una lección dura que mas de 5 ańos de derechos de autor tenia muy poco probabilidad de beneficiarlo, y podría fácilmente dańarlo, entonces dijo nada mas de 5.  No exijo que sean 10 si todo el mundo prefiere 5 acepto 5. pero quiero proponer algo que sea mas conservador, menos radical.


%%%%%%
\sectiontext{white}{black}{TIPOS DE OBRAS} %como se hace una sección

%%%%%%

Hay también el asunto de la amplitud del derecho de autor, que usos de una obra deberían ser controlados por el derecho de autor,para esto, no tengo una recomendación uniforme, distingo varios tipos de obras según su manera de contribuir a la sociedad, tres categorías, hay las obras del uso practico, es decir el uso para hacer trabajos en la vida, hay las obras que muestran el pensamiento de alguien, y hay las obras de arte y divertimiento, tres maneras de contribuir a la sociedad.

Primero las obras de{\em {\color{introcolor}{ uso practico, }}}estas obras tienen que ser libres hasta antes de su publicación, si alguien te da una obra de uso practico, deberías tener inmediatamente aunque seas el único que haya recibido una copia, mereces la libertad en el uso de la obra, estas obras tienen que ser libres es decir con las 4 libertades esenciales, por que es el mismo argumento, o los usuarios tienen el control sobre la obra o la obra ejerce el poder sobre ellos, entonces para tener el control de la obra necesitas las 4 libertades que incluyen desde la libertad de redistribución comercial, hasta la libertad de vender copias, por que para que lo usuarios tengan control estas libertades hacen falta.

Afortunadamente tenemos una respuesta al argumento que {\em {\color{introcolor}{ si no sedemos la libertad no escribirían estas obras, }}}podemos ver que hay bastante escritura de obras de uso practico que son libres en el mundo actual en el cual no tienen que ser libres, es opcional hacerlas libres pero sin embargo publican muchas obras libres.

Las obras que{\em {\color{introcolor}{ muestran el pensamiento de alguien, }}}or ejemplo las autobiografías, los ensayos de opinión. Publicar una versión cambiada de una tal obra, es presentar mal a otro y no contribuye a la sociedad, no veo por que permitir la distribución de versiones modificadas sin permiso, entonces no hay por que permitir el uso comercial, propongo que estas obras, todas sean libres, pero hay una libertad que todo el mundo necesita, es la libertad de compartir, cuando digo compartir quiere decir redistribuir de manera no comercial copias exactas de la obra es decir, no vender, no alquilar, no modificar. Es una libertad limitada pero esta libertad mínima es necesaria para que el derecho de autor vuelva a ser soportable en la sociedad, por que pondría fin a la guerra contra compartir.  

Hay medidas injustas que nos imponen son para que no compartamos, pero compartir es bueno, no se debe eliminar la practica de compartir, y {\em {\color{introcolor}{ atacar la practica de compartir es atacar la sociedad, }}}y también con la tecnología digital compartir es fácil, y si algo es bueno y fácil la gente lo hace y que podría ser capaz de hacer que la gente cese de compartir, únicamente medidas crueles y draconianas, entonces toman una serie de medidas crueles y draconianas, como el encarcelamiento de quienes comparten, hay que legalizar compartir. Propongo un sistema reducido de derecho de autor, que se aplicaría a todo uso comercial y a toda modificación pero legalizando compartir, incluso por la red.

%%%%%%%%%
\begin{entradilla} %codigo para una entradilla
Si algo es{\em {\color{introcolor}{ bueno y fácil }}}la gente lo hace.
\end{entradilla}

%%%%%%%%%

La tercera categoría, la del {\em {\color{introcolor}{ arte y entretenimiento, }}}n estas obras el asunto del compartir versiones cambiadas fue difícil por que hay argumentos en ambas lados pero por fin reconocí que un artista que quiere publicar una versión cambiada de una obra puede esperar, si no tiene que esperar , si no tiene que esperar a mas de 10 ańos, entonces propongo el sistema disminuido de derecho de autor por 10 ańos después de eso la obra caerá al dominio publico y quien quiera podrá publicar versiones cambiadas, teóricamente todo obra caerá algún día en el dominio publico, algún día se podrán publicar versiones cambiadas.  Pero hay un asunto mas, el remix, el remix quiere decir tomar partes de varias obras y combinarlas quizás con algo nuevo para hacer una obra nueva completamente diferente en su propósito y su punto y este no es igual ha hacer una versión modificada. No puedo poner una linea exacta entre los dos pero en los casos normales son muy diferentes, remix tiene que ser legal, por que {\em {\color{introcolor}{ el propósito del derecho de autor es fomentar la producción de obras }}} e interpretarlo de forma a ser obstáculo para la producción de obras es un absurdo, tan absurdo que solo podría suceder bajo un sistema que a distorsionado completamente el propósito del estado como sistema para apoyar a unas empresas. Tenemos que legalizar la practica de compartir incluso por la red. Las fabricas de musica es decir los editores de discos, dirían ?es horrible robas el dinero a los músicos? pero no es verdad por que esas empresas ya lo han robado todo a los músicos, por que imponen a todo excepto las estrellas establecidas a contratos explotativos que dicen teóricamente si compras una copia, una parte del dinero es para los músicos, pero nunca los reciben por que según el mismo contrato los gastos de producción y publicidad se consideran como avance a los músicos y esta fracción del precio tiene que devolverse, y nunca llegan a devolver completamente este avance, entonces nunca reciben dinero de la venta de sus discos.  Si una banda llega al fin del primer contrato y puede negociar otro contrato no explotativo, por ese contrato realmente recibirá dinero cuando la gente compre los discos bajo ese contrato pero los discos anteriores quedan bajo el primer contrato y nunca reciben dinero de esos discos. Los beneficios que 

%%%%%%%%%%%%%%%%%%%%%%%%%%%%%%%%%%%%%%%%%%%%%%%%%%%%%%%%%%%%
\ebOpage{introcolor}{0.35}{HABLANDO CON EL PADRE}
%%%%%%%%%%%%%%%%%%%%%%%%%%%%%%%%%%%%%%%%%%%%%%%%%%%%%%%%%%%%

reciben los músicos son ser mas conocidos, pueden tener mas conciertos, pueden cobrar mas y así gana mas, pueden vender mercancías en sus conciertos, aunque las fabricas  intenten agarrar eso también, antes de ir aun concierto, o antes de comprar mercancías en un concierto, seria correcto investigar que porción del dinero sera para los músicos mismos y que fracción para una fabrica de musica, Es un beneficio, entonces el resultado del contrato de discos puede beneficiar a los músicos, pero hay otras maneras de hacer conocerse, no hace falta este complejo musico-industrial, entonces legalizar la practica de compartir musica seria completamente bueno, y los músicos excepto las estrellas, no perderán nada y en cuanto a las empresas de discos, no estoy en contra de fabricar y vender discos, es mi manera de adquirir musica comercial, por que puedo pagar anónimamente y en efectivo. Las empresas grandes de discos que han comprado leyes injustas que han demandado a miles de adolescentes por miles de dolares solo merecen perder todo, quiero darles lo que merecen y lo que merecen es desaparecer.

Siempre que tengamos empresas de venta de discos seria bueno tener un derecho de autor que les exigiera realmente pagar a los músicos, por que eso es uso comercial, no propongo eliminar el derecho de autor para el uso comercial de las obras artísticas. También hay las películas, habrás oído sumas astronómicas para hacer una película, pero un productor me explico que son falsas primero por que mas de la mitad no fue para la producción sino la publicidad y en la parte que es menos de la mitad de la producción exageran las sumas con trucos de contabilidad entonces el coste real es mucho menos. 
Y también se puede decir que {\em {\color{introcolor}{ Hollywood produce sistemáticamente basura, }}} estoy completamente en contra de la censura, es injusto censurar una obra aunque sea odiosa por su violencia, he visto películas celebres que por su violencia me dieron asco, pero estoy en contra de censurarlas, por que la censura es peor que cualquier obra, pero no es asunto de censurar o no censurar, no propongo censurar, la cuestión es seguir nuestra libertad no para ayudar a que hollywood siga produciendo basura, y a eso digo no, no quiero ceder esa libertar, es verdad que le sistema actual apoya muy mal a los artistas y un sistema disminuido probablemente no los apoyaría mejor. 

\sectiontext{white}{black}{APOYO A LOS ARTISTAS} %sección

Podríamos apoyar mejor a los artistas, pues con legalizar compartir. Tengo dos métodos para proponer un método es usar dinero publico, puede ser una parte del dinero publico actual o puede ser recogido quizás sobre un impuesto especial quizás sobre conectividad y repartir este dinero entre los artista según el éxito, podemos medir este éxito por la frecuencia de compartir las obras en las redes peer-to-per, con la raíz cubica, es decir un artista A que tenga 1000 mil veces el éxito de otro artista B solo recibirá 10 veces el dinero que recibiría B, así para que el artista B reciba algo considerable no se tendría que hacer asquerosamente rico al artista A y sigue funcionando como mas éxito mas dinero, pero {\em {\color{introcolor}{ se transfiere la mayoría del dinero desde las pocas estrellas a los muchos artistas igualmente capaces de éxito mediano, a los que realmente necesitan mas apoyo, }}}usaría nuestro dinero de manera mas eficiente que el sistema actual. También propongo el sistema de pagos voluntario poniendo un botón en cada reproductor para que sea fácil darlo en los programas,si fuera fácil lo harías, la cantidad puede ser 200 pesos,de pagos pequeńos, para que muchos estén dispuestos a pagar. 


\sectiontext{white}{black}{MÁS INFORMACION} %sección

Quiero decirles donde encontrar mas información sobre el Software libre es gnu.org  y hay también Free Software Foundation cuyo sitio es fsf.org , pude hacerte miembro de FSF que es una manera de apoyarnos en nuestro trabajo, soy voluntario de tiempo completo, la FSF no me paga pero tiene un equipo de empelados, y como obtenemos el dinero para es con  los miembros, también hay Free Software Foundation Latín América  que es fsfla.org, también merece tu apoyo.

\sectiontext{white}{black}{PREGUNTA} %sección

Es de admirar que en ningún momento de su charla (de dos horas y diez minutos), se detuvo ni siquiera para tomar agua  Acabada la conferencia, subasto un peluche de un Ńu y dijo que antes de las fotos y firmar cosas iba a responder preguntas. La mayoría de las preguntas fueron de temas técnicos, que a mi personalmente no me parecía adecuado ya que el todas esa respuestas se pueden encontrar en  sus escritos publicados en la web por tal motivo no me tome el trabajo de transcribirlas (eran otros 40 minutos de charla si quieres obtener las grabaciones en la pagina oficial de GLUD  glud.org van a a estar subidas), ya que una pregunta no se le puede hacer todos los días Richard Stallman aproveche la oportunidad solucionar un interrogante que el generalmente no resuelve en su escritos publicados en la web, hay les va.




%%%%%%%%%%%%%%%%%%%%%%%%%%%%%%%%%%%%%%%%%%%%%%%%%%%
{\em {\color{introcolor}{ Usted menciona dentro de la charla algo que para muchos de nosotros como Colombianos ya es sabido y es que no tenemos independencia . żcual piensa usted que es el papel que juega o jugara el software Libre para lograr la independencia en un país como este? }}}

Lograr la independencia es {\em {\color{introcolor}{ una lucha mucho mas larga que la informática, }}}la migración a Software libre es una oportunidad para difundir ideas libertarias, de hacerse independiente del imperio, pero también es un campo en el cual es posible cobrar algo, esta lucha mas estrecha,{\em {\color{introcolor}{ puede fomentar la lucha mas amplia, }}} pero no puede conseguir solo la independencia de un país, hace falta un movimiento de democracia e independencia, que {\em {\color{introcolor}{ anulemos los tratados de libre exportación, }}}por que cada tratado de libre exportación es antidemocrático, la democracia no quiere decir tener elecciones, es un sistema, un método para la democracia.
{\em {\color{introcolor}{La democracia quiere decir que los mucho pobres se unen para hacer conjuntos mas fuertes que los ricos, }}}las elecciones son para que los mucho puedan decidir las políticas  del estado, pero solo tener elecciones no garantiza la democracia, los ricos tienen la influencia para decidir quien gane, no es democracia, {\em {\color{introcolor}{ el sistema de democracia necesita mantener débiles a los ricos o mantener que nadie sea tan rico, }}}ahora no hace falta que todos tengan la cantidad igual de bienes, seria imposible, seria una dictadura, pero hace falta asegurar que los ricos y las empresas no tengan mucho poder político, que hacen los tratados de libre explotación? Su meta es transferir o disminuir el poder del estado y aumentar el poder de las empresas, básicamente si un tratado facilita la mudanza de empresas de un país a otro, si el estado intenta reglamentar las empresas como se debe, las empresas pueden decir que no pueden reglamentarlas por que son de otro país, si el estado intenta proteger o el medio ambiente o la salud publica o el dinero de educación o el nivel general de la vida, o cualquier cosa mas importante que el negocio, el tratado de libre exportación es una arma para los negocios para oponerse a que el estado cumpla con sus funciones, pero mucho tratados de libre exportación van mas allá, por que ofrecen la posibilidad de demandar contra políticas necesarias del estado, como en los estados unidos han demandado contra una ley que prohibía la venta de cigarrillo con unos sabores ańadidos por que son medidas para hacer adictos a los nińos, mas a los adolescentes, el punto es que es una medida para la salud publica pero la organización mundial de comercio decidió considerarlo como un obstáculo al comercio, y el {\em {\color{introcolor}{ omercio es mas importante que la vida según esa organización, }}}s una organización asesina, que se acabe con esa organización.




\bibliographystyle{abbrv}
\begin{bibliografia}
\bibitem{kopka}
Conferencia Copyright Vs Comunidad en la era de las redes informáticas \emph{Richard Stallman},Bogota Colombia Agosto de 2012
\end{bibliografia} 

\begin{biografia}{images/mi_articulo/jinny.eps}{Jinny Daniel Salcedo Pulgarín} % ańadir fotografía tamańo [2.5 cm x 3.3 cm ]
Estudiante de Ingeniería Electrónica en la universidad Distrital Francisco José de Caldas, cursa actualmente sexto semestre, miembor activo del grupo GNU/Linux. activista del movimiento de tecnologias comunitaras, convencido que la tecnologia es una de las herramientas con las que se puede reducir la gran brecha solcial Colombiana.
\end{biografia}

\end{multicols} %termina el entorno multicols
%\eOpage %comienza una pagina nueva

%%rput(7.5,-2.0)
%{\resizebox{17cm}{!}{{\epsfbox{images/mi_articulo/final.eps}}}}

\clearpage
\pagebreak

% Esta obra está bajo una licencia Reconocimiento 2.5 España de Creative
% Commons. Para ver una copia de esta licencia, visite 
% http://creativecommons.org/licenses/by/2.5/es/
% o envie una carta a Creative Commons, 171 Second Street, Suite 300, 
% San Francisco, California 94105, USA.

% Seccion Introducción
%

\rput(2.5,-2.3){\resizebox{!}{5.7cm}{{\epsfbox{images/mi_articulo/firefoxos.eps}}}}%Imagen de el comienzo de el articulo, coordenadas desde 
                                                                                   %la parte superior izquierda del margen de la pagina

% -------------------------------------------------
% Cabecera
\begin{flushright}
\msection{introcolor}{black}{0.25}{Ensayo} %titulo de la sección

\mtitle{10cm}{Android desde mi pequeño punto de vista} %titulo del articulo 

\msubtitle{8cm}{¿Android en hardware abierto?} %subtitulo

{\sf por JUUC} %autor

{\psset{linecolor=black,linestyle=dotted}\psline(-12,0)}
\end{flushright}

\vspace{2mm}
% -------------------------------------------------

\begin{multicols}{2}


% Introducción
\intro{introcolor}{E}{n un tiempo el hardware era suficiente para todas las tareas en electrónica, se hacian circuitos de tal forma que se requería uno para cada tarea específica y en caso que se necesitara para otra labor, había que diseñar otro para que tuviera los nuevos requerimientos. Con la aparición de circuitos más complejos que permitían configurar su funcionamiento o tareas, de ahí el término programación y consecuentemente el del programa que define que es lo que va a hacer dicho hardware.
}

\vspace{2mm}

% Cuerpo del artículo
Android en la electrónica libre
En un tiempo el hardware era suficiente para todas las tareas en electrónica, se hacían circuitos de tal forma que se requería uno para cada tarea específica y en caso que se necesitara para otra labor, había que diseñar otro para que tuviera los nuevos requerimientos. Con la aparición de circuitos más complejos que permitían configurar su funcionamiento o tareas, de ahí el término programación y consecuentemente el del programa que define qué es lo que va a hacer dicho hardware.
Los teléfonos móviles en contante desarrollo llegaron a ser cada vez más complejos agregando funcionalidades que en vez de mejorar la calidad de la comunicación o la duración en batería lo que hacen es hacer parecer cada vez más estos dispositivos a un mini computador portátil con el que se puede hacer prácticamente de todo.
Recuerdo que conocí el primer celular hace unos 13 años y el también fallecido biper, eran dispositivos un poco incómodos que en menos de unos 5 años se volvieron asequibles y bonitos, sin más funcionalidad que los juegos que se le instalaban; a alguien se le ocurrió hacer los smartphones, un hardware mucho mejor en términos de rendimiento, en que ese software básico para llamar y jugar se hacía muy poco viable por ir en contra del aprovechamiento máximo de ese hardware.
Conozco intentos de sistemas operativos para smartphones como PalmOS, o muchos otro sistema operativo de Nokia y así mismo muchas del software que las compañías generaban para el innovador hardware que sacaban al mercado, pero quedaron cortos cuando el hardware se asemejaba más al de un computador, y no queda de otra que portar un sistema operativo con gran tiempo en desarrollo a los móviles. Ese software para smartphones que ha permanecido y seguirá por un largo tiempo por tener de soporte un sistema operativo para ordenadores personales con un gran tiempo de desarrollo a mi parecer es iOS, Windows Phone y Android o cualquier otro que use el kernel Linux.
El primer SO (sistema operativo) que uso el kernel Linux fue Android y lo llevó a dispositivos móviles, luego fue MeeGo y así aparecen cada cierto tiempo uno que promete ser lo mejor como FirefoxOS (muy bueno por cierto, en lo personal porque he aprendido más de desarrollo web que de cualquier otro tipo de programación) o Ubuntu Phone; pero todos soportados por el gran aporte al software (no solo por ser software libre, sino por la calidad del mismo) que resulta ser el kernel Linux.
Algo que me indigna muchas veces es que no se reconozca que Android es como es gracias al kernel Linux; si Google hubiera empezado de cero o a partir de un SO con licencia bsd o similar probablemente estaría muy lejos de alcanzar a iOS (a Windows Phone en estabilidad lo alcanza casi cualquiera), situación que probablemente no quisieran los de Google ya que siempre quieren que sus productos sean un éxito; le deben mucho de lo que son al software libre y por eso debería dejar de ocultar la realidad del kernel, así sea un poco menos “comercial”.
El mundo del software libre mucha veces pega por su precio y no por alguna otra cuestión técnica, situación de muchos fabricantes chinos, que no les gustaría pagar por una licencia de Windows Phone o recurrir en gastos de ingeniería de software para hacer uno propio y con pensamiento de ingeniero apoyo dicha decisión, ¿Por qué usar crear uno propio si existe Android? un super Sistema Operativo creado por mentes brillantes, estable, popular, apoyado por el monstruo Google y compañías productoras de hardware, y todo esto gratis; sería una estupidez no hacerlo. Pero que ha pasado últimamente, las patentes estúpidas están haciendo que se haga imposible implementar el software libre, libre de cualquier pago.
Ahora vamos con hardware abierto que usa Android; la verdad no conozco si no un proyecto que usa Android oficialmente, este es OUYA, una plataforma para juegos que busca regresarle el espíritu de hackers a las personas consideradas como gamers y que a su vez se han visto impedidos por las restricciones cada vez mayores que representan las consolas de juegos como el conocido Play Station, o el X-Box; meterle la mano a una máquina de estas es prácticamente imposible o incluso instalarle software también se hace muy difícil si no se hace por medios “oficiales” aun cuando la máquina es mía y tengo el derecho de hacerlo.
Hablando del derecho de hacerlo, compré hace un tiempo una tablet china con Android 4.0, consume muchos recursos y no es para nada rápida, a pesar de que tiene un buen hardware a mi consideración (habrá de ser porque el software es muy genérico y no se optimiza sino a parecer para los dispositivos que impulsa Google, los fabricantes de hardware que incorporan dicho software se preocupan muy poco por eso, pero para eso están las Organizaciones como Linaro ). Hay que registrarse para usar el software con su completa capacidad (como se hace también con dispositivos como el Kindle), cosa que no me gusta. Si no tengo una cuenta Google, a pesar de que soy dueño del dispositivo, no puedo usarlo, debería ser más fácil usarlo y no obligarme a crear una cuenta con el casi monopolio informático de Google.
OUYA como ya muchos sabrán fue impulsado en Kickstarter y resultó ser muy popular, demostrando así que el usuario de hoy en día quiere tener las libertades sobre el hardware o el software así no las use a cabalidad, pero que aun así hacen más fáciles las tareas (como de instalar juegos “piratas”); si eso no demuestra la capacidad de ganar dinero con negocios que respeten las libertades que debería tener un usuario y dueño del dispositivo ¿entonces qué?
En expresión de esa búsqueda de libertad hay aplicaciones como F-Droid (que es su logo muy correctamente tiene una C invertida) que es un catálogo con aplicaciones FOSS (Software libre y de código abierto, Free and Open Source Software) demostrando que no se está corto de software libre en esta plataforma y a pesar de que no estoy en contra de Google play, si del software privativo, deberían poner más de las aplicaciones libres que hay (aunque entiendo que tienen que hacer un control).
Otro punto que no me gusta de Android (si, si, al parecer lo que se hace popular cada vez se hace menos atrayente para algunas personas) es que se hace cada vez más excluyente, los dispositivos de baja gamma (entiendasen como dispositivos que no tienen los suficientes recursos de hardware para funcionar de una manera medianamente fluida con las nuevas versiones de Android, no necesariamente son malos, para nada) se han quedado con versiones antiguas de Android y para tener la nueva versión de Android hay que tener muchos recursos de hardware, que no se traduce en otra cosa que pagar más dinero para tener un dispositivo que funcione medianamente fluido (gamma media) o pagar aún más por un dispositivo como el Samsung Galaxy SIII o un mejorcito como el Sony(R) Xperia(TM) Z(Definitivamente el Nexus 4, resultó un dispositivo una ganga, pero imposible de acceder en países distintos a Estados Unidos; Si, se tienen ciertos servicios que hacen que pueda adquirirlo, pero eso hace que se aumente un poco más el precio por la adquisición del dispositivo, como siempre excluyente en todos los sentidos).
La nueva propuesta en la mesa es Firefox OS, un sistema operativo que realmente me ha gustado, porque viene a ser accesible, rápido, fácil y con el nuevo estándar HTML5, este estándar es lo que más me gusta ya que las únicas aplicaciones que soy capaz de hacer bien son web, además de que gracias a eso, hay una cantidad impresionante de posibles programadores para este sistema operativo y mejora muchísimo su rendimiento frente a otras alternativas eliminando capas innecesarias de middleware (software intermedio, como la máquina virtual de java presente en el sistema operativo Android) haciendo que pueda ser usado en dispositivos mucho más baratos y asequibles a la mayoría de la población sin tantos recursos como la media de la población colombiana. Según he leído por ahí, el lanzamiento va a ser en Brasil, en una alianza con telefónica, se sabe que no va a ser fácil hacerlo visible en un mercado que pareciera saturado, pero el hueco donde se pueden hacer y de ahí expandirsen es los dispositivos que mencionaba anteriormente, los de gamma baja. Ya probé el sistema operativo desde el navegador Firefox por medio de una extensión llamada Firefox OS Simulator, y me ha parecido fantástico, muy liviano, nada que ver con el pesado Android y ese lenguaje a mi parecer fastidioso que es Java. He visto unos de los primeros dispositivos con este software en el sitio http://www.geeksphone.com/ y son hermosos, espero que lleguen pronto a Colombia para impresionar a todas las personas con esta adquisición y ver como podría desarrollar aplicaciones para este SO.
Con la anterior crítica solo queda decir que la competencia siempre es buena, no basta con conformarse con el todo poderoso Google, ningún monopolio es bueno, por muy buenas que sean las intenciones de las compañías al fin y al cabo son eso, máquinas para hacer dinero, y el poder sosiega a las personas; dividir la cuota de mercado en muchas empresas es siempre mejor para el consumidor (ya he visto que la competencia de computadores es muy pareja y eso hace que cada vez esa competencia haga mejorar la oferta y su precio), si algún día llego a reemplazar a mi tradicional computador por un smartphone del futuro (que estoy seguro que va a llegar) espero que su sistema operativo sea una opción asequible para todo mundo y no se eleve por los cielos sin pensar en las consecuencias.

\bibliographystyle{abbrv}
\begin{bibliografia}
\bibitem{kopka}
Imágen extraída de, CC-by-nc-sa
http://www.flickr.com/photos/procter/8422068552/sizes/o/in/photostream/
\end{bibliografia}


\begin{biografia}{images/mi_articulo/autor.eps}{Ulises Useche} % añadir fotografía tamaño [2.5 cm x 3.3 cm ]
Estudiante de  la universidad Distrital Francisco José de Caldas, es partícipe del proyecto RadioGLUD y desea empezar a colaborar con la edicion de la revista GLUD Magazine además de crear artículos en el marco del proyecto SIESoL impulsando la revista poniendola dentro de las activades del proyecto Radio GLUD.
\end{biografia}

\end{multicols} %termina el entorno multicols
%\eOpage %comienza una pagina nueva

%\rput(7.5,-2.0){\resizebox{10cm}{!}{{\epsfbox{images/mi_articulo/salsilla.eps}}}}

\clearpage
\pagebreak

% Esta obra está bajo una licencia Reconocimiento Compartir Igual 3 de Creative
% Commons. Para ver una copia de esta licencia, visite 
% http://creativecommons.org/licenses/by-sa/3.0/es/

% Seccion Introducción
%
\definecolor{filacolor}{HTML}{E33D00}
\definecolor{columnacolor}{HTML}{E65B05}
\rput(2.5,-2.3){\resizebox{!}{5.7cm}{{\epsfbox{images/radioglud/entrada.eps}}}}%Imagen de el comienzo de el articulo, coordenadas desde 
                                                                                   %la parte superior izquierda del margen de la pagina

% -------------------------------------------------
% Cabecera
\begin{flushright}
\msection{introcolor}{black}{0.25}{Tecnologías Libres} %titulo de la sección

\mtitle{10cm}{Emisora en línea con Tecnologías Libres} %titulo del articulo 

\msubtitle{8cm}{Todo lo que necesito saber} %subtitulo

{\sf por Jorge Ulises Useche Cuellar} %autor

{\psset{linecolor=black,linestyle=dotted}\psline(-12,0)}
\end{flushright}

\vspace{2mm}
% -------------------------------------------------

\begin{multicols}{2}


% Introducción
\intro{introcolor}{E}{n nuestra sociedad la radio hace aún parte importante de nuestras vidas; tal vez escuchada más por personas que han venido envejeciendo con estos medios y desplazada por el Internet una nueva forma más eficiente de obtener información de todo tipo sin necesidad de publicidad ni cosas innecesarias porque siempre hay muchas opciones de donde escoger. 

Es necesario recuperar la calidez humana que nos brinda escuchar la información de otra persona, imaginar de cierta forma y creer en un mundo que es generado solo por una voz; aprovechar el Internet para que la gente de distintos rincones accedan a la información y a los pensamientos que queremos compartir.
El espíritu de compartir se ve truncado en veces por no acceder a las tecnologías por ser muy caras y ahí es donde entra el software libre a permitirnos compartir lo que queramos, y para la intensión anterior existe un conjunto de herramientas que permitirán que tengamos nuestra propia emisora en Internet o netradio, muy fácil de manejar y gratuita.
}

\vspace{2mm}

% Cuerpo del artículo
Siempre es mejor confiar en la estabilidad de un sistema GNU/Linux que en la de otros sistemas operativos; para este fin se recomienda ampliamente utilizar una distribución como Debian GNU/Linux o algo aún más libre como Trisquel GNU/Linux ya que nos ofrece paquetes de excelente calidad y muy estables. Por esta razón aparte de la cuestión moral de que sea libre, únicamente se utilizará tecnologías y contenidos libre que cumpla las 4 libertades del software según Richard Stallman adaptadas al tipo de contenido. La creación del sistema para la emisora en Internet se abordará con el siguiente método:\\

 1. żPor qué una radio en línea?\\
 2. Conceptos generales sobre servicios o demonios en GNU/Linux\\
 3. Terminología usada en la transmisión de medios\\
 4. Piezas de software usadas\\
 5. Alternativas para emisión de audio y ventajas y desventajas de éstas\\
 6. żQué música puedo poner?, licencias asociadas\\
 7. Ilustración general del método usado en Radio GLUD\\
 8. Instrucciones para la primera emisión en una red local\\

Con esto claro se entienden ya los alcances de este escrito y se quiere dejar en claro, que para entender lo que se hace para crear la emisora en línea es necesario que se expliquen los conceptos relacionados y la razón de todo lo que hallamos tenido que decidir, para que entendiendo esto se pueda automatizar procesos o mejorar la forma en que se hicieron, de lo contrario el artículo que convertiría en un tutorial más.

\begin{entradilla} %codigo para una entradilla
{\em {\color{introcolor}{Radio GLUD}} \\ Libertad para tus oídos.}
\end{entradilla}

Antes de desarrollar los contenidos, me parece necesario decir porque se cree tener el criterio necesario para escribir este artículo. Desde hace un tiempo existe la emisora en línea del GLUD, esta emisora se llama ``Radio GLUD'', y desde sus comienzos ha dado valioso conocimiento a todo el grupo con el cual se han solucionado algunos problemas con soluciones definitivas o parciales en que a pesar de todo el trabajo invertido hasta ahora, todavía falta un camino largo por recorrer para acercarla lo más posible a lo que puedan imaginar nuestras mentes.
La invitación cordial va a que visiten el sitio web de este proyecto \url{http://glud.udistrital.edu.co/radioglud}

\sectiontext{white}{black}{żPor qué una radio en línea?}

Es una pregunta que se hizo a su debido momento, por lo general se presentan muchas cosas a favor y otras en contra, pero las razones más valiosas que se encontraron fueron positivas y más enfocadas en {\em impacto social} que en cuestiones técnicas. La madre de todas es el poder social que se presenta cuando las personas comparten algo, pero żcuál era el medio correcto para compartir?; la radio es nuestro ancestro más cercano (para comunicarnos), todavía vivo, más interesante y nos ofrece las siguientes ventajas derivadas del poder social.

El primero de ellos es {\em el sentido de comunidad}, que se crea cuando muchas personas trabajan por una misma causa y es conocido que es mucho mas fácil hacer un programa radial que uno televisivo; el segundo es {\em compartir ideas y sentimientos}, que se ha visto mucho en otras radio, pero żrealmente son las que nos interesan en la comunidad académica?; la tercera es {\em las experiencias que se adquieren}, no puede ser solo libros, pueden haber otros espacios de esparcimiento muy provechosos; el cuarto es {\em compartir gustos musicales}, que hace que estemos más unidos, o żquién no ha cantado una canción que a todo mundo le gusta con un grupo de amigos?; y la última, {\em la creatividad como reflejo de sociedad}, tenemos derecho a expresarnos y nos volvemos más creativos (A mi parecer) solo mostrando nuestra voz que nuestros rostros.
\ebOpage{introcolor}{0.35}{Tecnologías Libres}%final de pagina 1

\sectiontext{white}{black}{Conceptos generales sobre servicios o demonios en GNU/Linux} %como se hace una sección

Primero abarcaré la forma en que el sistema operativo gestiona programas como icecast (ver sección Piezas de software usadas). Dicho programa se gestiona como un servicio (daemon en sistemas unix) donde la principal caracteristica es que es un programa que es persistente con una ejecución continua, se ejecuta en segundo plano y no posee una interfaz de usuario directa o shell.

Disk And Execution MONitor es lo que significa DAEMON; ``según una investigación realizada por Richard Steinberg, la palabra fue utilizada en 1963 por primera vez, en el área de la informática, para denominar a un proceso que realizaba backups en unas cintas, este proceso se utilizó en el proyecto MAC del MIT y en una computadora IBM 7094,1 dicho proyecto estaba liderado por Fernando J. Corbató, quien afirma que se basó en el demonio de James Maxwell''\cite {ref1}

Hay dos formas de iniciar un servicio si éste no se encuentra activo, la primera es ejecutando el programa relacionado con los parámetros para iniciarlo como daemon que indique el manual y la segunda es por medio de un mediador como service (programa en shell) que facilita la tarea ya que no hay que recordar la ruta ni el comando de ejecución del servicio.

\sectiontext{white}{black}{Terminología usada en la transmisión de medios} %sección

La transmisión de medios específicamente de audio, representa la acción esencial del proceso de crear una emisora, hay que tener claro que el proceso no es más que el streaming de audio que posee una interfaz web para poder escucharlo cómodamente.

\begin{center}
\resizebox{8cm}{!}{{\epsfbox{images/radioglud/mov-dat-mult.eps}}}
\mycaption{Gráfico del movimiento de los datos multimedia desde la generación del fuente hasta que lo escucha el usuario o oyente de la radio o canal de vídeo}
\end{center}

El streaming (que se refiere a corriente continua) es la distribución de contenido multimedia a través de una red de computadores con la ventaja de que por medio de un búfer de datos se agrega la posibilidad de disfrutar el contenido mientras este descarga; el componente búfer de datos en este caso es un espacio de memoria que almacena datos para evitar que la reproducción del medio se quede sin información durante la transferencia.

Latencia (relay - retardo del reloj) es el tiempo o retardos temporales en que se realiza un proceso o procesos (o en su defecto ninguno) debido por lo general al tiempo de propagación de los circuitos o incluso al tiempo en que se estabiliza que es debido a la densidad del material y la temperatura principalmente.

\begin{center}
\resizebox{8cm}{!}{{\epsfbox{images/radioglud/latencia.eps}}}
\mycaption{Gráfico de un sistema intermedio (sistema 2) donde se muestra que existe un tiempo desde que los datos entran hasta que salen del sistema y los causales posibles de ese Delta}
\end{center}

\begin{center}
\resizebox{8cm}{!}{{\epsfbox{images/radioglud/propagacion.eps}}}
\mycaption{Gráfico de los tiempos de propagación de un sistema, comenzando con el tiempo de propagación de bajo a alto (tplh) y terminando su respuesta con el tiempo de propagación de alto a bajo (tphl)}
\end{center}

El tiempo real (Real Time - RT), se presenta esta referencia para todos los sistemas en donde hay que considerar que ese tiempo real no tiene que ser necesariamente sinónimo de rapidez o inmediatez; como concepto, dicho tiempo es la latencia suficiente para que el sistema pueda resolver el problema para el cual está dedicado. Para el caso del audio en informática existe un tiempo de respuesta crítico con un porcentaje que no exceda el 3\% (aunque por lo general es así con todos los sistemas RT) que para llevarse a cabo necesita algunos componentes esenciales como un sistema operativo (OS) que soporte el tiempo real (para el caso de un sitema GNU/Linux solo sería el kernel) y un software diseńado para esto (el mismo OS RT ofrece un marco para el desarrollo de dichas aplicaciones).

Hay que distinguir entre 2 tipos de sistemas RT, los pasivos y los activos donde la diferencia es el dańo o fallo que puede causar un retraso en la seńal; por ejemplo tenemos que un sistema de audio o vídeo en directo sería un sistema RT pasivo debido a que un retraso en la seńal simplemente será un retraso en la visibilidad del vídeo o escucha del audio tal vez con desmejora del multimedia pero luego seguirá funcionando normal como si nada hubiera pasado y un ejemplo de un sistema RT activo son la mayoría de sistemas médicos (como el marcador de pasos artificiales o
\ebOpage{introcolor}{0.35}{Tecnologías Libres}%final de pagina 2
marcapasos) donde un retraso de la seńal puede causar el fallo del entorno donde el sistema interfiere causando una desmejora apreciable del paciente o incluso la muerte.

El servidor de sonido para streaming es un software con la función de gestionar los flujos de datos 
reproduciendo las muestras. Cualquier programa reproductor, grabador, gestor de audio, necesita soporte para trabajar con un determinado servidor de audio por ejemplo Jack Audio Connection Kit (en adelante simplemente JACK) es uno de los más conocidos para flujos en tiempo real.

El servidor de streaming de medios es un sistema de distribución de multimedia por streaming que utiliza un tipo de formato y un protocolo determinados para realizar el proceso de distribución de contenidos multimedia, para el caso de icecast el formato puede ser ogg, webm, aac o mp3 y el protocolo http o https.

El flujo multimedia se refiere al audio o al vídeo que es enviado desde un punto de la red hasta otro. Este flujo debe ser continuo aunque no necesariamente estable (fluctuante), pero debe tener un mínimo de velocidad de transferencia de un punto a otro para que el streaming sea efectivo, hay que introducir entonces algunos conceptos relacionados a las propiedades intrínsecas del archivo multimedia que se quiere enviar por streaming para estimar esa velocidad mínima que debe tener la red. Hay que tener en cuenta que cualquier archivo multimedia (audio y/o vídeo) posee un bitrate, que es la cantidad de bits que posee el flujo en unidad de tiempo, y que obedece a la composición del archivo, se mide por lo general en kbps (kilo bit por segundo).

También es importante conocer el término multi-bitrate que surgió de la necesidad de enviar una fuente multimedia a diferentes calidades, por ejemplo una calidad HD, normal y móvil; para cada una de estas es necesario un bitrate distinto para que el usuario pueda consumirlo de acuerdo a sus necesidades o limitación de ancho de banda de su conexión.

Ya que introducimos los bitrates y las velocidades de conexión de la red, es necesario entrar un poco a hacer el análisis de requerimientos de ancho de banda. Cuando queremos enviar un flujo multimedia desde nuestro computador hasta un servidor de streaming es pertinente revisar la velocidad de conexión (de subida) a este servidor para enviar nuestro flujo o flujos; para llevar a cabo este análisis es necesario realizar  un test de velocidad o Speed Test para saber nuestra velocidad real de conexión; hay que tener en cuenta que pueden presentarse fluctuaciones inesperadas en esa velocidad máxima por lo tanto es recomendable enviar flujos con bitrates que sumados estén dentro del rango de un 50\% a un 70\% de la capacidad total, una vez tengamos el dato podemos saber cuantos flujos y de que calidades podemos enviar a el servidor; por ejemplo si nuestra conexión es de 512 kbps se deben hacer los cálculos sobre la base de 358,4 kbps que es el 70\% de la conexión, con esto se puede enviar audio con bitrates de 256 kbps y 64 kbps sin mucho problema.

Source Stream (fuente) es entendido como la fuente de audio (o vídeo) de donde se obtienen los datos para despues distribuirlos en la red en forma de streaming, este fuente se genera desde un cliente como ices o idjc de los que se hablará más adelante.

Mount Point (Punto de Montaje) es una ruta por la cual se puede acceder a determinado streaming, debe ser configurada en el servidor de streaming, un ejemplo de un punto de montaje es ``/radioglud.ogg'' y para acceder a él desde un navegador web o un reproductor de audio la dirección se escribe de la forma http://ip:puerto/radioglud.ogg, no hay que preocuparse si no se ha entendido del todo ya que al hacer nuestro proceso de streaming en localhost el concepto llega sin mucho esfuerzo.

También vale la pena saber algunos conceptos de envío de información conocidas como {\em formas de transmisión de información} en una red, en lo que se consideran 3 como los principales conceptos, hay que decir que existen más pero éstos son los más relevantes y usados, el primero es {\em unicast} que consiste en que se envía la información uno a uno, donde obtiene la información el que lo necesita a través de requerimientos; el siguiente es {\em multicast} que necesita un rango de IPs ya reglamentado (conocidas como tipo D) para enviar simultáneamente los datos a todos los destinos escogidos a los que se les conoce como grupo multidifusión, no necesita enviar la información a cada destino por petición, sino que realiza una única transmisión de los fuentes a cada uno de los destinos y por eso no necesita una red con gran velocidad para realizar las transferencias; y el último de los tres es la forma {\em broadcast} que es parecido al anterior con la diferencia que se envía los datos a todos los puntos de la red o subred simultáneamente.

Icecast es unicast y hasta donde conozco no tiene soporte multicast, uno de los indicios principales para esta conclusión es el protocolo HTTP (Protocolo de Transferencia de HiperTexto) que es TCP (Protocolo de Control de Transmisión) con el que icecast distribuye el streaming, además de que supongo que como he leído desde hace un tiempo a pesar de que la multidifusión tiene muchas ventajas sobre el bandwidth (ancho de banda de la conexión de Internet) su tráfico es realmente complejo además de que su implementación a gran escala produce puntos de fallos que pueden ser utilizados para ataques de denegación de servicios. Como dato adicional solo los protocolos diseńados específicamente para trabajar con multicast son efectivos para su implementación y la mayoría de las aplicaciones conocidas lo hacen por UDP (Protocolo de Datagrama de Usuario).

%%%%%%%%%%%%%%%%%%%%%%%%%%%%%%%%%%%%%%%%%%%%%%%%%%%%%%%%%%%%
%\ebOpage{introcolor}{0.35}{INTRODUCCIÓN}
%%%%%%%%%%%%%%%%%%%%%%%%%%%%%%%%%%%%%%%%%%%%%%%%%%%%%%%%%%%%
%\rput(7.5,-2.0){\resizebox{10cm}{!}{{\epsfbox{images/radioglud/salsilla.eps}}}}

\begin{entradilla} %codigo para una entradilla
{\em {\color{introcolor}{Poder Social}} se presenta cuando las personas comparten.}
\end{entradilla}

\sectiontext{white}{black}{Piezas de software usadas}
A continuación se listan las aplicaciones que se han usado en Radio GLUD a través de su desarrollo, más adelante se dirá cuales de estas se usan actualmente.

\begin{itemize}
\item JACK: Es un servidor de sonido (daemon) de baja latencia, GNU GPL
\end{itemize}
\ebOpage{introcolor}{0.35}{Tecnologías Libres}%final de pagina 3
\begin{itemize}
\item Icecast: Es un proyecto y a la vez un programa servidor de streaming de medios, creado por la Xiph.Org Foundation, su propósito es crear un flujo de medios para una estación de radio, jukebox privado (rockola) y otros, a partir de un source client, GNU GPL v2
\item Ices: Es un programa generador de fuentes (source client) para un servidor de streaming, su propósito es generar un flujo de audio (stream) para el servidor de streaming del proyecto ICECAST, GNU GPL
\item IDJC: Cliente generador de fuentes para un servidor de streaming, es mucho más versátil que ices ya que permite mezclar distintos flujos de audio, provenientes de cualquier dispositivo o programa (incluso de si mismo) que pueden estar gestionados principalmente por el servidor de audio local. Su función es principalmente servir de estudio de audio casero, GNU GPL v2
\item QjackCtl: Es una aplicación QT, que nos da la posibilidad de controlar algunos parámetros de JACK (audio server daemon) de una forma simple y gráfica, GNU GPL v2
\item JACK Rack: Es un programa con un conjunto de efectos para JACK a traves de complementos (plugins) su procesamiento del audio es a baja latencia, GNU GPL
\item JAMin: Es una interfaz muy poderosa de compresión y ecualización de audio, GNU GPL
\item Liquidsoap:  Es un lenguaje de programación de audio para generar source streams, al ser un interprete con un lenguaje script dedicado la versatilidad y la adaptabilidad lo hace lo suficientemente poderoso para acomodarse a cualquier emisora, por complejo que se quiera realizar el(los) source stream(s), GNU GPL
\item jme (jmediaelement): Es una aplicación web, reproductor de audio y vídeo a través de elementos de HTML5.
\item Otras aplicaciones de audio muy útiles en \url{http://jackaudio.org/applications}
\end{itemize}

\begin{entradilla} %codigo para una entradilla
{\em {\color{introcolor}{Jack Audio Connection Kit}} multiplataforma y lo mejor que el software libre ha podido hacer por el mundo de la música.}
\end{entradilla}

\sectiontext{white}{black}{Alternativas para emisión de audio y ventajas y desventajas de estas}

Son muchas las aplicaciones que se utilizan en el entorno comercial para la emisión de medios, aunque contamos con pocas opciones de software libre, todas son de gran calidad y muy estables. Se listan las alternativas privativas que conozco como para dar ciertas observaciones al respecto, las que no conozco debido a que nunca las he usado, ni escuchado, ni desearía hacerlo aparecen con un ``--''.

\begin{center}
\begin{tablehere}
\begin{tabular}{|>{\columncolor{columnacolor}} c |>{\columncolor{columnacolor}} c |}
\hline
\multicolumn{2}{|>{\columncolor{filacolor}}c|}{Semi-equivalentes entre software libre y privativo}\\
\hline
\rowcolor{filacolor}Alternativa libre & Alternativa privativa \\ \hline 
Icecast & SHOUTcast\\ \hline
Icecast & oddcast o edcast\\ \hline
Ices, Darkice & Winamp, VLC\\ \hline
IDJC & Winamp, Zararadio\\ \hline
QjackCtl & --\\ \hline
JACK Rack & --\\ \hline
JAMin & Winamp\\ \hline
liquidsoap & --\\ \hline
jme (Reproductor HTML5) & Reproductor en Flash\\ \hline
WebcamStudio & WebcamMax\\ \hline
\end{tabular}
\caption{Equivalentes lo más parecido posible a las herramientas libres}
\end{tablehere}
\end{center}


Hace algún tiempo vi una conferencia en el CPCO4 (Campus Party Colombia), la verdad no recuerdo el nombre de la persona y no la encontré por la web por eso no la cito, en esta se enseńó como hacer una emisora online con software privativo. Algunas de las quejas del expositor en la conferencia eran, ``shoutcast solo funciona con winamp''; ``winamp solo trabaja con la tarjeta de audio escogida como principal''; ``la configuración de archivos de texto con notepad es feo (eso puede ser debido a que pone el texto todo de un solo color y posiblemente no se aprecia la indentación como debe ser)'' y dijo también ``estoy más perdido que el hijo de limber'' y por las constantes fallas en el proceso decía ``eso le pasa al chavo, al coyote y a mi''. Mis observaciones fueron que el equipo se colgaba fácilmente, la razón es que posiblemente el software era muy pesado, empezando por el sistema operativo.

Una de las cosas que si se hacían bien con el software privativo era cuando usaba herramientas como oddcast y zararadio aunque en sus versiones gratis poseen algunas limitaciones en cuanto a funcionalidad. Otra aplicación que usaba era WebcamMax que sirve para generar un fuente de vídeo con imágenes u otros vídeos para posteriormente unirla con el audio y enviarla a un canal en {\em ustream}, buscando un rato en la web encontré la alternativa libre {\em WebcamStudio} aunque por estar escrita para Java 1.6 ha resultado hasta el momento imposible hacerlo funcionar con nuevas versiones de Java, solo queda esperar a que la versión alpha siga desarrollándose y llegue a ser estable.

Queda rescatar que tu libertad está sobre todas las cosas, así que aunque algunas veces el software privativo sea más funcional no hay que usarlo porque luego te esclavizaras a los desarrolladores de dicha herramienta.

\sectiontext{white}{black}{żQué música puedo poner?, licencias asociadas}

Cuando hacemos nuestra propia emisora en línea es necesario conocer algunas leyes colombianas e internacionales que nos impiden poner música de cualquier tipo sin
\ebOpage{introcolor}{0.35}{Tecnologías Libres}%final de pagina 4
considerar que existen los derechos de autor (patrimoniales y conexos) y que hay que posiblemente pagar por ellos a un ente recaudador de derechos como Sayco y Acinpro (al parecer va a ser cambiada por otro ente debido a la corrupción del mismo), de lo contrario se está cometiendo un delito.

Los derechos de autor tienen distintas clases, entre los que interesan para el desarrollo de la pregunta están:
\begin{itemize}
\item Los derechos Morales: son los derechos ligados al autor y perduran por la eternidad, estos derechos son irrenunciables e imprescriptibles (ni ceder, vender o transferir).
\item Los derechos Patrimoniales: son los derechos que definen la explotación de la obra (retribuciones por uso, reproducción, difusión) hasta cierto límite después de la muerte del autor, después de este plazo se convierte en lo que se llama dominio público.
\item Los derechos Conexos: son los derechos que cobijan a personas distintas al autor (artistas, interpretes, traductores, editores, productores y otros)
\end{itemize}

\begin{center}
\begin{tablehere}
\begin{tabular}{|>{\columncolor{columnacolor}} c 
		|>{\columncolor{columnacolor}} c 
		|>{\columncolor{columnacolor}} c 
		|>{\columncolor{columnacolor}} c |}\hline
\multicolumn{4}{|>{\columncolor{filacolor}}c|}{Comparativa de tarifas de Sayco y Acinpro}\\ \hline
\rowcolor{filacolor}Lugar & Categoría & Estrato & Precio \\ \hline
Piojó 		& 6 & 1 & \$278.533   \\ \hline
Natagaima 	& 4 & 2 & \$371.472   \\ \hline
Ibagué 		& 2 & 2 & \$566.700   \\ \hline
Bogotá 		& E & 3 & \$1.133.400 \\ \hline
\end{tabular}
\caption{Datos tomados de \url{http://www.saycoacinpro.org.co/test_tarifas.php} con el código 102 (Establecimientos de cualquier naturaleza que para fines de su objeto social vendan productos para llevar y consumir o usar en sitios diferentes al del establecimiento comercial. Ejemplo: Almacenes de cadena, Supermercados, Hipermercados, Bancos, Estación.) y Capacidad/Aforo de 50 personas}
\end{tablehere}
\end{center}

Tal vez para una emisora libre de publicidad o una emisora estudiantil, sería muy difícil invertir recursos para la música y tener que suspender el uso de piezas musicales cuando es bien conocido que la música (ver arriba los argumentos) hace parte importante del proceso de comunidad que es lo que se busca en un grupo como GLUD.

Existen algunos tipos de productos musicales debidos al cumplimiento de los derechos anteriormente nombrados y que se podrán utilizar en nuestra emisora teniendo en cuenta que uso es el que se le va a dar
\begin{itemize}
\item Música libre: Es la música que cumple los mismos principios del software libre
\item Música con algunos derechos reservados: Es la música que no permiten las libertades de obra derivada y/o uso comercial.
\end{itemize}
Dentro de la música que se puede usar está la de Dominio Público, mucha de la mejor música del planeta se encuentra bajo esta licencia, entre ellas obras de arte como la música clásica; también encontramos piezas de audio que el autor ha decidido registrar con licencias como Creative Commons, Coloriuris o ArtLibre.

Las licencias más usadas actualmente para contenidos libres o semilibres son las Creative Commons que son 6 licencias que surgen de la conjugación de las características o funciones que son, Atribución (by), No comercial (nc), Sin derivar (nd) y Compartir igual (sa). Las licencias libres\cite {ref2} son CC-by y CC-by-sa, las otras 4 que son CC-by-nd, CC-by-nc, CC-by-nd-nc y CC-by-nc-sa son las que se consideran semi-libres o con algunos derechos reservados. Pueden buscar su licencia ideal en la dirección \url{http://creativecommons.org/choose/} y posteriormente poner los meta-datos en tu sitio web.

Estas licencias CC (Creative Commons) según cuenta la historia surgió debido a la aparición del Internet (hay que darse cuenta la importancia de esta en lo que que consideramos que es la libertad por lo menos en la tecnología) y el espíritu de compartir que hicieron que una persona Eric Eldred fuera subiendo a la red obras que quedaban en Dominio Público cada ańo, en la escena apareció Mary Bono viuda de Sonny Bono un actor estadounidense que quería que los derechos de autor durarán para siempre, y luego de su muerte, Mary Bono tomó el cargo que él tenía en la cámara de representantes y comenzó a promover lo que ella consideraba la última voluntad de su marido.

Todo este lío resulta ser anticonstitucional pero igual se aumentó los ańos que se protegía el derecho patrimonial del autor, situación que causó descontento (para Eric Eldred y sus seguidores) ya que para ese ańo y otros venideros no habría liberación de obras; en esté dilema aparece Lawrence Lessig con un argumento contundente ``en la constitución de USA decía que se garantizaba el derecho monopólico por un tiempo limitado para asegurar y promover el progreso'', hubieron discusiones hasta que lo denominado ``The Mickey Mouse Protection Act''\cite {ref3} impuso con su poderío una influencia lo bastante grande para que el tiempo del derecho patrimonial se extendiera sin regulación aparente cada vez que llegaba la hora de que Mickey Mouse pasara a dominio público\cite {ref4} en 1976 decidieron parar estos cambios casi anuales. 

Mickey Mouse ahora se esconde bajo una marca registrada que impide usarlo por ser prácticamente el logo de la compańía The Walt Disney Company y en el transcurso de todo esto Lawrence Lessig creó lo que actualmente es la organización sin fines de lucro Creative Commons como oposición directa al copyright.

\begin{entradilla} %codigo para una entradilla
{\em {\color{introcolor}{Internet}} y ĄĄĄel espíritu de compartir!!!.}
\end{entradilla}
\ebOpage{introcolor}{0.35}{Tecnologías Libres}%final de pagina 5
\end{multicols}

\sectiontext{white}{black}{Ilustración general del método usado en Radio GLUD}

\begin{center}
\myfig{0}{images/radioglud/radioglud.eps}{0.8}
\mycaption{Gráfico del método usado en Radio GLUD, consiste en 2 computadores donde el client streaming que genera el source stream es Mafalda GLUD y el servidor de streaming es glud.udistrital.edu.co}
\end{center}

\begin{multicols}{2}

Como se puede ver el método (ha de entenderse método como el esquema general o a grandes rasgos de la gestión, configuración y flujo del audio) implementado hasta ahora no incluye el uso de liquidsoap, y hay dos fuentes (generada por IDJC) y 2 puntos de montaje con baja (64 kbps) y alta (128 kbps) calidad de sonido (configurando el bitrate de las fuentes en IDJC) en el formato ogg vorbis, Se recomienda hacer bien los cálculos para que se pueda dar la suficiente calidad si son pocos o muchos los usuarios.\cite{ref5} La desventaja de éste es que claramente no hay un sistema de respaldo y en caso de que el IDJC falle, nuestro punto de montaje desaparecerá, se recomienda tener otros fuentes que deberían ser administrados a través de liquidsoap.

\begin{entradilla} %codigo para una entradilla
{\em {\color{introcolor}{``Aquel que sacrifica la libertad por seguridad, no merece ninguna de las dos''}} -- Benjamin Franklin}
\end{entradilla}
El ideal para todos estos tipos de sistemas se encuentra en la wiki de RadioŃú la dirección de dicho esquema se encuentra en \url{http://wiki.radiognu.org/como_funciona_una_radio} éste esquema es la proyección a futuro de Radio GLUD; pongo la imagen para que se pueda apreciar rápidamente.

\begin{center}
\resizebox{8cm}{!}{{\epsfbox{images/radioglud/radiognu1.eps}}}
\mycaption{Esquema de como debería ser el flujo de datos en un sistema de streaming, extraído de \url{http://wiki.radiognu.org/como_funciona_una_radio}}
\end{center}

Si desean saber más de las tecnologías usadas en Radio GLUD, pueden visitar la dirección \url{http://radio.glud.org/files/documentos/InformeFinaldeActividades-RadioGLUD1.odt} donde se explica detalladamente todo lo que se ha hecho y donde se puede descargar.

Para finalizar esta sección quiero dar crédito a los proyecto RadioŃú y Radio Liberación que ha dado mucho a la comunidad hispano-hablante ya que que con su documentación han dado un paso increíble para los que no sabemos leer muy bien Inglés (o definitivamente no sabemos).

\sectiontext{white}{black}{Instrucciones para la primera emisión en una red local}

Ahora que ha terminado toda la teoría se hace necesario
\ebOpage{introcolor}{0.35}{Tecnologías Libres}%final de pagina 6
ir a lo técnico, pero antes hay que aclarar la notación que se va a utilizar. La distribución en que se hizo fue Fedora GNU/Linux 16 Verne con gnome 3, para evitar todos estos pasos se podría instalar una distribución como Musix GNU+Linux que es una distro dedicada a la música; la notación \verb!$ comando! representa los comandos en una terminal como usuario normal (sin permisos de root o administrador); la notación \verb!# comando! representa la ejecución de los comandos especificados como root, para ello solo basta con digitar \verb!$ su! y escribir la clave de root, en caso de que se maneje con la utilidad \verb!sudo! habría que digitarse \verb!$ sudo su! y la clave del usuario común; en particular debido a que \verb!sudo! es una herramienta muy conveniente para copiar y pegar comandos en terminal sin cambiar de usuario, es la que se utiliza en este mini-tutorial, pero si no se tiene, pueden loquearse como root y ejecutar solamente los comandos que están con sudo.

También es importante decir que los pasos no están tan detallados como se quisiera, pero se espera que con la teoría se pueda deducir lo que se dice que se haga o buscar en Internet para que quede claro, por ejemplo como se modifican los archivos con el editor nano, el lenguaje usado en los archivos o incluso como se abre una terminal y que es, ya que lastimosamente no hace parte de lo que se quiere en este artículo, pero por suerte hay mucha información en la red sobre estos temas.

\begin{enumerate}

\item Instalar los paquetes. 

\lstset{language=html,frame=tb,framesep=2pt,basicstyle=\footnotesize} 
\begin{lstlisting}
$ sudo yum install -y jack-audio-connection-kit 
		alsa-plugins-jack qjackctl 
		pulseaudio-module-jack idjc
\end{lstlisting}

\item Agregar tu usuario al los grupos relacionados con el audio.

\begin{lstlisting}
$ sudo usermod -a -G audio $USER
\end{lstlisting}

\item Configurar el archivo \verb!limits.conf!. 

\begin{lstlisting}
$ sudo nano /etc/security/limits.conf 
\end{lstlisting}

Agregar al final del archivo las líneas.

\begin{lstlisting}
@audio - rtprio 99
@audio - memlock unlimited
@audio - nice -10
\end{lstlisting}

\item Modificar el archivo \verb!default.conf!.
 
\begin{lstlisting}
$ sudo nano /etc/pulse/default.pa 
\end{lstlisting}

Dejar la parte del archivo que se muestra de esta forma:

\begin{lstlisting}
### Load audio drivers statically (it's 
		probably better to not load
### these drivers manually, but instead 
		use module-udev-detect --
### see below -- for doing this automatically)
#load-module module-alsa-sink
#load-module module-alsa-source device=hw:1,0
#load-module module-oss device="/dev/dsp" 	
	sink_name=output source_name=input
#load-module module-oss-mmap device="/dev/dsp" 
	sink_name=output source_name=input
#load-module module-null-sink
#load-module module-pipe-sink
load-module module-jack-source
load-module module-jack-sink

### Automatically load driver modules depending 
		on the hardware available
#.ifexists module-udev-detect.so
#load-module module-udev-detect
#.else
### Alternatively use the static hardware 
	detection module (for systems that
### lack udev support)
#load-module module-detect
#.endif
\end{lstlisting}

\item Agregar QJackCtl a las aplicaciones en el inicio, para este fin ejecutar el comando.

\begin{lstlisting}
$ gnome-session-properties
\end{lstlisting}

\item Abrir QJackCtl y configurarlo activando las opciones.

    Iniciar el servidor jack al iniciar qjackctl
    Habilitar ícono en bandeja del sistema
    Iniciar minimizado en la bandeja del sistema

\item Para activar el soporte de mp3 para IDJC instalar. 

\begin{lstlisting}
$ sudo yum install -y lame lame-libs
\end{lstlisting}

\item Ahora por un problema de nombres de algunos archivos ejecutar. 

\begin{lstlisting}
$ sudo ln -s /usr/lib64/libmp3lame.so.0.0.0 
		/usr/lib64/libmp3lame.so
$ sudo ln -s /usr/lib64/libmad.so.0.2.1 
		/usr/lib64/libmad.so
\end{lstlisting}

\item Solución de problemas comunes: Si al completar todos los pasos por alguna razón no te funciona el arranque del servidor jackd con QJackCtl, prueba quitando el pulseaudio de la lista de aplicaciones al inicio y de no funcionar algunos lo solucionan destildando el tiempo real en el setup de QJackCtl e incluso he escuchado que ejecutándolo como sudo funciona, y pues por si las dudas miren que la interfaz que se encuentra por defecto en (default) sea la correcta.

\item Ahora se va a instalar icecast.

\begin{lstlisting}
$ sudo yum install -y icecast
\end{lstlisting}

\item Abrimos el archivo de configuración.

\begin{lstlisting}
$ sudo nano /etc/icecast.xml
\end{lstlisting}

Modificar los parámetros

\begin{itemize}
\item Cambiar en la línea 49
\begin{lstlisting}
    <hostname>localhost</hostname>
\end{lstlisting}
por la ip local del equipo, por ejemplo:
\begin{lstlisting}
    <hostname>192.168.0.5</hostname>
\end{lstlisting}
\item Agregar en la línea 121 el punto de montaje
\begin{lstlisting}
   <mount>
	<mount-name>/stream.ogg</mount-name>
        <username>hackme</username>
        <password>hackme</password>
        <max-listeners>100</max-listeners>
    </mount>
\end{lstlisting}
\end{itemize}

\item Iniciar el servidor Icecast.

\begin{lstlisting}
$ sudo service icecast start
\end{lstlisting}

\item Ahora se inicia IDJC, se agrega música y se agrega una nueva conexión de icecast 2 master con los datos del punto de montaje que se pusieron arriba; posteriormente se establece conexión con dicho servidor.

\item Si todo resulta bien podrán acceder a el streaming local a través de la dirección \verb!http://192.168.0.5:8000/stream.ogg! bien sea con VLC o con algún otro reproductor que soporte streaming.

\end{enumerate}

\bibliographystyle{abbrv}
\begin{bibliografia}
\bibitem{ref1}
Copia textual del párrafo 3 del artículo Demonio (informática). Wikipedia la enciclopedia libre.
Recuperado de \url{es.wikipedia.org/wiki/Demonio_(informática)} el 19 de agosto de 2012.
\bibitem{ref2}
Artículo Música Libre. Wikipedia la enciclopedia libre.
Recuperado de \url{es.wikipedia.org/wiki/Música_libre} el 10 de agosto de 2012.
\bibitem{ref3}
Artículo Mickey Mouse. Wikipedia la enciclopedia libre.
Recuperado de \url{es.wikipedia.org/wiki/Mickey_Mouse} el 19 de agosto de 2012.
\bibitem{ref4}
Artículo Copyright Term Extension Act. Wikipedia the free encyclopedia.
Recuperado de \url{en.wikipedia.org/wiki/Sonny_Bono_Copyright_Term_Extension_Act} el 19 de agosto de 2012.
\bibitem{ref4}
Artículo Estándares para Calidad y Ancho de Banda. RadioŃú.
Recuperado de \url{wiki.radiognu.org/calidad} el 22 de agosto de 2012.
\end{bibliografia}


\begin{biografia}{images/radioglud/autor.eps}{Jorge Ulises Useche Cuellar} Estudiante de Ingeniería Electrónica, Interesado en el desarrollo de tecnologías libres que reduzcan la brecha social que existe hoy en día en Colombia al ser un país subdesarrollado. Viniendo de otro sector de Colombia encontró una simpatía inmediata al conocer el software libre e invierte su tiempo conociendo y desarrollando para la comunidad de software libre que le ha dado tanto.
\end{biografia}

\end{multicols} %termina el entorno multicols
%\eOpage %comienza una pagina nueva



\clearpage
\pagebreak

% Esta obra está bajo una licencia Reconocimiento 2.5 Espańa de Creative
% Commons. Para ver una copia de esta licencia, visite 
% http://creativecommons.org/licenses/by/2.5/es/
% o envie una carta a Creative Commons, 171 Second Street, Suite 300, 
% San Francisco, California 94105, USA.

% Seccion Introducción
%

\rput(2.5,-2.3){\resizebox{!}{5.7cm}{{\epsfbox{images/pfipy/Kokopelli.eps}}}}%Imagen de el comienzo de el articulo, coordenadas desde 
                                                                                   %la parte superior izquierda del margen de la pagina

% -------------------------------------------------
% Cabecera
\begin{flushright}
\msection{introcolor}{black}{0.25}{ENSAYO} %titulo de la sección

\mtitle{10cm}{Programación Funcional en Python} %titulo del articulo 

\msubtitle{8cm}{Un paradigma en nuestro radar} %subtitulo

{\sf por José Javier Vargas Serrato} %autor

{\psset{linecolor=black,linestyle=dotted}\psline(-12,0)}
\end{flushright}

\vspace{2mm}
% -------------------------------------------------

\begin{multicols}{2}


% Introducción
\intro{introcolor}{E}{l paradigma funcional aparece como una necesidad a la falencia que proporcionaba el paradigma imperativo 
en el campo de la  investigación, área como la  inteligencia artificial, pruebas de teoremas, procedimientos de lenguajes naturales,
se veían muy limitados, dando lugar a los  lenguajes de programación funcionales puros e híbridos.
python es un lenguaje de programación híbrido que puede proporcionar enfoques dogmáticos a cada uno de estos paradigmas.
}

% Cuerpo del artículo
\sectiontext{white}{black}{Programación Funcional} \\ %como se hace una sección
La programación funcional declarativa pertenece a una clasificación de los paradigmas de programación, 
Es descritas por las cualidad de implementar  meramente mecanismos de evaluación de expresiones,  
``en vez de la ejecución de comandos como lo hace la programación imperativa'' por medio de funciones, 
esto quiere decir que utiliza funciones como factor principal en el tratamiento  computacional, evitando 
los estados o datos mutable.\\
Existen lenguajes de programación dedicados  a este paradigma, denominados lenguajes de programación funcional, 
aquellos que son exclusivos en esta concepción  se les denominan ``funcionales puros'' y son muy implementados en el 
campo académico y de investigación, sin embargo no quiere decir que no hallan explorado el campo comercial.\\
Existen varios casos de lenguajes funcionales muy famosos en los que se han implementado en aplicaciones comerciales e industriales, 
para satisfacer necesidades exclusivas. También lenguajes de uso especifico como SQL han utilizado  elementos y cualidades 
de la programación funcional para mejor sus procesos. 
Ahora bien, vemos como algunos lenguajes de programación que no esta diseńados específicamente para la programación funcional,
están cada vez más implementándola e incorporando capacidades de esta, es el caso de Perl, JavaScript, Python, a estos se les 
denominan lenguajes funcionales híbridos.\\
Entre las características que toma Paython de la programación funcional están las Funciones de orden superior o las funciones 
lambda (funciones anónimas).\\
Las Funciones de orden se refiere a  la manipulación de defunciones como si fuera un valor cualquiera cualquier, posibilitando 
que las funciones puedan pasarse como parámetros a otras funciones, devolver funciones como valor de retorno, cualquier sentencia 
puede convertirse en función, las funciones pueden almacenarse en estructuras de datos.\\

\sectiontext{white}{black}{Phyton y la programación funcional} %como se hace una sección
Python nos proporcionan unos features o ``característica'' para iterar las funciones al estilo programación funcional (P.F) pero de 
una manera más fácil. żPor que esto? en la programación funcional no existen construcciones estructuradas como las secuencias y esto 
obliga a  hacerse con funciones recursivas.\\
Para no violar el paradigma de programación funcional implementando los bucles típicos del lenguaje imperativo, python nos 
proporciona estos ``features, funciones o sentencias'' que son equivalentes y son especializadas u orientados para el paradigma P.F.\\
A continuación hablaremos de estas funcionase que proporciona python; una breve descripciones  y un pequeńo ejemplo para entenderlo.\\

\sectiontext{white}{black}{MAP}\\ %como se hace una sección
\textit{\textbf{map(funcion1, secuencia[elemento1, elemento2.....])}}\\
la funciona map aplicara a cada elemento de la secuencia la funcion1, devolviendo una lista con el resultado de cada operación entre la funcion1 
y el elemento de la secuencia.\\

\lstset{language=Python,frame=tb,framesep=5pt,basicstyle=\footnotesize}   
\begin{lstlisting}
def multiply(n):
  return n * 2
l1 = [3, 4]
lr = map(multiply, l1)
print lr		# imprime [6, 8]
\end{lstlisting}

\sectiontext{white}{black}{FILTER}\\ %como se hace una sección
\textit{\textbf{filter(funtion, sequence)}}\\
la funcion filter verifica que los  elementos de una secuencia cumplan una determinada condición. Devolviendo una secuencia con los elementos que cumplan
esa condición. Es decir para cada elemento de sequence se aplica la fucion, si el resultado es true se ańade a la lista y en caso contrario se descarta. 

%%%%%%%%%%%%%%%%%%%%%%%%%%%%%%%%%%%%%%%%%%%%%%%%%%%%%%%%%%%%
\ebOpage{introcolor}{0.35}{INTRODUCCIÓN}
%%%%%%%%%%%%%%%%%%%%%%%%%%%%%%%%%%%%%%%%%%%%%%%%%%%%%%%%%%%%

\lstset{language=Python,frame=tb,framesep=5pt,basicstyle=\footnotesize}   
\begin{lstlisting}
def esPar(n):
  return (n % 2 == 0)
l = [1, 2, 3]
lr = filter(esPar, l)
print lr	# imprime [2]
\end{lstlisting}

la definición anterior de Filter y el ejemplo son extraído del libro \cite{kopka}Python para todos por Raúl Gonzáles Duque.  
rápidamente notamos que todo depende del valor de retornado por la función definida, esta  función debe retornar datos booleanos,
de lo contrario Filter no cumplirá con su objetivo..\\

Ejemplo erróneo\\
\lstset{language=Python,frame=tb,framesep=5pt,basicstyle=\footnotesize}   
\begin{lstlisting}
def multiply(n):
  return n * 2
l1 = [3, 4]
lr = filter(multiply, l1)                  
print lr	#imprime [3, 4]
\end{lstlisting}
Filter actúa  frente a la respuesta booleana de la función. .\\
\lstset{language=Python,frame=tb,framesep=5pt,basicstyle=\footnotesize}   
\begin{lstlisting}
def esPar(n):
  return True
l = [1, 2, 3]
lr = filter(esPar, l)
print lr	 # imprime [1, 2, 3]
\end{lstlisting}
\sectiontext{white}{black}{REDUCE}\\ %como se hace una sección
\textit{\textbf{Reduce(funtion, sequence[ ])}}\\
reduce aplica la funtion a pares de elementos de la secuencia hasta operarlos todos.\\

\lstset{language=Python,frame=tb,framesep=5pt,basicstyle=\footnotesize}   
\begin{lstlisting}
def resta(x, y):
  return (x - y)
l = [15, 15, 7]
lr = reduce(resta, l)
print lr 	# imprime -7
\end{lstlisting}

como podemos ver reduce opera el primer par de elementos de la secuencias en este caso 15-15,
dando como resultado 0 este cero ahora se opera con el siguiente elemento de de la secuencias, 
dando como resultado final el -7, ya que entran como ultima iteración en el valor de X el 0 ``cero''
resultado de la primera iteración en el Y el 7.\\

\sectiontext{white}{black}{FUNCIÓN LAMBDA}\\ %como se hace una sección
el operador lambda sirve para crear funciones anónimas, esto quiere decir que no tiene nombre y que no se podrán llamar después.\\
\textit{
\textbf{Estructura}\\
\textbf{lambada parametro1, parametro2: cuerpo de la función}}\\

\lstset{language=Python,frame=tb,framesep=5pt,basicstyle=\footnotesize}   
\begin{lstlisting}
l = [2, 3, 4]
lr = map(lambda n: n * 2, l)
print lr 	# imprime [4,6,8]
\end{lstlisting}

\sectiontext{white}{black}{LIST COMPREBENSIONS }\\ %como se hace una sección

en la versión de python 3000 también llamada python 3.0 o py3k las funcionas map, filter y reduce  
perderán protagonismo para darle pazo a comprensión de litas.\\
La compresión de listas es una característica tomada del lenguaje de programación funcional Haskell que esta presente desde python 2,0, esta permite crear listas a partir de otras listas.
Cada construcción consta de una expresaron que sera la que determine la modificación a cada uno de los elementos de la lista original, seguida de unas clausulas for y opcionalmente una o 
varias clausulas if.\\
\textit{\textbf{L2 = [expresión modificadora for CadaElemento in ListaOriginal]}}\\

\lstset{language=Python,frame=tb,framesep=5pt,basicstyle=\footnotesize}   
\begin{lstlisting}
l = [1, 2, 3]
l2 = [n + 2 for n in l]
print l2 	# imprime [3, 4, 5]
\end{lstlisting}
esta expresión se leería ``para cada n en l hacer n+2''. Como podemos ver la expresión modificadora es n + 2 , seguido del for,
el nombre que vamos a utilizar para identificar a cada elemento de la lista Original, el in  y la lista Original que sera la
que le hacemos las operaciones.\\
Este resultado puede asignarse a otra lista o a si misma.\\

\lstset{language=Python,frame=tb,framesep=5pt,basicstyle=\footnotesize}   
\begin{lstlisting}
l = [1, 2, 3]
l = [n + 2 for n in l]
print l 	# imprime [3, 4, 5]
\end{lstlisting}
otro ejemplo en el implementamos el if.\\

\lstset{language=Python,frame=tb,framesep=5pt,basicstyle=\footnotesize}   
\begin{lstlisting}
l = [2, 3, 4, 5]
lr = [n for n in l if n % 2 == 0]
print lr	# imprimir [2, 4]
\end{lstlisting}

%%%%%%%%%%%%%%%%%%%%%%%%%%%%%%%%%%%%%%%%%%%%%%%%%%%%%%%%%%%%
\ebOpage{introcolor}{0.35}{INTRODUCCIÓN}
%%%%%%%%%%%%%%%%%%%%%%%%%%%%%%%%%%%%%%%%%%%%%%%%%%%%%%%%%%%%

Otro ejemplo en que implementamos varios for.\\

\lstset{language=Python,frame=tb,framesep=5pt,basicstyle=\footnotesize}   
\begin{lstlisting}
l = [0, 1, 2, 3]
m = ["a", "b"]

lr = [s * v for s in m
	     for v in l
	     if v > 0
       ]
# imprime ['a', 'aa', 'aaa', 'b', 'bb', 'bbb']
print lr 

# totalmente equivalentes

lr2 = []
for s in m:
  for v in l:
    if v > 0:
      lr2.append(s * v)
      
# imprime ['a', 'aa', 'aaa', 'b', 'bb', 'bbb']
print lr2 
\end{lstlisting}


\sectiontext{white}{black}{GENERADOR}\\ %sección
esta expresión generador  funciona de forma muy parecida a las comprensión de lista y su sintaxis es la misma excepto que no se utiliza
paréntesis ``()'' sino corchetes ``[]''. su gran diferencia es que no devuelve una lista como en el caso de la comprensión de lista  sino 
un generador.\\

\lstset{language=Python,frame=tb,framesep=5pt,basicstyle=\footnotesize}   
\begin{lstlisting}
l = [1, 2, 3]
lr1 = [n + 2 for n in l]
print lr1	   # imprime [3, 4, 5]
lr2 = (n + 2 for n in l)
# imprime 
#<generator object <genexpr> at 0x7f1c78b6e7d0>
print lr2	
\end{lstlisting}

En caso de utilizare en una función para retornarlo se utiliza yield en vez de return.\\

\lstset{language=Python,frame=tb,framesep=5pt,basicstyle=\footnotesize}   
\begin{lstlisting}
def generador(n):
  yield n + 1
  
#imprime 
#<generator object generador at 0x7ffc0b23a7d0>
print generador(2)           
\end{lstlisting}

Entonces que extraordinarias propiedades otorga Generador.?\\

\lstset{language=Python,frame=tb,framesep=5pt,basicstyle=\footnotesize}   
\begin{lstlisting}
def mi_generador(n, m, s):
  while(n <= m):
    yield n
    n += s

x = mi_generador(0, 5, 2)

# imprime 
#<generator object mi_generador at 0x7f710e1b67d0>
print x 

for i in x:
  print i		# imprime de 0 a 4

for n in mi_generador(0, 5, 2):
  print n		# imprime de 0 a 4         
\end{lstlisting}

Que esta sucediendo aquí? El generador solo crea un solo valor cada vez  que se necesite. Optimizando
la memoria, caso contrario sucede con las listas que se crea toda su composición en memoria.\\

\cite{kopka} \textit{La idea de los Generador se basa en dos conceptos: los iteradores y la evaluación perezosa.
En un FOR Por debajo, el intérprete lo que hace es crear un iterador mediante el método iter() y avanzar posiciones con el método next() del iterador devuelto.
La evaluación perezosa (lazy evaluation), dice que una expresión se resuelve únicamente cuando se la necesita, no antes.
Uniendo ambos conceptos tenemos el generador: una estructura, sobre la que podemos iterar, cuyos componentes se van obteniendo según se va avanzando.
La diferencia fundamental con la lista, es que la lista tiene una evaluación ansiosa/codiciosa(eager evaluation). Esto implica que cuando queremos manejar una 
lista de 10 elementos la lista necesita disponer de esos 10 elementos en el momento de su declaración. Sin embargo el generador solamente necesita saber cómo 
generar el siguiente valor por lo que no necesita ni reservar espacio ni conocer a priori ningún elemento.}\\

El paradigma de programación funcional no es la panacea, incluso ningún paradigmas lo es, veo que la riqueza de cada uno de 
ellos es que proporciona criterios teóricos y prácticos  para atacar a un problema frente a unas necesidades especificas y 
obtener  un mejor manipulación de la información y mayor efectividad para obtener los resultados deseados..\\

Existen muchos paradigmas de programación, pero ninguno predomina en los desarrollos, precisamente porque los requerimientos de cada
uno de ellos van orientados a diferentes necesidades, y es de cada paradigma proporcionar soluciones adecuadas con un  modelo coherente
, flexible y reutilizable.\\

El lenguaje python no nos impide fusionar el paradigma funcional e imperativo, pero esto nos dará un desarrollo en el cual no 
obtendremos los beneficios de ninguno, ya que las cualidades intrínsecas son excluyentes; Me explico ?lo que hace fuerte a la una es
por no actuar diferente a la otra y viceversa?.\\

A la hora de decidir que paradigmas implementar en la elaboración de desarrollo, lo mas adecuado es incluir en el equipo 
de trabajo personal con talento y experiencia en modelo de negocio, mas que en las mismas tecnologías, ya que este conocimiento
proporcionara la base fundamental para elegir que paradigmas proporcionan mayor efectividad para el sistema.\\

Cuando implementemos algún paradigmas de programación, es de vital importancia no estructurarnos hacer lo que ellas dicten 
al pie de la letra, ya que podría llegar a convertirse en un arma de doble filo, y combatir nuestro desarrollo en un trabajo tedioso,
es mas una tarea de diversión y creatividad, analizar la mejor manera en que dicho paradigma pueda proporcionarle mejor 
funcionalidad a nuestro sistema; aunque siempre existe la barrera de implementar Vs tiempos de cumplimiento.\\


\bibliographystyle{abbrv}
\begin{bibliografia}
\bibitem{kopka}
Raúl Gonzáles Duque, 
\emph{Python para todos}, 
\hskip 1em plus 0.5em minus 0.4em
\relax Disponible en: http://mundogeek.net/tutorial-python/ 

\bibitem{kopka1}Hernando, Carlos. (jue 01 septiembre 2011).
\emph{chernando.eu. Recuperado el 14 de Febrero de 2013}
\hskip 1em plus 0.5em minus 0.4em
\relax Disponible en: http://chernando.eu/python/python-generators/

\bibitem{kopka2}Web Document
\emph{}
\hskip 1em plus 0.5em minus 0.4em
\relax Disponible en: http://sinusoid.es/python-avanzado/slides/tema-01.pdf

\bibitem{kopka3}Web Document
\emph{Departamento de Electrónica, sistema informático y automática Universidad de Huelva}
\hskip 1em plus 0.5em minus 0.4em
\relax Disponible en: https://we.riseup.net/assets/68470/progFuncional-Slides.pdf

\bibitem{kopka4}Web Document
\emph{Rivadera, Gustabo Ramiro(cuadernos de la facultad n 3, 2008)}
\hskip 1em plus 0.5em minus 0.4em
\relax Disponible en: http://www.ucasal.net/templates/unid-academicas/ingenieria/apps/3-p63-Rivadera.pdf
  
\end{bibliografia}


\begin{biografia}{images/pfipy/jota.eps}{José Javier Vargas Serrato} % ańadir fotografía tamańo [2.5 cm x 3.3 cm ]
Estudiante de Ingeniería de Systemas en la universidad Distrital Francisco José de Caldas, 
es activista del movimiento del software libre.
En la actualidad se desenpeńa como miembros activos del GLUD (Grupo GNU/Linux de la Universidad Distrital
Francisco José de Caldas), es participe de el proyecto  \textbf{Equidna \textit{La belleza de Python}} y desea empezar a colaborar con la edicion de la revista GLUD 
Magazine.  
\end{biografia}

\end{multicols} %termina el entorno multicols
%\eOpage %comienza una pagina nueva

%\rput(7.5,-2.0){\resizebox{10cm}{!}{{\epsfbox{images/mi_articulo/salsilla.eps}}}}

\clearpage
\pagebreak



\end{document}
