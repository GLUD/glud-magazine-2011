% Esta obra está bajo una licencia Reconocimiento 2.5 Espańa de Creative
% Commons. Para ver una copia de esta licencia, visite 
% http://creativecommons.org/licenses/by/2.5/es/
% o envie una carta a Creative Commons, 171 Second Street, Suite 300, 
% San Francisco, California 94105, USA.

% Seccion Introducción
%

\rput(2.5,-2.3){\resizebox{!}{5.7cm}{{\epsfbox{images/ensayo/Kokopelli.eps}}}}%Imagen de el comienzo de el articulo, coordenadas desde 
                                                                                   %la parte superior izquierda del margen de la pagina

% -------------------------------------------------
% Cabecera
\begin{flushright}
\msection{introcolor}{black}{0.25}{OPINIÓN} %titulo de la sección

\mtitle{12cm}{Implicaciones de utilizar software propietario para un integrante Egresado del GLUD} %titulo del articulo 

\msubtitle{8cm}{Análisis sobre una cruda realidad} %subtitulo

{\sf Por: Fernando Pineda} %autor

{\psset{linecolor=black,linestyle=dotted}\psline(-12,0)}
\end{flushright}

\vspace{2mm}
% -------------------------------------------------

\begin{multicols}{2}


% Introducción
\intro{introcolor}{E}{n el presente ensayo se pretende desarrollar un análisis de las implicaciones que puede tener el
usar software privativo por parte de uno de los miembros egresados del GLUD (Grupo GNU/Linux de la Universidad Distrital), 
en ese aspecto de desarrollan de forma mas detallada tres aspectos antes de llegar a lo que son la implicaciones reales y 
unas pequeńas conclusiones. 
}

\vspace{2mm}
% Cuerpo del artículo
Primero se establece en la sección ``La Elección'', los parámetros que pueden influir en un usuario, miembro egresado
del GLUD para que vea como una opcion el hecho de usar software privativo en vez de software libre, se explica como esta
persona es libre de tomar la elección y de porque su entono, las personas y las circunstancias lo obligan o lo coaccionan
para que pueda tomar una decisión en el hecho de usar software privativo o Libre.\\

En La segunda sección se presenta una visión desde el punto de vista social, teniendo en cuenta la interacción del egresado 
del GLUD con las demás personas pertenecientes a las comunidad y su comportamiento dentro de las misma. En la sección que 
le sigue se presenta el aspecto ético, el cual se ve condicionado por las reglas de la ética que debe tener en cuenta un 
miembro del GLUD y un egresado del mismo.\\

Finalmente en el documento se presenta el tema principal que son la implicaciones que tienen para un egresado del GLUD el 
uso de software privativo como también las consecuencias de las mismas y el hecho de que al volverse miembro del grupo 
se adquieren unas responsabilidades relacionadas con el cumplimiento de las normas, la filosofía del software libre, 
la convivencia en sociedad, el construir comunidad y seguir promoviendo la filosofía del mismo aun cuando este fuera del GLUD. 



\sectiontext{white}{black}{LA ELECCIÓN} %como se hace una sección

``żQué hace hombre a un hombre? se preguntó un amigo mío alguna vez. żSon sus orígenes? La forma en que llegó a la vida? 
No lo creo. Son las elecciones que hace. No es la forma en que inicia las cosas, sino la forma en que decide poner fin 
a ellas''.\cite{hellboy} Esta frase dicha por un personaje de ficción hace pensar, żqué realmente es una construcción?, 
como mucha personas pienso que la evolución de todo ser (incluyendo al ser humano) depende principalmente del entono que 
lo rodea, teniendo en cuenta ese aspecto podemos mirar realmente que hace que las personas que nos rodean y las 
situaciones que se presentan ante nosotros sean tan influyentes de cierto modo.\\

Ahora tomando el tema del software libre y combinándolo con la frase que inicia el párrafo anterior tendríamos lo siguiente:\\
``żQue hace usuario del software libre a un usuario de software libre?''. żSon sus maestros?, żla forma como entra la mundo del 
software libre?, żlas personas que lo rodean?, żel software que usa?. Como saberlo, es como buscar una aguja en un pajar, pero, 
si lo que realmente lo hace ser un usuario de software son las decisiones que toma, su forma de hacer las cosas, la tendencia 
a hacer respetar la filosofía del software libre y a difundirla. En mi concepción del mundo está, que un partidario del software 
libre es aquella persona que no tiene miedo a caer una y mil veces con tal de aprender algo, pero de sus errores sabe hacer una 
construcción que realmente lo inspira a dar mas de su parte para superarlos y siempre los ve como una enseńanza. \\

Si miramos lo que realmente hace que un usuario del software libre tome o no una decisión entonces tenemos que tener en cuenta 
que hizo que tomara esa decisión, si fueron las circunstancias, si fue el medio, si fue una influencia negativa. Pero la decisión
siempre está en manos de dicho sujeto, de como se comporte, de que tan bueno sea su pensamiento con respecto a la filosofía del 
Software libre y su fuerza para poder afrontar las condiciones adversas que pone las sociedad para quienes entran en este mundo. 
Debido a que las personas que entran al software libre siempre están expuestos a los ataques que contra ellos propician los 
usuarios de software privativo (que particularmente siempre creen tener las razón), mientras que el usuario de software libre 
no busca convencer a nadie sino que expone unos argumentos, unos beneficios y espera a que la persona que lo escucha tome o no 
su propia decisión con respecto al software libre (claro que no falta el fanático del software libre que se cree dios y creé tener
la verdad revelada).


%%%%%%%%%%%%%%%%%%%%%%%%%%%%%%%%%%%%%%%%%%%%%%%%%%%%%%%%%%%%
\ebOpage{introcolor}{0.35}{OPINIÓN}
%%%%%%%%%%%%%%%%%%%%%%%%%%%%%%%%%%%%%%%%%%%%%%%%%%%%%%%%%%%%

\sectiontext{white}{black}{EL ASPECTO SOCIAL} %sección

Este aspecto comprende la interrelación del miembro del GLUD con la comunidad, misma que ha de desarrollarse de manera armónica y 
siempre con la finalidad de mejorar la visión que tienen las personas del grupo y la calidad de la enseńanza y difusión del software 
libre; así como la de ejercer una influencia positiva en el entorno. Si en general no se cumple este propósito el integrante o 
egresado del GLUD no ha aprovechado su estancia en el mismo para poder aprender el concepto de comunidad y las relaciones que 
entre los sujetos se establecen al vivir en tal.\\
Teniendo en cuenta lo dicho anteriormente entonces se dice que para un egresado o integrante del GLUD es de vital importancia 
poner en practica las enseńanzas que adquirió en el mismo para con todas las personas, sean componentes de su entorno y las que 
no conoce, de esta forma hace que el nombre del GLUD quede en alto siempre. También debe recordar no entrar en conflicto a causa 
de su filosofía del software libre, sino que tiene que recordar lo que realmente es el software libre y poner en claro los argumentos 
de los cuales esta se vale para producir un avance en las sociedad.\\
Por otra parte sin dejar de lado lo dicho anteriormente tomemos el aspecto del software privativo y la relación de uso que un egresado o
integrante del GLUD puede establecer con el mismo, de ser así este estaría incurriendo en un hecho que se puede ver desde distintas 
interpretaciones, la cuales se exponen a continuación:\\
\begin{itemize}
 \item \textit{El punto de vista radical:} Si no usa el software que se le dice se pueden generar malas interpretaciones,  rencillas 
 y disgustos con personas que se relacionan con usted. 
 \item \textit{El punto de vista diplomático:} En este aspecto usted usa el software propietario, pero no esta de acuerdo con la 
 filosofía del mismo y eventualmente deja de usarlo. 
 \item \textit{El punto de vista resignado:} Aquí las persona no solo usa el software privativo sino que acepta su filosofía y nunca lo abandona.
\end{itemize}

Estos son los diferentes aspectos sobre los cuales es posible ver el efecto que produce a nivel social el uso y la aceptación del 
software privativo por parte de uno de los egresados del GLUD, todo depende de la forma como la persona que ve las cosas las interprete,
esto se presenta como la relación que tiene el grupo con las personas que pasaron por él y ahora no están, en ese aspecto es necesario
que los egresados tengan en cuenta que son los representantes del GLUD en el mundo y que por este hecho es que deben hacer los posible por 
dar una buena impresión y dejar en alto la cara del GLUD y siempre tener claro que es la filosofía del software libre y no solo saber que es 
sino también seguirla.


%\begin{entradilla}
%{\em {\color{introcolor}{Latex}} facilito las cosas en el desarrollo del proyecto
%}
%\end{entradilla}


\sectiontext{white}{black}{El ASPECTO ÉTICO}

Este es uno de los aspectos más relevantes en el arte del software, en el que la profesión alcanza una alta dignidad. El desarrollador y 
el usuario en su quehacer, deben estar guiado por dos principios el amor a su comunidad y del amor a su arte. Esta idea un poco orientada 
hacia la filosofía de Hipocrates, esta plantea que es un aspecto en el que la ética se basa como un conjunto de las reglas que se dan como 
alternativas para los usuario y desarrolladores, de esta forma se hace evidente que el mismo amor al arte que estos desempeńan es el 
que hace que tomen o no alguna decisiones cruciales para su desempeńo ético y sumado a eso esta la lealtad que deben a su comunidad.\\

La teoría clásica acerca de la ética de la tecnología nos plantea que esta no puede ser considerada ni buena ni mala, sino que el resultado 
de lo que haga el hombre con la tecnología es lo que determina si la tecnología se usa de una forma buena o mala, entonces en este sistema 
clásico la tecnología no depende sino de lo que el hombre haga con ella, mas el hombre no esta condicionado por la misma. Mientras que en 
la teoría moderna se plantea con una idea menos aristotélica y mas platónica, por ejemplo que la tecnología es un medio por el cual se puede 
ejercer control a las personas y paso a citar:\\
``la educación es el principal elemento represivo, el medio más eficaz para el control, el más apropiado homogeneizador social. En la 
educación se hará al ciudadano: se condicionara su sensibilidad, su voluntad y su pensamiento, de modo que nada pueda desear sino aquella 
situación que <<por naturaleza>> le pertenece''\cite{platon}. Si analizamos esta frase y la vemos desde la situación actual, es claro que 
en la educación nos condicionan ``para'' y es en ese sentido donde algunos egresados miembros y demás vinculados con el GLUD fallan, 
por que en algún momento se dan a la tarea de pensar que la mejor forma de competir es sabiendo mucho y entran en ese pensamiento y terminan 
diciendo que no importan los medios por los cuales sea lo importante es ganar, ganar y ganar.\\ 

Esta ultima frase del párrafo anterior a la cual me refiero ahora es entonces la que genera una incomprensión y un pensamiento errado con 
respecto al software libre y allí migran los egresados del GLUD al software privativo porque supuestamente es mas competitivo y tienen 
mejores capacidades, pero eso no es cierto, porque es solo la tecnología que puede ser mas avanzada pero esta nos condiciona a usarla mientras 
que el software libre no condiciona a nadie a usarlo y por lo contrario le plantea las libertades, en conclusión es uso de software privativo 
por parte de un egresado del GLUD es condicionarse a las reglas que el distribuidor, el vendedor o hasta el mismo software en sus condiciones 
de uso y distribución le impongan, por ende los mejor es tener clara la filosofía del software libre y llevarla en la conciencia siempre.

%%%%%%%%%%%%%%%%%%%%%%%%%%%%%%%%%%%%%%%%%%%%%%%%%%%%%%%%%%%%
\ebOpage{introcolor}{0.35}{OPINIÓN}
%%%%%%%%%%%%%%%%%%%%%%%%%%%%%%%%%%%%%%%%%%%%%%%%%%%%%%%%%%%%

\sectiontext{white}{black}{LAS IMPLICACIONES}

El titulo de este documento habla acerca de las implicaciones que tienen para un egresado del GLUD usar software propietario, pero durante todo 
el documento no se han expuesto estas, ha sido de manera premeditada, ya que era pertinente primero dar algunas nociones introductorias y establecer
unos parámetros que permitieran a los lectores poder tener una visión mas acorde al caso y por ese motivo en las secciones anteriores se habla 
de las elecciones a las que se ve expuesto el egresado y miembro del grupo, de como su filosofía puede ser atacada y vulnerada. También se cometa 
acerca de los principales aspectos (porque hay muchos mas) que deben ser tenidos en cuenta en las decisiones tomadas por alguien que pertenece o 
perteneció al GLUD y por ultimo se dan una pequeńas conclusiones.\\

Antes de comenzar con el tema central de esto quisiera expresar que este es mi pensamiento personal acerca del tema y que no corresponde a un 
pensamiento global del GLUD, por tanto no debe culparse de la atrocidades que aquí consigno a las personas pertenecientes a este grupo mas que a mi.\\
Es evidente que dentro de los objetivos del GLUD no esta la enseńanza ni promoción de software privativo, por este motivo si alguno de los integrantes 
hace uso del mismo es responsabilidad enteramente de él mismo y de nadie mas (a menos que sea influenciado por alguien mas). En ese caso es el y 
su ética los que tienen la opción de decidir que es lo mejor para si mismo, con esto quiero decir que la persona que use software privativo 
después de haber estado o estando en el GLUD no lo hace porque en este se le haya enseńado esto y lo hace bajo su responsabilidad.\\

Tomando entonces el párrafo anterior como un referente tenemos que si alguno de los egresados o miembros del GLUD toman la decisión de usar software 
privativo estaría en una clara contradicción con lo aprendido en este grupo y no tendría porque hacerse llamar miembro o egresado del mismo, por 
el solo acto de conciencia, además de eso también esta poniendo en tela de juicio que su paso por el grupo fuese fructífero, mas allá de la sola 
explotación de los recursos y el talento humano que en este grupo encontró. Por que si este es el caso, entonces se presenta que no estaba allí 
por la enseńanza y el aprendizaje de lo que es la comunidad y el software libre, mas que por la adquisición de algunos conocimientos técnicos 
y los recursos de los que pudiera sacar provecho en su momento, entonces si el conocimiento que adquirió en el grupo no le sirve para la vida esto 
implica que su ética esta corrupta y que la codicia humana que habita en él es mucho mas grande que el conocimiento y el deseo de crecimiento 
en forma de comunidad, el compartir conocimiento, el desarrollar ideas para que los otros miembros de la comunidad se beneficien y así la comunidad 
se fortalezca.\\

Teniendo todo lo dicho anteriormente en cuenta de no ser por un caso extremo, si un egresado del GLUD llega a usar software privativo la implicación 
mas grave es que tendría que llevar en su conciencia todo el peso de lo que es traicionar las enseńanzas que el grupo se esforzó por transmitirle y 
defraudar la confianza que un día el grupo puso en él al permitirle que fuera uno mas de los miembros activos.

\sectiontext{white}{black}{CONCLUSIONES}
\begin{itemize}
 \item Los integrantes del GLUD adquieren responsabilidades como miembros activos que deben seguir ejerciendo aun cuando ya no lo sean y estas son 
 llevar siempre el buen nombre del grupo y no defraudar su filosofía.
 \item Muchas veces es importante ser fiel a la filosofía del software libre a pesar de las adversidades hasta cuando estas ya sean infranqueables y 
 siempre aceptando el punto de vista de los demás y aprendiendo de ellos.
 \item El software privativo es malo, es antisocial, debe desaparecer.
\end{itemize}


\bibliographystyle{abbrv}
\begin{bibliografia}
\bibitem{hellboy}
Gillermo.~del Toro, \emph{Hellboy}, 1.\hskip 1em plus
  0.5em minus 0.4em\relax USA: Columbia-Pictures, 2004.
\bibitem{platon}
A.~Platon, \emph{La Republica}, Platon Diálogos, Tomo~1 - Prologo, pg~6. \hskip 1em plus
  0.5em minus 0.4em\relax , Bogotá: Ediciones Universales, 1995.
\end{bibliografia}


\begin{biografia}{images/ensayo/autor.eps}{Wilmar Fernando Pineda Rojas} % ańadir fotografía tamańo [2.5 cm x 3.3 cm ]
Estudiante de Ingeniería Catastral y Geodesia en la universidad Distrital Francisco José de Caldas, cursa actualmente noveno semestre factorial, 
es activista del movimiento del software libre y aun cree que ``este cuento hay que lucharlo por la gente que lo que pasa es que estamos en malas manos, 
cree en la democracia, en la libertad y que el software libre es la solución a todos los problemas de la humanidad, eso es un norte demasiado largo''. 
En la actualidad se desempeńa como Miembro activo del GLUD (Grupo GNU/Linux de la Universidad Distrital Francisco José de Caldas), 
es participe de el proyecto SIGLA, la edición de la revista GLUD Magazine y colaborador de RadioGLUD.  
\end{biografia}


\end{multicols} %termina el entorno multicols
%\eOpage %comienza una pagina nueva

%\rput(7.5,-2.0){\resizebox{10cm}{!}{{\epsfbox{images/mi_articulo/salsilla.eps}}}}

%\clearpage
%\pagebreak
