% Este fichero es parte del Número 5 de la Revista Occam's Razor
% Revista Occam's Razor Número 5
%
% (c)  2010 The Occam's Razor Team
%
% Esta obra está bajo una licencia Reconocimiento 2.5 Espańa de Creative
% Commons. Para ver una copia de esta licencia, visite 
% http://creativecommons.org/licenses/by/2.5/es/
% o envie una carta a Creative Commons, 171 Second Street, Suite 300, 
% San Francisco, California 94105, USA.

%\rput(10.0,-19.0){\resizebox{10cm}{!}{{\epsfbox{images/general/feliz_2011.eps}}}}
\rput(8,-1.6){\resizebox{!}{5cm}{{\epsfbox{images/general/editorial.eps}}}}
\rput(0.0,-13){\resizebox{7cm}{35.0cm}{{\epsfbox{images/general/bar1.eps}}}}
\rput(0.4,-4.5){\resizebox{!}{4.8cm}{{\epsfbox{images/portada_editorial.eps}}}}
\rput(0.7,-26.5){\resizebox{!}{0.9cm}{{\epsfbox{images/general/licencia.eps}}}}


\begin{textblock}{9.2}(3,0)
\begin{flushright}
{\resizebox{!}{1cm}{\textsc{Editorial}}}

\vspace{2mm}

{\LARGE Welcome to the jungle\\ GLUD magazine!}

by GLUD
\end{flushright}
\end{textblock}

\vspace{4mm}


\definecolor{barcolor}{rgb}{0.044,0.09,0.16}

\begin{textblock}{30}(-1.5, -1)
\begin{minipage}{0.12\linewidth}
\sf\color{barcolor}
\begin{center}

\vspace{1cm}

\colorbox{black}{
{\resizebox{3cm}{0.7cm}{\textcolor{white}{\bf\sf\large GLUD}}}
}

{\resizebox{2.5cm}{0.4cm}{\bf\sf\large Magazine}}

\vspace{3mm}

{\bf Número 1. Septiembre 2011}

\vspace{5cm}

\hrule

\vspace{3mm}

{\bf Dirección: }

\vspace{1mm}

Wilmar Fernando Pineda 

\vspace{2mm}

{\bf Editor:}

\vspace{1mm}

Wilmar Fernando Pineda 

Y Otros

\vspace{4mm}

{\bf Comite Editorial:}

\vspace{1mm}

Es necesario llenar esta parte con algunos nombres de las personas que deseen colaborar con esto.


\vspace{4.0mm}

{\bf Maquetación y Grafismo}

\vspace{1mm}

Este espacio esta reservado para quien este interesado en hacer el desarrollor de la portada, la contraportada
 y además nos desee colaborar con el diseńo grafico que concierne a la revista. 

\vspace{1mm}

\hrule

\vspace{2mm}

{\bf Publicidad}

\vspace{1mm}

Tambien podemos hacer un poco de publicidad

{\tt occams-razor@uvigo.es}

\vspace{2mm}

\hrule

\vspace{4mm}

{\bf Impresión}

Por ahora tu mismo\ldots Si te apetece

\vspace{2mm}

\hrule

\vspace{6mm}

\copyright  2011 GLUD

Esta obra está bajo una licencia Reconocimiento 2.5 Espańa de Creative
Commons. Para ver una copia de esta licencia, visite 

{\scriptsize http://creativecommons.org/licenses/by/2.5/es/} 

o envie una carta a Creative Commons, 171 Second Street, Suite 300, San Francisco, California 94105, USA.

\medskip

%{\color{black}{\textbf{Consulta la página 32 para las excepciones a esta licencia}}}

\end{center}
\end{minipage}

\end{textblock}

\begin{textblock}{20}(3,2.0)

\begin{minipage}{.45\linewidth}
\colorbox{introcolor}{
\begin{minipage}{1\linewidth}

{{\resizebox{!}{1.0cm}{E}}{n el primer numero de la revista del GLUD (Grupo GNU/Linux de la universidad Distrital) 
quisiéramos presentar nuestros mas sinceros agradecimientos a los desarrolladores de la 
idea del código fuente de la misma, es decir a nuestros amigos los gestores y encargados 
de mantener en funcionamiento la revista Occam's Razor, los cuales con su aporte hacen 
posible que no solo este sino que otros proyectos se desarrollen como una alternativa a 
la edición de documentos y que mejor que documentos como una revista de divulgación. Para 
concluir con el reconocimiento a nuestras amigos de Occam's Razor queremos desearles 
muchos éxitos en sus próximos números, que no olviden revisar nuestro proyecto y gracias 
muchachos por su colaboración en el área de las fuentes de Latex. Y a los lectores que no 
olviden visitar la pagina oficial del proyecto que es: http://webs.uvigo.es/occams-razor.





\bigskip

}
}

\end{minipage}
}

\vspace{6mm}

Ahora pasando a lo concerniente al proyecto de GLUD Magazine que es un proyecto el cual no 
retoma la idea original de el padre de ser una revista de tecnología, sino que en esta 
enfatiza netamente en los conceptos de software  libre y de código abierto. Es decir los 
temas que son tratados en esta generalmente se hacen en sistemas GNU/Linux acerca de: desarrollo 
de software, proyectos culturales, difusión, instalación, configuración, apropiación, traducciones, 
capacitaciones y filosofía. Todos estos tema relacionados claro con los movimientos  de el software 
libre y el open source o código abierto en nuestro idioma.\\ 

La idea de la revista nace con la finalidad de dar voz a todos aquellos que no la tienen y que merecen
un espacio en donde puedan plasmar su conocimiento, para que el mundo no se pierda de lo mucho
que estas personas puede o podemos dar, por que si estas leyendo esto significa que tu y todos a
quienes puedas distribuir este mensaje están invitados a publicar en la revista del GLUD, solo recuerda
que no hay un conocimiento pequeńo y que siempre habrá en el mundo alguien que valore lo que
haces, así que vamos a colaborar con el proyecto ya que esta pequeńa que ahora tienes en las manos es
un  proyecto joven y necesita de muchos aportes para poder darle al mundo algo de el conocimiento
del que se ha perdido porque no había un espacio en el cual plasmarlo.\\

Así que si tienes un proyecto o cualquier tema relacionado con respecto al software y la cultura libre y
open source te invitamos a que te contactes con nosotros en glud@udistrital.edu.co y que visites
nuestro sitio en internet que es, http://glud.org. Y si no sabes mucho del tema te invitamos a continuar
leyendo ya que hay unos artículos a continuación que te pueden aclarar muchas cosas acerca del
mismo.\\ 

Así que no siendo mas nosotros el GLUD y los amigos de Occam's Razor te invitamos a que sigas leyendo 
esta publicación y esperamos que la disfrutes.\\

\bigskip

%Deseamos que la distrutéis y ....

%\medskip
%{\large Feliz 2011!}

\begin{flushright}
{\Large\sc{GLUD \\}}
\end{flushright}


\vspace{5cm}

\bigskip


\end{minipage}

\bigskip

\colorbox{introcolor}{
\begin{minipage}{.45\linewidth}

\bigskip

{\footnotesize\sf{\color{white}
Las opiniones expresadas en los artículos, así como los contenidos de
los mismos, son responsabilidad de los autores de éstos.

Puede obtener la versión electrónica de esta publicación, así como el
{\em código fuente} de la misma y los distintos ficheros de datos
asociados a cada artículo en el sitio web:

{\tt http://webs.uvigo.es/occams-razor}

}}

\bigskip

\end{minipage}
}


\end{textblock}

\pagebreak
