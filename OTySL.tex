% Esta obra está bajo una licencia Reconocimiento 2.5 Espańa de Creative
% Commons. Para ver una copia de esta licencia, visite 
% http://creativecommons.org/licenses/by/2.5/es/
% o envie una carta a Creative Commons, 171 Second Street, Suite 300, 
% San Francisco, California 94105, USA.

% Seccion Introducción
%

\rput(2.5,-2.3){\resizebox{!}{5.7cm}{{\epsfbox{images/OTySL/Kokopelli.eps}}}}%Imagen de el comienzo de el articulo, coordenadas desde 
                                                                                   %la parte superior izquierda del margen de la pagina

% -------------------------------------------------
% Cabecera
\begin{flushright}
\msection{introcolor}{black}{0.25}{OPINIÓN} %titulo de la sección

\mtitle{12cm}{LA INFORMACIÓN GEOGRÁFICA, EL SOFTWARE LIBRE Y EL ORDENAMIENTO TERRITORIAL} %titulo del articulo 

\msubtitle{8cm}{El software libre en el ordenamiento del territorio} %subtitulo

{\sf Por: Edda Camila Rodríguez Mojica} %autor

{\psset{linecolor=black,linestyle=dotted}\psline(-12,0)}
\end{flushright}

\vspace{2mm}
% -------------------------------------------------

\begin{multicols}{2}


% Introducción
\intro{introcolor}{E}{
l proceso de progreso en los ámbitos sociales y culturales se refleja en las intervenciones sobre
el territorio. En éste sentido se han generado dos intrumentos desde la parte social, el plan de desarrollo y desde la parte territorial y espacial,
 el plan de ordenamiento territorial.  Sin embargo la visión de la población es corta cuando desconocen el territorio  y el software libre, en éste sentido permite 
construir escenarios futuros, partiendo del conocimiento de la comunidad que se puede implementar desde la generación y aprobación de proyectos de inversión.
En resumen ésa es la idea que conforma el presente artículo, que debe ser enriquecido poco a poco con la experiencia que se obtenga.
}

\vspace{2mm}
% Cuerpo del artículo
\begin{entradilla} %codigo para una entradilla
{\em {\color{introcolor}{CONSTRUIR, SIEMPRE}} \begin{center}{``Que la gente pueda opinar no es suficiente, que pueda actuar es necesario, y que pueda actuar en aquello que le interesa, en su comunidad, en su barrio, en su municipio. Pero para poder actuar tiene que tener bases, instrumentos culturales y materiales''.
Estanislao Zuleta.}
\end{center}
}
\end{entradilla}

La gestión del territorio se logra a partir de las diferentes decisiones que pueden tomar los actores del mismo. Sin embargo desde que no se establezcan mecanismos para confrontar el conocimiento constante del espacio, las problemáticas sociales, ambientales y económicas, entonces la organización del territorio no alcanzará la expresión que se ha buscado; en consecuencia los escenarios deseados sólo se constituirán en un ideal ambiguo, que sirve únicamente para la propuesta de gobierno y no para  la construcción social y territorial. 
Los hechos y modos de actuar de las entidades estatales son claras con el manejo de la información geográfica, porque en realidad no es conveniente que se sepa del territorio sino lo mínimo. Por eso los pocos que se han puesto en la tarea de estudiarlo en Colombia con la minucia necesaria han encontrado problemas al describir lo que ellos ven.  Aun así hay información que debe estar disponible a escalas adecuadas, de manera que los diferentes interesados puedan acceder sin ningún problema, es el caso de los ríos, las zonas ecológicamente necesarias y no aptas de reconstrucción, parques naturales, zonas urbanas, veredas, zonas de riesgo, vías, zonas mineras, etc., ya sea con fines empresariales o por el proceso de generación de identidad territorial y de cumplimiento de los POT, que implica el reconocimiento de todos los municipios y de todas las problemáticas presentes.

La producción de la información geográfica requiere conocimientos de cómo adquirirla y plasmarla en un mapa, para que éste sea entendido por aquellos que tomarán las decisiones. Respetando el proceso de producción cartográfico żPor qué no disminuir la inversión en las licencias arcGIS, Erdas, PCI geomatics, y enfocar la inversión en la investigación y desarrollo en el software libre GIS? żPorqué no manejar la estructura de las bases de datos bajo software libre? Por otro lado żla información sería libre en algunos casos y bajo qué casos?.  El ordenamiento del territorio requiere de esa información, porque se constituye en la realidad del uso del suelo que permitirá definir los lineamientos a los cuáles se dirige la política de gobierno, sea a nivel nacional, departamental o municipal. Dentro del proceso de organización física del espacio el primer paso es la construcción de escenarios, y éstos deberían ser participativos, a lo cual sigue el estudio de proyectos serán que definirán la intervención territorial. 
 



\sectiontext{white}{black}{Construcción de escenarios  participativos desde el software Libre} %como se hace una sección


Los escenarios permiten anticiparse al futuro, identificar  dificultades, opciones y caminos posibles para llegar a fines determinados. En éste caso el análisis del futuro territorial es  una alternativa de generación de ventajas comparativas,  basándose en la  prospectiva más no en la proyección, argumentando que la primera tendrá en cuenta el pasado, pero no dependerá de éste para explicar los escenarios futuros. Por lo tanto el escenario es la herramienta que permitirá generar estrategias de desarrollo y ejecución, dentro del proceso  participación política y de intervención que se ha ido restableciendo nuevamente.  El reconocimiento del territorio es necesario para saber cómo se ve el país desde el punto de vista físico, con respecto a los habitantes. 
%%%%%%%%%%%%%%%%%%%%%%%%%%%%%%%%%%%%%%%%%%%%%%%%%%%%%%%%%%%%
\ebOpage{introcolor}{0.35}{OPINIÓN}
%%%%%%%%%%%%%%%%%%%%%%%%%%%%%%%%%%%%%%%%%%%%%%%%%%%%%%%%%%%%
Pero el país es descentralizado, por lo tanto las responsabilidades recaen en últimas en el mínimo espacio de desarrollo : los municipios.  
Entonces a nivel municipal generar la información territorial desde el software libre para que otros puedan distribuirla, modificarla, mejorarla y actualizarla es el proceso que debe articularse para obtener así mayor conciencia de lo que el territorio significa. La significación del territorio debe comprenderse desde el espacio dinámico y  la criticidad en algunos aspectos como el económico. En este sentido la parte educativa debe efectuarse de manera que se puedan construir mapas fácilmente e identificar zonas  de interés. Esta parte es fundamental. La construcción de escenarios y de argumentos culturales e incluso políticos es difícil de realizar. Esta construcción debe estar guiada y debe ser amplia en puntos de vista, no cerrarse a las opciones de diversidad de pensamiento y de soluciones, en general con éste mecanismo se podrán identificar puntos comunes que permitan diagnosticar los problemas territoriales.  
Es claro que si bien los escenarios pueden implicar múltiples características, éstos deben separarse según la desagregación del territorio en subsitemas para poder construir bases fuertes con los cuales ejecutar las decisiones: subsistema natural, subsitema social, subsistema económico y subsitema urbano-regional (rural y suburbano). Con respecto a esto la información debe disponerse para cualquier persona.
Lo ideal en la construcción de escenarios como procedimiento es el siguiente:
\begin{enumerate}
\item Reconocimiento del territorio y sus problemáticas por subsistemas: En éste proceso se trata de incluir un espacio de participación política desde las empresas y desde la vida cotidiana de las personas, es decir la generación de información geográfica puede estar a cargo de éstas personas a manera de taller. Se plantea nivel rural y urbano. Para esto puede crearse a partir del software libre mapas online para que la gente los recree, y para que en algunos casos la información esté protegida, pero disponible. 
\item Caracterización histórica de los procesos. Dentro de los subsistemas las tendencias deben ser visibles y sus análisis permitirían observan las posibles consecuencias de las mismas, así como lo que originaron en el pasado.
\item Identificación de fines comunes y de posibilidades aptas: después de analizar la información adquirida y los puntos de vista de la población, que es suficiente para saber qué escenario es el adecuado, se busca responder a la pregunta żqué se puede realizar y que no? żqué posibilidades hay y cuales son imposibles de conseguir? 
\end{enumerate}
Con respecto al proceso anterior, éste debe ser realizado por los alcaldes para saber qué se debe plantear como prioritario dentro de cualquier período gestión. Desde el enfoque de participación dado es necesario que se obtengan los recursos adecuados.
Las ventajas del procedimiento anterior y de la conceptualización obligatoria que se requiere dentro de lo planteado :
\begin{itemize}
 \item Desde la moral y ética, la población adquiere formas de pensar diferentes, acercándose al territorio como el actor de cambio, introduciendo la identidad del territorio a su vida cotidiana y aprendiendo poco a poco a respetar al mismo.
 \item Profundización y expansión en la filosofía de software libre: En general si el software libre es la base para la generación de información geográfica y todo el mundo  puede usarla, modificarla y redistribuirla, se observa que poco a poco la nueva información y los procesos investigativos de progreso pueden encaminarse.
 \item Por parte de los gobernantes tienen asegurado el conocimiento de lo que la gente quiere. Sin embargo ésta participación en la construcción del escenario no es suficiente para que las personas se inmiscuyan en los diferentes procesos políticos que se pueden desempeńar dentro del municipio, sino que deben estar complementados con talleres constantes en problemáticas actuales de su territorio. Por ejemplo en procesos medioambientales y mineros, es el caso de Tasco, ża cuanta gente le preguntaron si preferían el páramo o la mina?
\end{itemize}
Sin embargo lo anterior es idealismo y no realidad. La realidad lamentablemente depende de aquellos que poseen los recursos para la inversión en el territorio, y hasta que éstos intervengan de manera comunitaria, realmente se verán los resultados. 


\sectiontext{white}{black}{De los escenarios al ordenamiento territorial} %sección

 Dentro de las decisiones que toma el alcalde en el municipio, los proyectos son el principal mecanismo de ejecución de actividades. Los proyectos tienen que ser analizados y la corrupción debe evitarse en la mayor medida posible; por otro lado deben ser coherentes con el plan de desarrollo que es la parte social del plan de gobierno y con el plan de ordenamiento territorial , donde se plasmarán las zonas ambientalmente protegidas y que no son susceptibles a intervención, las zonas que por su importancia cultural, histórica y patrimonial  tampoco tienen posibilidad de manejo, siempre y cuando no sea la conservación, zonas de  servicios indispensables (salud, educación y vivienda), zonas de alto riesgo(vulnerabilidad y amenaza), zonas aptas para la ocupación, etc., que deben ser concordantes con los aspectos planteados para el desarrollo según el gobierno. 

%%%%%%%%%%%%%%%%%%%%%%%%%%%%%%%%%%%%%%%%%%%%%%%%%%%%%%%%%%%%
\ebOpage{introcolor}{0.35}{OPINIÓN}
%%%%%%%%%%%%%%%%%%%%%%%%%%%%%%%%%%%%%%%%%%%%%%%%%%%%%%%%%%%%

En   Colombia los proyectos hacen parte de un plan, y éste se encuentra sujeto a un programa, sin embargo las actividades dependen de los resultados  parciales del proyecto, por lo que el proyecto es el generador de las intervenciones del territorio, estos últimos  surgen  de las siguientes entidades: de la administración de la alcaldía, de la comunidad, de las organizaciones gremiales, de las ONG.
Cualquier proyecto, afecta a alguno de los subsistemas anteriormente mencionados, por lo que se requerirá la información geográfica para los correspondientes estudios, y ésta información no siempre está disponible porque el único ente que la tiene es el IGAC. En éste sentido la información de conseguirse libre y se debe manejar con software libre, porque se debe hacer con software legal, y esto disminuiría los gastos para las empresas como para los generadores de la IG.

%\begin{entradilla}
%{\em {\color{introcolor}{Latex}} facilito las cosas en el desarrollo del proyecto
%}
%\end{entradilla}


\sectiontext{white}{black}{Conclusiones}
\begin{itemize}
 \item Los procesos de generación de los POT y los PD requieren siempre una construcción de escenarios participativos, con fines de generación de identidad territorial, en los cuales la información geográfica se recrea, se actualiza y se presenta como información general de uso libre. Este tipo de recopilación, alimenta las bases de datos y además integra las personas al territorio, y permiten entrever la significación e importancia que éste adquiere para cualquier solución e intervención territorial.
 \item El proceso de identificación de fortalezas del software libre en el manejo de la información es amplia, sin embargo sino se intervienen con proyectos dentro de las administraciones, los procesos no se llevarán a cabo. En éste sentido se respeta lo que ha intentado hacer casa del bosque, pero no se apoya la exclusión de pequeńas comunidades de software libre, cualquiera que sea, y mucho menos se apoya la generación de problemáticas que puedan desempeńarse a través de los diferentes procesos que con el SL nazcan.
 \item Dentro de las expectativas que se deben plantear, la idea es implementar el software libre por pasos,  y uno de esos pasos es que se intervengan en las administraciones con proyectos de éste tipo. En este sentido si se genera la posibilidad por medio de una rosca, pues es una buena oportunidad, y no hay que dejarla pasar.
 \item Por otro lado la comunidad es una de la originadoras de las ideas de proyectos como seńalé anteriormente y se pueden conquistar espacios en los cuales se pueda intervenir a través de las estrategias computacionales, tratando de no perder la idea de poder tener un mundo donde el software y  en general el conocimiento intelectual no sea para unos pocos, sino para la generalidad de la población.
\end{itemize}



\bibliographystyle{abbrv}
\begin{bibliografia}
\bibitem{Secretaría de desarrollo social,Secretaría de medio ambiente y recursos naturales,instituto nacional de ecología, Universidad autónoma de México} Indicadores para la
caracterización y ordenamiento del territorio.
\emph{}, 1.\hskip 1em plus
 0.5em minus 0.4em

\bibitem{Anónimo}
\emph{Documento de contextualización polítca}
\bibitem{JUAN UGUARTE CORTES}
 \emph{Teoría general del Municipio: Fines, fundamento y estructura, 2003}
\hskip 1em plus
0.5em minus 0.4em 
\bibitem{ERNESTO FIRMENICH BIANCHI}
\emph{Metodología para la construcción de escenarios,2008} \hskip 1em plus
0.5em minus 0.4em
\bibitem{JULLY ELENA BOLAŃOS LOPEZ}
\emph{Manual de procedimientos: Banco de programas y proyectos de Inversión pública del municipio la Victoria, Boyacá,2010.}
\hskip 1em plus
0.5em minus 0.4em
\bibitem{Revisado por Gabriel Suarez}
\emph{Abc del POT Bogotá, 2009}
\hskip 1em plus
0.5em minus 0.4em

\end{bibliografia}


\begin{biografia}{images/OTySL/autor.eps}{Edda Camila Rodriguez Mojica} % ańadir fotografía tamańo [2.5 cm x 3.3 cm ]
Estudiante de Ingeniería Catastral y Geodesia en la universidad Distrital Francisco José de Caldas, que se encuentra desempeńando actividades de trabajo de grado
y está interesada en el software libre como potencialidad que se debe desarrollar y desempeńar en cualquier carrera para que se obtenga una construcción social y progresista.
Hace parte del GLUD, del cual se ha enamorado, y piensa que es una opción bonita de contribuir a mejorar el país.  
\end{biografia}


\end{multicols} %termina el entorno multicols
%\eOpage %comienza una pagina nueva

%\rput(7.5,-2.0){\resizebox{10cm}{!}{{\epsfbox{images/mi_articulo/salsilla.eps}}}}

%\clearpage
%\pagebreak
