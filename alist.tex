% Esta obra está bajo una licencia Reconocimiento 2.5 Espańa de Creative
% Commons. Para ver una copia de esta licencia, visite 
% http://creativecommons.org/licenses/by/2.5/es/
% o envie una carta a Creative Commons, 171 Second Street, Suite 300, 
% San Francisco, California 94105, USA.

% Seccion Introducción
%

\rput(2.5,-2.3){\resizebox{!}{5.7cm}{{\epsfbox{images/mi_articulo/Kokopelli.eps}}}}%Imagen de el comienzo de el articulo, coordenadas desde 
                                                                                   %la parte superior izquierda del margen de la pagina

% -------------------------------------------------
% Cabecera
\begin{flushright}
\msection{introcolor}{black}{0.25}{ENSAYO} %titulo de la sección

\mtitle{10cm}{Libertad, realidad o ilusión} %titulo del articulo 

\msubtitle{8cm}{Libertad de la información en Colombia} %subtitulo

{\sf por List Cabanzo Ivan Alejandro} %autor

{\psset{linecolor=black,linestyle=dotted}\psline(-12,0)}
\end{flushright}

\vspace{2mm}
% -------------------------------------------------

\begin{multicols}{2}


% Introducción
\intro{introcolor}{D}{esde pequeńos, queremos saber todo acerca del mundo 
que nos rodea, żCómo funciona esto?, żQué es aquello?, żPor qué sucede esto?...  
y con el tiempo vamos accediendo al limitado conocimiento que nos brindan, 
desde el colegio hasta el trabajo, siempre se aprende algo nuevo, muchas
personas se quedan con la idea  que obtuvieron  por sus superiores, 
los medios de comunicación, algún amigo o familiar, no es de muchos introducirse
en un tema y saciarse con todo lo que ha aprendido, causando que la censura
sea un acto indiferente, de hecho, desconocido.
}

\vspace{2mm}

% Cuerpo del artículo

Nuestro país, abundante de sabiduría y tradiciones generadas por antepasados, se ha dejado
llevar por falsas promesas, cegados por falsas informaciones que dan en los medios y
la indiferencia de quienes saben la verdad, se podría  declarar una Colombia parcialmente censurada.
Cuando un colombiano promedio observa alguna oposición en contra de alguna Ley mal hecha,
lo primero que piensa es en él mismo, - Ąpara que hacer ello si nada logramos!, 
- Ąlo único que ellos hacen es perturbar la tranquilidad!, - Ąotro trancón más! , 
basándose en lo poco informado que esta dicha persona, el statu quo (todo esta bien, no pongas atención) 
impide saber que esta pasando en realidad.\\

Afortunadamente, no todas las personas se limitan a este tipo de información, una verdad a medias no 
aporta nada, el estudio y ardúo esfuerzo de varias comunidades, ha permitido continuar con la poca
libertad que hay en internet, y digo poca porque muchas paginas importantes se encuentran bloqueadas por
las ISP, y, aunque los autoritarios en Colombia no son muy competentes tienen claro su objetivo; dentro de
todas esas comunidades, me enfocaré en una en especial, aquella que nació gracias a la colaboración y el
trabajo mutuo, que permite que internet exista, aquella comunidad que no vive entre las sombras como
muchos creen: los hackers, los verdaderos, no los vándalos ni bandidos, son aquellos que son muy buenos
en lo que hacen, ellos hicieron de internet su herramienta más valiosa.\\



\begin{entradilla} %codigo para una entradilla
{\em {\color{introcolor}{Los Auténticos Programadores}} llamados así en su inicio, provenían
habitualmente de disciplinas como la ingeniería o la física y con frecuencia se trataba de radioaficionados.\cite{kopka1}}
\end{entradilla}

\sectiontext{white}{black}{Las malas leyes} %como se hace una sección

La información tiene valor, todo tiene un precio ? si bien puede ser cierto o no, quienes controlan el tipo 
de información que se publica en nuestro país tienen claro este concepto, existen ciertos datos que pueden 
llegar a ser ?perjudiciales? pero no precisamente para nosotros, por ello se ha intervenido en la radio, la
televisión y la prensa; el Internet en cambio se ha mantenido relativamente  de pie, y  digo relativamente por
que a pesar de tener acceso a datos importantes, que dan paso a la libertad de expresión, poco a poco se van
privatizando con leyes como SOPA, PIPA, ACTA o la misma Ley Lleras, financiadas por corporaciones 
para satisfacer sus ?proyectos? para controlar Internet.\\

El Internet no puede tener un dueńo, desde sus inicios como ArpaNet se pensó como una herramienta para compartir 
el saber,(independientemente de su uso) pese a su noble creación, entes (que a partir de ahora los llamaré como autoritarios) 
quieren poner precio sobre el contenido que circula por la red de redes, ocultar todo lo que le perjudique,
saber todo absolutamente de todos; En ese sentido, la poca libertad que tenemos se quiere exterminar, 
żpuedo sentirme libre sabiendo que estoy siendo vigilado todo el tiempo o se me esté negando información útil? Muy seguramente 
la respuesta será negativa, a nadie le gusta ser sometido a espionaje (lo privado es mio y solo mio),  y tenemos derecho a 
saber la verdad. sin embargo, no se puede negar que existen ciertos contenidos  que deben publicarse con restricciones, 
(libertad no tes igual a libertinaje). \\

Los autoritarios prosperan en la censura y el secreto. Y desconfían de la cooperación voluntaria y del intercambio de información
?sólo les agrada la cooperación que tienen bajo su control. La libertad es buena, los nińos necesitan guía, y los criminales,
restricciones; por tanto se puede estar de acuerdo en aceptar algún tipo de autoridad.\\


%%%%%%%%%%%%%%%%%%%%%%%%%%%%%%%%%%%%%%%%%%%%%%%%%%%%%%%%%%%%
\ebOpage{introcolor}{0.35}{INTRODUCCIÓN}
%%%%%%%%%%%%%%%%%%%%%%%%%%%%%%%%%%%%%%%%%%%%%%%%%%%%%%%%%%%%

\sectiontext{white}{black}{Las salidas de la censura} %sección


% A continuación un ejemplo de como se puede hacer una entradilla de código en 
% lenguaje C, de parte de los amigos de occam's razor.

%\lstset{language=C,frame=tb,framesep=5pt,basicstyle=\footnotesize}   
%\begin{lstlisting}
%#include <osgDB/ReadFile>
%#include <osgViewer/Viewer>
%
%using namespace osgDB;
%int main(int ac, char **a)
%{
%  osgViewer::Viewer viewer;
%  viewer.setSceneData(readNodeFile(a[1]));
%
%  return viewer.run();
%}
%\end{lstlisting}

Dejar ese pensamiento individualista es un paso bastante grande, se esta atacando su libertad, mi libertad,
la libertad de todos los que te rodean, ser indiferente es apoyar (sin saberlo) al dominio del contenido de Internet;
hacktivistas siempre demuestran su esfuerzo en contra de ello, pero la mayoría no son ?crakers?, ni delincuentes,
son personas que están cocientes  de esta situación, no necesito una mascara para estar en contra de estas acciones,
infórmate bien, no te quedes con lo poco que te dan, despierta ese interés que se ha perdido de pequeńo,
siempre habrá algo más que aprender.







% A continuación un ejemplo de como se puede hacer una entradilla de código en 
% lenguaje Make, de parte de los amigos de occam's razor.


%\lstset{language=Make,frame=tb,framesep=5pt,basicstyle=\footnotesize}   
%\begin{lstlisting}
%MY_CFLAGS=-I${DEV_DIR}/include
%MY_OSG_LIBS=-L${DEV_DIR}/lib -losgViewer
%
%mini: mini-viewer.cpp
%	g++ -o $@ $< ${MY_CFLAGS} ${MY_OSG_LIBS}
%\end{lstlisting}

\sectiontext{white}{black}{LA REALIZACIÓN}

Ejemplo de una tabla\\

\begin{tabular}{|>{\columncolor{encabezado}} c |>{\columncolor{introcolor}} c |>{\columncolor{introcolor}} c |>{\columncolor{introcolor}} c |>{\columncolor{introcolor}} c |}
\hline
\multicolumn{5}{|>{\columncolor{encabezado}}c|}{multicolumna 1-2}\\
\hline
\rowcolor{encabezado}Col 1 & Col 2 & Col 3 & Col 4 & Col 5 \\
\hline
rgb & cmyk & gray & predefinido & definido\\ \hline
\end{tabular}

Para que la libertad florezca, el Internet debe mantenerse libre del control gubernamental y  corporativo;
Pero como hemos visto con la DMCA, las corporaciones que quieren controlar la red tienen que hacerlo mediante
la compra de malas leyes del gobierno (En Colombia la Ley Lleras), por tanto, el frente más importante en la 
batalla sigue siendo la partida de malas leyes y reglamentos.


\bibliographystyle{abbrv}
\begin{bibliografia}
\bibitem{kopka}
H.~Kopka and P.~W. Daly, \emph{A Guide to \LaTeX}, 3rd~ed.\hskip 1em plus
  0.5em minus 0.4em\relax Harlow, England: Addison-Wesley, 1999.
\bibitem{kopka1}
A.~Kopka and P.~W. Daly, \emph{A Guide to \LaTeX}, 3rd~ed.\hskip 1em plus
  0.5em minus 0.4em\relax Harlow, England: Addison-Wesley, 1999.
\bibitem{kopka2}
B.~Kopka and P.~W. Daly, \emph{A Guide to \LaTeX}, 3rd~ed.\hskip 1em plus
  0.5em minus 0.4em\relax Harlow, England: Addison-Wesley, 1999.
\end{bibliografia}


\begin{biografia}{images/mi_articulo/autor.eps}{Iván Alejandro List Cabanzo} % ańadir fotografía tamańo [2.5 cm x 3.3 cm ]
Futuro desarrollador de software libre, Hoy estudiante de la Universidad Distrital, la curiosidad es algo que nunca me falta sobre todo a la hora de hablar de informática, apasionado a la seguridad informática desde pequeńo, seguidor de la filosofía GNU/Linux. El único limite del ser humano es su imaginación
\end{biografia}

\end{multicols} %termina el entorno multicols
%\eOpage %comienza una pagina nueva

%\rput(7.5,-2.0){\resizebox{10cm}{!}{{\epsfbox{images/mi_articulo/salsilla.eps}}}}

\clearpage
\pagebreak
